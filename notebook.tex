
% Default to the notebook output style

    


% Inherit from the specified cell style.




    
\documentclass[11pt]{article}

    
    
    \usepackage[T1]{fontenc}
    % Nicer default font (+ math font) than Computer Modern for most use cases
    \usepackage{mathpazo}

    % Basic figure setup, for now with no caption control since it's done
    % automatically by Pandoc (which extracts ![](path) syntax from Markdown).
    \usepackage{graphicx}
    % We will generate all images so they have a width \maxwidth. This means
    % that they will get their normal width if they fit onto the page, but
    % are scaled down if they would overflow the margins.
    \makeatletter
    \def\maxwidth{\ifdim\Gin@nat@width>\linewidth\linewidth
    \else\Gin@nat@width\fi}
    \makeatother
    \let\Oldincludegraphics\includegraphics
    % Set max figure width to be 80% of text width, for now hardcoded.
    \renewcommand{\includegraphics}[1]{\Oldincludegraphics[width=.8\maxwidth]{#1}}
    % Ensure that by default, figures have no caption (until we provide a
    % proper Figure object with a Caption API and a way to capture that
    % in the conversion process - todo).
    \usepackage{caption}
    \DeclareCaptionLabelFormat{nolabel}{}
    \captionsetup{labelformat=nolabel}

    \usepackage{adjustbox} % Used to constrain images to a maximum size 
    \usepackage{xcolor} % Allow colors to be defined
    \usepackage{enumerate} % Needed for markdown enumerations to work
    \usepackage{geometry} % Used to adjust the document margins
    \usepackage{amsmath} % Equations
    \usepackage{amssymb} % Equations
    \usepackage{textcomp} % defines textquotesingle
    % Hack from http://tex.stackexchange.com/a/47451/13684:
    \AtBeginDocument{%
        \def\PYZsq{\textquotesingle}% Upright quotes in Pygmentized code
    }
    \usepackage{upquote} % Upright quotes for verbatim code
    \usepackage{eurosym} % defines \euro
    \usepackage[mathletters]{ucs} % Extended unicode (utf-8) support
    \usepackage[utf8x]{inputenc} % Allow utf-8 characters in the tex document
    \usepackage{fancyvrb} % verbatim replacement that allows latex
    \usepackage{grffile} % extends the file name processing of package graphics 
                         % to support a larger range 
    % The hyperref package gives us a pdf with properly built
    % internal navigation ('pdf bookmarks' for the table of contents,
    % internal cross-reference links, web links for URLs, etc.)
    \usepackage{hyperref}
    \usepackage{longtable} % longtable support required by pandoc >1.10
    \usepackage{booktabs}  % table support for pandoc > 1.12.2
    \usepackage[inline]{enumitem} % IRkernel/repr support (it uses the enumerate* environment)
    \usepackage[normalem]{ulem} % ulem is needed to support strikethroughs (\sout)
                                % normalem makes italics be italics, not underlines
    

    
    
    % Colors for the hyperref package
    \definecolor{urlcolor}{rgb}{0,.145,.698}
    \definecolor{linkcolor}{rgb}{.71,0.21,0.01}
    \definecolor{citecolor}{rgb}{.12,.54,.11}

    % ANSI colors
    \definecolor{ansi-black}{HTML}{3E424D}
    \definecolor{ansi-black-intense}{HTML}{282C36}
    \definecolor{ansi-red}{HTML}{E75C58}
    \definecolor{ansi-red-intense}{HTML}{B22B31}
    \definecolor{ansi-green}{HTML}{00A250}
    \definecolor{ansi-green-intense}{HTML}{007427}
    \definecolor{ansi-yellow}{HTML}{DDB62B}
    \definecolor{ansi-yellow-intense}{HTML}{B27D12}
    \definecolor{ansi-blue}{HTML}{208FFB}
    \definecolor{ansi-blue-intense}{HTML}{0065CA}
    \definecolor{ansi-magenta}{HTML}{D160C4}
    \definecolor{ansi-magenta-intense}{HTML}{A03196}
    \definecolor{ansi-cyan}{HTML}{60C6C8}
    \definecolor{ansi-cyan-intense}{HTML}{258F8F}
    \definecolor{ansi-white}{HTML}{C5C1B4}
    \definecolor{ansi-white-intense}{HTML}{A1A6B2}

    % commands and environments needed by pandoc snippets
    % extracted from the output of `pandoc -s`
    \providecommand{\tightlist}{%
      \setlength{\itemsep}{0pt}\setlength{\parskip}{0pt}}
    \DefineVerbatimEnvironment{Highlighting}{Verbatim}{commandchars=\\\{\}}
    % Add ',fontsize=\small' for more characters per line
    \newenvironment{Shaded}{}{}
    \newcommand{\KeywordTok}[1]{\textcolor[rgb]{0.00,0.44,0.13}{\textbf{{#1}}}}
    \newcommand{\DataTypeTok}[1]{\textcolor[rgb]{0.56,0.13,0.00}{{#1}}}
    \newcommand{\DecValTok}[1]{\textcolor[rgb]{0.25,0.63,0.44}{{#1}}}
    \newcommand{\BaseNTok}[1]{\textcolor[rgb]{0.25,0.63,0.44}{{#1}}}
    \newcommand{\FloatTok}[1]{\textcolor[rgb]{0.25,0.63,0.44}{{#1}}}
    \newcommand{\CharTok}[1]{\textcolor[rgb]{0.25,0.44,0.63}{{#1}}}
    \newcommand{\StringTok}[1]{\textcolor[rgb]{0.25,0.44,0.63}{{#1}}}
    \newcommand{\CommentTok}[1]{\textcolor[rgb]{0.38,0.63,0.69}{\textit{{#1}}}}
    \newcommand{\OtherTok}[1]{\textcolor[rgb]{0.00,0.44,0.13}{{#1}}}
    \newcommand{\AlertTok}[1]{\textcolor[rgb]{1.00,0.00,0.00}{\textbf{{#1}}}}
    \newcommand{\FunctionTok}[1]{\textcolor[rgb]{0.02,0.16,0.49}{{#1}}}
    \newcommand{\RegionMarkerTok}[1]{{#1}}
    \newcommand{\ErrorTok}[1]{\textcolor[rgb]{1.00,0.00,0.00}{\textbf{{#1}}}}
    \newcommand{\NormalTok}[1]{{#1}}
    
    % Additional commands for more recent versions of Pandoc
    \newcommand{\ConstantTok}[1]{\textcolor[rgb]{0.53,0.00,0.00}{{#1}}}
    \newcommand{\SpecialCharTok}[1]{\textcolor[rgb]{0.25,0.44,0.63}{{#1}}}
    \newcommand{\VerbatimStringTok}[1]{\textcolor[rgb]{0.25,0.44,0.63}{{#1}}}
    \newcommand{\SpecialStringTok}[1]{\textcolor[rgb]{0.73,0.40,0.53}{{#1}}}
    \newcommand{\ImportTok}[1]{{#1}}
    \newcommand{\DocumentationTok}[1]{\textcolor[rgb]{0.73,0.13,0.13}{\textit{{#1}}}}
    \newcommand{\AnnotationTok}[1]{\textcolor[rgb]{0.38,0.63,0.69}{\textbf{\textit{{#1}}}}}
    \newcommand{\CommentVarTok}[1]{\textcolor[rgb]{0.38,0.63,0.69}{\textbf{\textit{{#1}}}}}
    \newcommand{\VariableTok}[1]{\textcolor[rgb]{0.10,0.09,0.49}{{#1}}}
    \newcommand{\ControlFlowTok}[1]{\textcolor[rgb]{0.00,0.44,0.13}{\textbf{{#1}}}}
    \newcommand{\OperatorTok}[1]{\textcolor[rgb]{0.40,0.40,0.40}{{#1}}}
    \newcommand{\BuiltInTok}[1]{{#1}}
    \newcommand{\ExtensionTok}[1]{{#1}}
    \newcommand{\PreprocessorTok}[1]{\textcolor[rgb]{0.74,0.48,0.00}{{#1}}}
    \newcommand{\AttributeTok}[1]{\textcolor[rgb]{0.49,0.56,0.16}{{#1}}}
    \newcommand{\InformationTok}[1]{\textcolor[rgb]{0.38,0.63,0.69}{\textbf{\textit{{#1}}}}}
    \newcommand{\WarningTok}[1]{\textcolor[rgb]{0.38,0.63,0.69}{\textbf{\textit{{#1}}}}}
    
    
    % Define a nice break command that doesn't care if a line doesn't already
    % exist.
    \def\br{\hspace*{\fill} \\* }
    % Math Jax compatability definitions
    \def\gt{>}
    \def\lt{<}
    % Document parameters
    \title{Business Intelligence}
    
    
    

    % Pygments definitions
    
\makeatletter
\def\PY@reset{\let\PY@it=\relax \let\PY@bf=\relax%
    \let\PY@ul=\relax \let\PY@tc=\relax%
    \let\PY@bc=\relax \let\PY@ff=\relax}
\def\PY@tok#1{\csname PY@tok@#1\endcsname}
\def\PY@toks#1+{\ifx\relax#1\empty\else%
    \PY@tok{#1}\expandafter\PY@toks\fi}
\def\PY@do#1{\PY@bc{\PY@tc{\PY@ul{%
    \PY@it{\PY@bf{\PY@ff{#1}}}}}}}
\def\PY#1#2{\PY@reset\PY@toks#1+\relax+\PY@do{#2}}

\expandafter\def\csname PY@tok@w\endcsname{\def\PY@tc##1{\textcolor[rgb]{0.73,0.73,0.73}{##1}}}
\expandafter\def\csname PY@tok@c\endcsname{\let\PY@it=\textit\def\PY@tc##1{\textcolor[rgb]{0.25,0.50,0.50}{##1}}}
\expandafter\def\csname PY@tok@cp\endcsname{\def\PY@tc##1{\textcolor[rgb]{0.74,0.48,0.00}{##1}}}
\expandafter\def\csname PY@tok@k\endcsname{\let\PY@bf=\textbf\def\PY@tc##1{\textcolor[rgb]{0.00,0.50,0.00}{##1}}}
\expandafter\def\csname PY@tok@kp\endcsname{\def\PY@tc##1{\textcolor[rgb]{0.00,0.50,0.00}{##1}}}
\expandafter\def\csname PY@tok@kt\endcsname{\def\PY@tc##1{\textcolor[rgb]{0.69,0.00,0.25}{##1}}}
\expandafter\def\csname PY@tok@o\endcsname{\def\PY@tc##1{\textcolor[rgb]{0.40,0.40,0.40}{##1}}}
\expandafter\def\csname PY@tok@ow\endcsname{\let\PY@bf=\textbf\def\PY@tc##1{\textcolor[rgb]{0.67,0.13,1.00}{##1}}}
\expandafter\def\csname PY@tok@nb\endcsname{\def\PY@tc##1{\textcolor[rgb]{0.00,0.50,0.00}{##1}}}
\expandafter\def\csname PY@tok@nf\endcsname{\def\PY@tc##1{\textcolor[rgb]{0.00,0.00,1.00}{##1}}}
\expandafter\def\csname PY@tok@nc\endcsname{\let\PY@bf=\textbf\def\PY@tc##1{\textcolor[rgb]{0.00,0.00,1.00}{##1}}}
\expandafter\def\csname PY@tok@nn\endcsname{\let\PY@bf=\textbf\def\PY@tc##1{\textcolor[rgb]{0.00,0.00,1.00}{##1}}}
\expandafter\def\csname PY@tok@ne\endcsname{\let\PY@bf=\textbf\def\PY@tc##1{\textcolor[rgb]{0.82,0.25,0.23}{##1}}}
\expandafter\def\csname PY@tok@nv\endcsname{\def\PY@tc##1{\textcolor[rgb]{0.10,0.09,0.49}{##1}}}
\expandafter\def\csname PY@tok@no\endcsname{\def\PY@tc##1{\textcolor[rgb]{0.53,0.00,0.00}{##1}}}
\expandafter\def\csname PY@tok@nl\endcsname{\def\PY@tc##1{\textcolor[rgb]{0.63,0.63,0.00}{##1}}}
\expandafter\def\csname PY@tok@ni\endcsname{\let\PY@bf=\textbf\def\PY@tc##1{\textcolor[rgb]{0.60,0.60,0.60}{##1}}}
\expandafter\def\csname PY@tok@na\endcsname{\def\PY@tc##1{\textcolor[rgb]{0.49,0.56,0.16}{##1}}}
\expandafter\def\csname PY@tok@nt\endcsname{\let\PY@bf=\textbf\def\PY@tc##1{\textcolor[rgb]{0.00,0.50,0.00}{##1}}}
\expandafter\def\csname PY@tok@nd\endcsname{\def\PY@tc##1{\textcolor[rgb]{0.67,0.13,1.00}{##1}}}
\expandafter\def\csname PY@tok@s\endcsname{\def\PY@tc##1{\textcolor[rgb]{0.73,0.13,0.13}{##1}}}
\expandafter\def\csname PY@tok@sd\endcsname{\let\PY@it=\textit\def\PY@tc##1{\textcolor[rgb]{0.73,0.13,0.13}{##1}}}
\expandafter\def\csname PY@tok@si\endcsname{\let\PY@bf=\textbf\def\PY@tc##1{\textcolor[rgb]{0.73,0.40,0.53}{##1}}}
\expandafter\def\csname PY@tok@se\endcsname{\let\PY@bf=\textbf\def\PY@tc##1{\textcolor[rgb]{0.73,0.40,0.13}{##1}}}
\expandafter\def\csname PY@tok@sr\endcsname{\def\PY@tc##1{\textcolor[rgb]{0.73,0.40,0.53}{##1}}}
\expandafter\def\csname PY@tok@ss\endcsname{\def\PY@tc##1{\textcolor[rgb]{0.10,0.09,0.49}{##1}}}
\expandafter\def\csname PY@tok@sx\endcsname{\def\PY@tc##1{\textcolor[rgb]{0.00,0.50,0.00}{##1}}}
\expandafter\def\csname PY@tok@m\endcsname{\def\PY@tc##1{\textcolor[rgb]{0.40,0.40,0.40}{##1}}}
\expandafter\def\csname PY@tok@gh\endcsname{\let\PY@bf=\textbf\def\PY@tc##1{\textcolor[rgb]{0.00,0.00,0.50}{##1}}}
\expandafter\def\csname PY@tok@gu\endcsname{\let\PY@bf=\textbf\def\PY@tc##1{\textcolor[rgb]{0.50,0.00,0.50}{##1}}}
\expandafter\def\csname PY@tok@gd\endcsname{\def\PY@tc##1{\textcolor[rgb]{0.63,0.00,0.00}{##1}}}
\expandafter\def\csname PY@tok@gi\endcsname{\def\PY@tc##1{\textcolor[rgb]{0.00,0.63,0.00}{##1}}}
\expandafter\def\csname PY@tok@gr\endcsname{\def\PY@tc##1{\textcolor[rgb]{1.00,0.00,0.00}{##1}}}
\expandafter\def\csname PY@tok@ge\endcsname{\let\PY@it=\textit}
\expandafter\def\csname PY@tok@gs\endcsname{\let\PY@bf=\textbf}
\expandafter\def\csname PY@tok@gp\endcsname{\let\PY@bf=\textbf\def\PY@tc##1{\textcolor[rgb]{0.00,0.00,0.50}{##1}}}
\expandafter\def\csname PY@tok@go\endcsname{\def\PY@tc##1{\textcolor[rgb]{0.53,0.53,0.53}{##1}}}
\expandafter\def\csname PY@tok@gt\endcsname{\def\PY@tc##1{\textcolor[rgb]{0.00,0.27,0.87}{##1}}}
\expandafter\def\csname PY@tok@err\endcsname{\def\PY@bc##1{\setlength{\fboxsep}{0pt}\fcolorbox[rgb]{1.00,0.00,0.00}{1,1,1}{\strut ##1}}}
\expandafter\def\csname PY@tok@kc\endcsname{\let\PY@bf=\textbf\def\PY@tc##1{\textcolor[rgb]{0.00,0.50,0.00}{##1}}}
\expandafter\def\csname PY@tok@kd\endcsname{\let\PY@bf=\textbf\def\PY@tc##1{\textcolor[rgb]{0.00,0.50,0.00}{##1}}}
\expandafter\def\csname PY@tok@kn\endcsname{\let\PY@bf=\textbf\def\PY@tc##1{\textcolor[rgb]{0.00,0.50,0.00}{##1}}}
\expandafter\def\csname PY@tok@kr\endcsname{\let\PY@bf=\textbf\def\PY@tc##1{\textcolor[rgb]{0.00,0.50,0.00}{##1}}}
\expandafter\def\csname PY@tok@bp\endcsname{\def\PY@tc##1{\textcolor[rgb]{0.00,0.50,0.00}{##1}}}
\expandafter\def\csname PY@tok@fm\endcsname{\def\PY@tc##1{\textcolor[rgb]{0.00,0.00,1.00}{##1}}}
\expandafter\def\csname PY@tok@vc\endcsname{\def\PY@tc##1{\textcolor[rgb]{0.10,0.09,0.49}{##1}}}
\expandafter\def\csname PY@tok@vg\endcsname{\def\PY@tc##1{\textcolor[rgb]{0.10,0.09,0.49}{##1}}}
\expandafter\def\csname PY@tok@vi\endcsname{\def\PY@tc##1{\textcolor[rgb]{0.10,0.09,0.49}{##1}}}
\expandafter\def\csname PY@tok@vm\endcsname{\def\PY@tc##1{\textcolor[rgb]{0.10,0.09,0.49}{##1}}}
\expandafter\def\csname PY@tok@sa\endcsname{\def\PY@tc##1{\textcolor[rgb]{0.73,0.13,0.13}{##1}}}
\expandafter\def\csname PY@tok@sb\endcsname{\def\PY@tc##1{\textcolor[rgb]{0.73,0.13,0.13}{##1}}}
\expandafter\def\csname PY@tok@sc\endcsname{\def\PY@tc##1{\textcolor[rgb]{0.73,0.13,0.13}{##1}}}
\expandafter\def\csname PY@tok@dl\endcsname{\def\PY@tc##1{\textcolor[rgb]{0.73,0.13,0.13}{##1}}}
\expandafter\def\csname PY@tok@s2\endcsname{\def\PY@tc##1{\textcolor[rgb]{0.73,0.13,0.13}{##1}}}
\expandafter\def\csname PY@tok@sh\endcsname{\def\PY@tc##1{\textcolor[rgb]{0.73,0.13,0.13}{##1}}}
\expandafter\def\csname PY@tok@s1\endcsname{\def\PY@tc##1{\textcolor[rgb]{0.73,0.13,0.13}{##1}}}
\expandafter\def\csname PY@tok@mb\endcsname{\def\PY@tc##1{\textcolor[rgb]{0.40,0.40,0.40}{##1}}}
\expandafter\def\csname PY@tok@mf\endcsname{\def\PY@tc##1{\textcolor[rgb]{0.40,0.40,0.40}{##1}}}
\expandafter\def\csname PY@tok@mh\endcsname{\def\PY@tc##1{\textcolor[rgb]{0.40,0.40,0.40}{##1}}}
\expandafter\def\csname PY@tok@mi\endcsname{\def\PY@tc##1{\textcolor[rgb]{0.40,0.40,0.40}{##1}}}
\expandafter\def\csname PY@tok@il\endcsname{\def\PY@tc##1{\textcolor[rgb]{0.40,0.40,0.40}{##1}}}
\expandafter\def\csname PY@tok@mo\endcsname{\def\PY@tc##1{\textcolor[rgb]{0.40,0.40,0.40}{##1}}}
\expandafter\def\csname PY@tok@ch\endcsname{\let\PY@it=\textit\def\PY@tc##1{\textcolor[rgb]{0.25,0.50,0.50}{##1}}}
\expandafter\def\csname PY@tok@cm\endcsname{\let\PY@it=\textit\def\PY@tc##1{\textcolor[rgb]{0.25,0.50,0.50}{##1}}}
\expandafter\def\csname PY@tok@cpf\endcsname{\let\PY@it=\textit\def\PY@tc##1{\textcolor[rgb]{0.25,0.50,0.50}{##1}}}
\expandafter\def\csname PY@tok@c1\endcsname{\let\PY@it=\textit\def\PY@tc##1{\textcolor[rgb]{0.25,0.50,0.50}{##1}}}
\expandafter\def\csname PY@tok@cs\endcsname{\let\PY@it=\textit\def\PY@tc##1{\textcolor[rgb]{0.25,0.50,0.50}{##1}}}

\def\PYZbs{\char`\\}
\def\PYZus{\char`\_}
\def\PYZob{\char`\{}
\def\PYZcb{\char`\}}
\def\PYZca{\char`\^}
\def\PYZam{\char`\&}
\def\PYZlt{\char`\<}
\def\PYZgt{\char`\>}
\def\PYZsh{\char`\#}
\def\PYZpc{\char`\%}
\def\PYZdl{\char`\$}
\def\PYZhy{\char`\-}
\def\PYZsq{\char`\'}
\def\PYZdq{\char`\"}
\def\PYZti{\char`\~}
% for compatibility with earlier versions
\def\PYZat{@}
\def\PYZlb{[}
\def\PYZrb{]}
\makeatother


    % Exact colors from NB
    \definecolor{incolor}{rgb}{0.0, 0.0, 0.5}
    \definecolor{outcolor}{rgb}{0.545, 0.0, 0.0}



    
    % Prevent overflowing lines due to hard-to-break entities
    \sloppy 
    % Setup hyperref package
    \hypersetup{
      breaklinks=true,  % so long urls are correctly broken across lines
      colorlinks=true,
      urlcolor=urlcolor,
      linkcolor=linkcolor,
      citecolor=citecolor,
      }
    % Slightly bigger margins than the latex defaults
    
    \geometry{verbose,tmargin=1in,bmargin=1in,lmargin=1in,rmargin=1in}
    
    

    \begin{document}
    
    
    \maketitle
    
    

    
    \begin{quote}
Universidade Federal da Bahia
\end{quote}

\begin{quote}
Instituto de Matemática e Estatística
\end{quote}

\begin{quote}
Departamento de Ciência da Computação
\end{quote}

\begin{quote}
MATA60 - Banco de Dados
\end{quote}

\begin{quote}
Docente: Vaninha Vieira
\end{quote}

\begin{quote}
Alunos: Angelmário Santana, Tassia Silva e Litiano Moura
\end{quote}

\begin{quote}
Data: 31 de Maio de 2018
\end{quote}

\begin{quote}
\begin{quote}
\subsection{Tema: Business
Intelligence}\label{tema-business-intelligence}
\end{quote}
\end{quote}

\begin{quote}
\begin{quote}
\begin{quote}
\mbox{}%
\paragraph{RELATÓRIO DA ANÁLISE ESTATÍSTICA DA BASE DE DADOS SOBRE
PEDIDOS DE ALIMENTO REALIZADOS POR APLICATIVO NO 2° SEMESTRE DE
2016.}\label{relatuxf3rio-da-anuxe1lise-estatuxedstica-da-base-de-dados-sobre-pedidos-de-alimento-realizados-por-aplicativo-no-2-semestre-de-2016.}
\end{quote}
\end{quote}
\end{quote}

    \paragraph{Introdução}\label{introduuxe7uxe3o}

Nos últimos anos um mercado que vem chamando a atenção de investidores é
o mercado de delivery, que está cada vez mais virando tendência entre os
brasileiros. Segundo a Associação Brasileira de Bares e Restaurantes
(ABRASEL) já em 2015, o mercado brasileiro movimentava 9 bilhões de
reais por ano, em 2017 o faturamento passou dos 10 bilhões.

Muitos estabelecimentos que tinha apenas espaços físicos, têm investido
no serviço de delivery para atrair clientes e aumentar o faturamento. O
SEBRAE reforça a preferência dos consumidores por lugares que ofereçam
entrega em domicílio, e afirma que 12\%, segundo pesquisa realizada, não
possuem nem loja física, trabalhando apenas com entregas inclusive por
aplicativos.

Muitos consumidores entrevistados afirmaram que pelo comodismo de não
ter que enfrentar o trânsito e pelas taxas de como estacionamento e
outras, faz mais sentido ir no aplicativo e escolher no cardápio
diversificado, a facilidade de ver vários tipos de comida com apenas
alguns cliques. Mesmo havendo taxa de entrega, ainda sim vale a pena
conferir as ofertas oferecidas pelas lojas nos aplicativos.

Neste trabalho, foi nos concedido uma base de dados que mostra a
respeito das entregas de alimentos em algumas cidades brasileiras com
intuito de inferir informações a respeito e aplicar Business
Intelligence.

    \paragraph{1. Analisando os dados por meio
estatístico}\label{analisando-os-dados-por-meio-estatuxedstico}

Após pesquisar sobre o assunto, foi realizada uma análise na base da
equipe. Foi necessário analisar os atributos da base para entender o
comportamento dos dados. A equipe contou com ajuda do
\href{http://jupyter.org/}{Jupyter Notebook} que através de programação
em Python nos ajudou a manipular os dados e desenvolver os gráficos que
serão apresentados a seguir para construção do trabalho.

    \begin{Verbatim}[commandchars=\\\{\}]
{\color{incolor}In [{\color{incolor}5}]:} \PY{k+kn}{import} \PY{n+nn}{numpy} \PY{k}{as} \PY{n+nn}{np}
        \PY{k+kn}{import} \PY{n+nn}{pandas} \PY{k}{as} \PY{n+nn}{pd}
        \PY{k+kn}{import} \PY{n+nn}{statsmodels} \PY{k}{as} \PY{n+nn}{st}
        \PY{k+kn}{import} \PY{n+nn}{matplotlib}\PY{n+nn}{.}\PY{n+nn}{pyplot} \PY{k}{as} \PY{n+nn}{plt}
        \PY{k+kn}{import} \PY{n+nn}{seaborn} \PY{k}{as} \PY{n+nn}{sns}
        \PY{k+kn}{from} \PY{n+nn}{pandas}\PY{n+nn}{.}\PY{n+nn}{plotting} \PY{k}{import} \PY{n}{table}
        \PY{k+kn}{from} \PY{n+nn}{ipywidgets} \PY{k}{import} \PY{n}{widgets}
        \PY{k+kn}{import} \PY{n+nn}{datetime}
\end{Verbatim}


    \begin{Verbatim}[commandchars=\\\{\}]
{\color{incolor}In [{\color{incolor}6}]:} \PY{c+c1}{\PYZsh{}Carregando a base de dados}
        \PY{n}{base} \PY{o}{=} \PY{n}{pd}\PY{o}{.}\PY{n}{read\PYZus{}csv}\PY{p}{(}\PY{l+s+s1}{\PYZsq{}}\PY{l+s+s1}{aplicativo.csv}\PY{l+s+s1}{\PYZsq{}}\PY{p}{)}
\end{Verbatim}


    \textbf{1.1} O primeiro passo foi observar a base de dados para obter
uma visão geral dos atributos que existem na base.

    \begin{Verbatim}[commandchars=\\\{\}]
{\color{incolor}In [{\color{incolor}7}]:} \PY{c+c1}{\PYZsh{}5 primeiros registros \PYZhy{} Visão inicial dos dados}
        \PY{n}{base}\PY{o}{.}\PY{n}{head}\PY{p}{(}\PY{l+m+mi}{5}\PY{p}{)}
\end{Verbatim}


\begin{Verbatim}[commandchars=\\\{\}]
{\color{outcolor}Out[{\color{outcolor}7}]:}   DATA\_PEDIDO HORA\_PEDIDO DIA\_PEDIDO  VALOR\_PRODUTOS  TAXA\_ENTREGA  \textbackslash{}
        0  2016-07-05       19:51     Sunday            16.0           4.0   
        1  2016-07-05       20:58     Sunday            28.0           4.0   
        2  2016-07-05       21:35     Sunday            13.0           4.0   
        3  2016-07-06       23:22     Monday            11.5           4.0   
        4  2016-07-07       20:08    Tuesday            19.0           4.0   
        
           TOTAL\_PEDIDO FORMA\_PAGAMENTO  AVALIACAO    STATUS  ID\_ESTABELECIMENTO  \textbackslash{}
        0          20.0        Dinheiro        NaN  Entregue                  16   
        1          32.0        Dinheiro        NaN  Recusado                  16   
        2          17.0        Dinheiro        NaN  Recusado                  16   
        3          15.5        Dinheiro        NaN  Entregue                  16   
        4          23.0        Dinheiro        NaN  Recusado                  16   
        
          TIPO\_ESTABELECIMENTO  ID\_USUARIO DDD\_USUARIO DATA\_CADASTRO\_USUARIO  \textbackslash{}
        0           Lanchonete       50720          77            2016-07-05   
        1           Lanchonete       48784          77            2016-06-18   
        2           Lanchonete        7016          77            2015-08-10   
        3           Lanchonete       48536          77            2016-06-15   
        4           Lanchonete       21160          77            2015-12-21   
        
          PRIMEIRO\_PEDIDO BAIRRO\_USUARIO        CIDADE\_USUARIO SO\_DISPOSITIVO  
        0             Sim        Ipanema  Vitória da Conquista        Android  
        1             Não       Candeias  Vitória da Conquista            iOS  
        2             Não        Urbis I  Vitória da Conquista        Android  
        3             Não     Alto Maron  Vitória da Conquista        Android  
        4             Não       Candeias  Vitória da Conquista        Android  
\end{Verbatim}
            
    \begin{Verbatim}[commandchars=\\\{\}]
{\color{incolor}In [{\color{incolor}8}]:} \PY{c+c1}{\PYZsh{}Gráfico que exibe a quantidade valores ausentes }
        \PY{c+c1}{\PYZsh{}para cada atributo da base}
        
        \PY{n}{fontsize} \PY{o}{=} \PY{l+m+mi}{15}
        \PY{n}{font\PYZus{}size} \PY{o}{=} \PY{l+m+mi}{15}
        \PY{n}{df} \PY{o}{=} \PY{n}{pd}\PY{o}{.}\PY{n}{DataFrame}\PY{p}{(}\PY{n}{base}\PY{o}{.}\PY{n}{apply}\PY{p}{(}\PY{k}{lambda} \PY{n}{x}\PY{p}{:} 
                                     \PY{n+nb}{sum}\PY{p}{(}\PY{n}{x}\PY{o}{.}\PY{n}{isnull}\PY{p}{(}\PY{p}{)}\PY{p}{)}\PY{p}{,}\PY{n}{axis}\PY{o}{=}\PY{l+m+mi}{0}\PY{p}{)}\PY{p}{)}
        \PY{n}{df}\PY{o}{.}\PY{n}{columns} \PY{o}{=} \PY{p}{[}\PY{l+s+s1}{\PYZsq{}}\PY{l+s+s1}{Quantidade}\PY{l+s+s1}{\PYZsq{}}\PY{p}{]}
        \PY{n}{ax8} \PY{o}{=}  \PY{n}{plt}\PY{o}{.}\PY{n}{subplot2grid}\PY{p}{(}\PY{p}{(}\PY{l+m+mi}{2}\PY{p}{,} \PY{l+m+mi}{2}\PY{p}{)}\PY{p}{,} \PY{p}{(}\PY{l+m+mi}{0}\PY{p}{,} \PY{l+m+mi}{0}\PY{p}{)}\PY{p}{,} \PY{n}{colspan}\PY{o}{=}\PY{l+m+mi}{2}\PY{p}{)}
        \PY{n}{grafico} \PY{o}{=} \PY{n}{df}\PY{o}{.}\PY{n}{plot}\PY{o}{.}\PY{n}{barh}\PY{p}{(}\PY{n}{figsize}\PY{o}{=}\PY{p}{(}\PY{l+m+mi}{10}\PY{p}{,}\PY{l+m+mi}{16}\PY{p}{)}\PY{p}{,} \PY{n}{fontsize}\PY{o}{=} \PY{n}{fontsize}\PY{p}{,}   
                               \PY{n}{ax}\PY{o}{=}\PY{n}{ax8}\PY{p}{,} 
                               \PY{n}{facecolor}\PY{o}{=}\PY{l+s+s1}{\PYZsq{}}\PY{l+s+s1}{blue}\PY{l+s+s1}{\PYZsq{}}\PY{p}{,} 
                               \PY{n}{edgecolor}\PY{o}{=}\PY{l+s+s1}{\PYZsq{}}\PY{l+s+s1}{white}\PY{l+s+s1}{\PYZsq{}}\PY{p}{,} 
                               \PY{n}{legend} \PY{o}{=} \PY{k+kc}{False}\PY{p}{)}
        \PY{n}{plt}\PY{o}{.}\PY{n}{title}\PY{p}{(}\PY{l+s+s1}{\PYZsq{}}\PY{l+s+s1}{Valores ausentes}\PY{l+s+s1}{\PYZsq{}}\PY{p}{)}
        
        \PY{n}{df} \PY{o}{=} \PY{n}{df}\PY{p}{[}\PY{p}{(}\PY{n}{df}\PY{p}{[}\PY{l+s+s1}{\PYZsq{}}\PY{l+s+s1}{Quantidade}\PY{l+s+s1}{\PYZsq{}}\PY{p}{]} \PY{o}{\PYZgt{}} \PY{l+m+mi}{0}\PY{p}{)}\PY{p}{]}
        \PY{n}{df}
        \PY{n}{mpl\PYZus{}table} \PY{o}{=} \PY{n}{table}\PY{p}{(}\PY{n}{ax8}\PY{p}{,} \PY{n}{df}\PY{p}{,} \PY{n}{loc}\PY{o}{=}\PY{l+s+s1}{\PYZsq{}}\PY{l+s+s1}{best}\PY{l+s+s1}{\PYZsq{}}\PY{p}{,} \PY{n}{rowLoc}\PY{o}{=}\PY{l+s+s1}{\PYZsq{}}\PY{l+s+s1}{left}\PY{l+s+s1}{\PYZsq{}}\PY{p}{,} \PY{n}{colLoc} \PY{o}{=} \PY{l+s+s1}{\PYZsq{}}\PY{l+s+s1}{center}\PY{l+s+s1}{\PYZsq{}}\PY{p}{)}
        \PY{n}{mpl\PYZus{}table}\PY{o}{.}\PY{n}{auto\PYZus{}set\PYZus{}font\PYZus{}size}\PY{p}{(}\PY{k+kc}{False}\PY{p}{)}
        \PY{n}{mpl\PYZus{}table}\PY{o}{.}\PY{n}{set\PYZus{}fontsize}\PY{p}{(}\PY{n}{font\PYZus{}size}\PY{p}{)}
        \PY{n}{mpl\PYZus{}table}\PY{o}{.}\PY{n}{scale}\PY{p}{(}\PY{l+m+mf}{0.2}\PY{p}{,}\PY{l+m+mf}{2.8}\PY{p}{)}
\end{Verbatim}


    \begin{center}
    \adjustimage{max size={0.9\linewidth}{0.9\paperheight}}{output_7_0.png}
    \end{center}
    { \hspace*{\fill} \\}
    
    \textbf{1.5} No gráficos de histograma, é possível através de atributos
numéricos observar a distribuição na base de dados

    \begin{Verbatim}[commandchars=\\\{\}]
{\color{incolor}In [{\color{incolor}33}]:} \PY{c+c1}{\PYZsh{}Histograma dos atributos numéricos para }
         \PY{c+c1}{\PYZsh{}observar a distribuição na base de dados.}
         
         \PY{n}{fontsize} \PY{o}{=} \PY{l+m+mi}{15}
         \PY{n}{font\PYZus{}size} \PY{o}{=} \PY{l+m+mi}{15}
         
         \PY{n}{fig} \PY{o}{=} \PY{n}{plt}\PY{o}{.}\PY{n}{figure}\PY{p}{(}\PY{n}{figsize}\PY{o}{=}\PY{p}{(}\PY{l+m+mi}{12}\PY{p}{,}\PY{l+m+mi}{10}\PY{p}{)}\PY{p}{)}
         \PY{n}{ax5} \PY{o}{=} \PY{n}{fig}\PY{o}{.}\PY{n}{add\PYZus{}subplot}\PY{p}{(}\PY{l+m+mi}{221}\PY{p}{)}
         \PY{n}{base}\PY{o}{.}\PY{n}{hist}\PY{p}{(}\PY{n}{column}\PY{o}{=}\PY{l+s+s1}{\PYZsq{}}\PY{l+s+s1}{VALOR\PYZus{}PRODUTOS}\PY{l+s+s1}{\PYZsq{}}\PY{p}{,} \PY{n}{bins}\PY{o}{=}\PY{l+m+mi}{100}\PY{p}{,} \PY{n}{figsize}\PY{o}{=}\PY{p}{(}\PY{l+m+mi}{20}\PY{p}{,}\PY{l+m+mi}{5}\PY{p}{)}\PY{p}{,} \PY{n}{ax}\PY{o}{=}\PY{n}{ax5}\PY{p}{)}
         \PY{n}{plt}\PY{o}{.}\PY{n}{title}\PY{p}{(}\PY{l+s+s1}{\PYZsq{}}\PY{l+s+s1}{Valor dos produtos}\PY{l+s+s1}{\PYZsq{}}\PY{p}{)}
         
         
         \PY{n}{ax6} \PY{o}{=} \PY{n}{fig}\PY{o}{.}\PY{n}{add\PYZus{}subplot}\PY{p}{(}\PY{l+m+mi}{222}\PY{p}{)}
         \PY{n}{base}\PY{o}{.}\PY{n}{hist}\PY{p}{(}\PY{n}{column}\PY{o}{=}\PY{l+s+s1}{\PYZsq{}}\PY{l+s+s1}{TAXA\PYZus{}ENTREGA}\PY{l+s+s1}{\PYZsq{}}\PY{p}{,} \PY{n}{bins}\PY{o}{=}\PY{l+m+mi}{100}\PY{p}{,} \PY{n}{figsize}\PY{o}{=}\PY{p}{(}\PY{l+m+mi}{20}\PY{p}{,}\PY{l+m+mi}{5}\PY{p}{)}\PY{p}{,}  \PY{n}{ax}\PY{o}{=}\PY{n}{ax6}\PY{p}{)}
         \PY{n}{plt}\PY{o}{.}\PY{n}{title}\PY{p}{(}\PY{l+s+s1}{\PYZsq{}}\PY{l+s+s1}{Taxa de entrega}\PY{l+s+s1}{\PYZsq{}}\PY{p}{)}
         
         
         \PY{n}{ax7} \PY{o}{=} \PY{n}{fig}\PY{o}{.}\PY{n}{add\PYZus{}subplot}\PY{p}{(}\PY{l+m+mi}{223}\PY{p}{)}
         \PY{n}{base}\PY{o}{.}\PY{n}{hist}\PY{p}{(}\PY{n}{column}\PY{o}{=}\PY{l+s+s1}{\PYZsq{}}\PY{l+s+s1}{TOTAL\PYZus{}PEDIDO}\PY{l+s+s1}{\PYZsq{}}\PY{p}{,} \PY{n}{bins}\PY{o}{=}\PY{l+m+mi}{100}\PY{p}{,} \PY{n}{figsize}\PY{o}{=}\PY{p}{(}\PY{l+m+mi}{20}\PY{p}{,}\PY{l+m+mi}{5}\PY{p}{)}\PY{p}{,}   \PY{n}{ax}\PY{o}{=}\PY{n}{ax7}\PY{p}{)}
         \PY{n}{plt}\PY{o}{.}\PY{n}{title}\PY{p}{(}\PY{l+s+s1}{\PYZsq{}}\PY{l+s+s1}{Total de pedidos}\PY{l+s+s1}{\PYZsq{}}\PY{p}{)}
         
         \PY{n}{dff} \PY{o}{=} \PY{n}{pd}\PY{o}{.}\PY{n}{DataFrame}\PY{p}{(}\PY{n}{base}\PY{o}{.}\PY{n}{groupby}\PY{p}{(}\PY{l+s+s1}{\PYZsq{}}\PY{l+s+s1}{ID\PYZus{}USUARIO}\PY{l+s+s1}{\PYZsq{}}\PY{p}{)}\PY{o}{.}\PY{n}{size}\PY{p}{(}\PY{p}{)}\PY{p}{)}
         \PY{n}{dff}\PY{o}{.}\PY{n}{columns} \PY{o}{=} \PY{p}{[}\PY{l+s+s1}{\PYZsq{}}\PY{l+s+s1}{Quantidade}\PY{l+s+s1}{\PYZsq{}}\PY{p}{]}
         \PY{n}{dff}\PY{o}{.}\PY{n}{sort\PYZus{}values}\PY{p}{(}\PY{n}{by} \PY{o}{=}\PY{p}{[}\PY{l+s+s1}{\PYZsq{}}\PY{l+s+s1}{Quantidade}\PY{l+s+s1}{\PYZsq{}}\PY{p}{]}\PY{p}{,} \PY{n}{ascending}\PY{o}{=}\PY{k+kc}{False}\PY{p}{)}
         \PY{n}{ax8} \PY{o}{=} \PY{n}{fig}\PY{o}{.}\PY{n}{add\PYZus{}subplot}\PY{p}{(}\PY{l+m+mi}{224}\PY{p}{)}
         
         \PY{n}{dff}\PY{o}{.}\PY{n}{hist}\PY{p}{(}\PY{n}{column}\PY{o}{=}\PY{l+s+s1}{\PYZsq{}}\PY{l+s+s1}{Quantidade}\PY{l+s+s1}{\PYZsq{}}\PY{p}{,} \PY{n}{bins}\PY{o}{=}\PY{l+m+mi}{100}\PY{p}{,} \PY{n}{figsize}\PY{o}{=}\PY{p}{(}\PY{l+m+mi}{20}\PY{p}{,}\PY{l+m+mi}{5}\PY{p}{)}\PY{p}{,}   \PY{n}{ax}\PY{o}{=}\PY{n}{ax8}\PY{p}{)}
         \PY{n}{plt}\PY{o}{.}\PY{n}{title}\PY{p}{(}\PY{l+s+s1}{\PYZsq{}}\PY{l+s+s1}{Quantidade de pedidos por usuário}\PY{l+s+s1}{\PYZsq{}}\PY{p}{)}
\end{Verbatim}


\begin{Verbatim}[commandchars=\\\{\}]
{\color{outcolor}Out[{\color{outcolor}33}]:} Text(0.5,1,'Quantidade de pedidos por usuário')
\end{Verbatim}
            
    \begin{center}
    \adjustimage{max size={0.9\linewidth}{0.9\paperheight}}{output_9_1.png}
    \end{center}
    { \hspace*{\fill} \\}
    
    \begin{Verbatim}[commandchars=\\\{\}]
{\color{incolor}In [{\color{incolor} }]:} \PY{c+c1}{\PYZsh{}base.describe()}
        \PY{c+c1}{\PYZsh{}min(base[\PYZsq{}DATA\PYZus{}PEDIDO\PYZsq{}])}
        \PY{c+c1}{\PYZsh{}max(base[\PYZsq{}DATA\PYZus{}PEDIDO\PYZsq{}])}
\end{Verbatim}


    \begin{Verbatim}[commandchars=\\\{\}]
{\color{incolor}In [{\color{incolor}27}]:} \PY{c+c1}{\PYZsh{} Formas de pagamentos e a respectiva quantidade,}
         \PY{c+c1}{\PYZsh{} dado que o status do pedido é Recusado.}
         
         \PY{c+c1}{\PYZsh{}Pergunta para ser respondida: Quais as formas de pagamento e a quantidade, }
         \PY{c+c1}{\PYZsh{} que os pedidos foram recusados?}
         
         \PY{n}{filt} \PY{o}{=} \PY{n}{base}\PY{p}{[}\PY{p}{(}\PY{n}{base}\PY{p}{[}\PY{l+s+s1}{\PYZsq{}}\PY{l+s+s1}{STATUS}\PY{l+s+s1}{\PYZsq{}}\PY{p}{]} \PY{o}{==} \PY{l+s+s2}{\PYZdq{}}\PY{l+s+s2}{Recusado}\PY{l+s+s2}{\PYZdq{}}\PY{p}{)}\PY{p}{]}
         \PY{n}{dff} \PY{o}{=} \PY{n}{pd}\PY{o}{.}\PY{n}{DataFrame}\PY{p}{(}\PY{n}{filt}\PY{o}{.}\PY{n}{groupby}\PY{p}{(}\PY{l+s+s1}{\PYZsq{}}\PY{l+s+s1}{FORMA\PYZus{}PAGAMENTO}\PY{l+s+s1}{\PYZsq{}}\PY{p}{)}\PY{o}{.}\PY{n}{size}\PY{p}{(}\PY{p}{)}\PY{p}{)}
         \PY{n}{dff}\PY{o}{.}\PY{n}{columns} \PY{o}{=} \PY{p}{[}\PY{l+s+s1}{\PYZsq{}}\PY{l+s+s1}{Quantidade}\PY{l+s+s1}{\PYZsq{}}\PY{p}{]}
         \PY{n}{dff}\PY{o}{.}\PY{n}{sort\PYZus{}values}\PY{p}{(}\PY{n}{by} \PY{o}{=}\PY{p}{[}\PY{l+s+s1}{\PYZsq{}}\PY{l+s+s1}{Quantidade}\PY{l+s+s1}{\PYZsq{}}\PY{p}{]}\PY{p}{,} \PY{n}{ascending}\PY{o}{=}\PY{k+kc}{False}\PY{p}{)}
\end{Verbatim}


\begin{Verbatim}[commandchars=\\\{\}]
{\color{outcolor}Out[{\color{outcolor}27}]:}             Quantidade
         ID\_USUARIO            
         40872              100
         27784               79
         10392               79
         41408               78
         41304               73
         6856                71
         264                 68
         37264               66
         14480               65
         44328               64
         29872               58
         38200               56
         46048               55
         6256                55
         29408               52
         32368               52
         50648               51
         53288               50
         11232               50
         15488               49
         47128               48
         29888               48
         20976               47
         49856               47
         21536               47
         24912               46
         19304               45
         1568                45
         11008               45
         53352               45
         {\ldots}                {\ldots}
         58472                1
         58464                1
         58456                1
         15208                1
         15280                1
         49456                1
         58656                1
         14432                1
         58760                1
         58880                1
         14488                1
         14504                1
         49232                1
         49240                1
         32480                1
         58792                1
         32416                1
         14624                1
         58752                1
         32384                1
         49344                1
         58736                1
         58728                1
         14704                1
         58712                1
         49360                1
         58688                1
         58680                1
         58672                1
         70176                1
         
         [4530 rows x 1 columns]
\end{Verbatim}
            
    \begin{Verbatim}[commandchars=\\\{\}]
{\color{incolor}In [{\color{incolor}13}]:} \PY{c+c1}{\PYZsh{} Formas de pagamentos e a respectiva quantidade,}
         \PY{c+c1}{\PYZsh{} dado que o tipo de estabelecimento do pedido é Lanchonete.}
         
         \PY{c+c1}{\PYZsh{}Pergunta para ser respondida: Quantos estabelecimentos do}
         \PY{c+c1}{\PYZsh{} tipo lanchonete tem na base de dados?}
         
         
         \PY{n}{filt} \PY{o}{=} \PY{n}{base}\PY{p}{[}\PY{p}{(}\PY{n}{base}\PY{p}{[}\PY{l+s+s1}{\PYZsq{}}\PY{l+s+s1}{TIPO\PYZus{}ESTABELECIMENTO}\PY{l+s+s1}{\PYZsq{}}\PY{p}{]} \PY{o}{==} \PY{l+s+s2}{\PYZdq{}}\PY{l+s+s2}{Lanchonete}\PY{l+s+s2}{\PYZdq{}}\PY{p}{)}\PY{p}{]}
         \PY{n}{dff} \PY{o}{=} \PY{n}{pd}\PY{o}{.}\PY{n}{DataFrame}\PY{p}{(}\PY{n}{filt}\PY{o}{.}\PY{n}{groupby}\PY{p}{(}\PY{l+s+s1}{\PYZsq{}}\PY{l+s+s1}{FORMA\PYZus{}PAGAMENTO}\PY{l+s+s1}{\PYZsq{}}\PY{p}{)}\PY{o}{.}\PY{n}{size}\PY{p}{(}\PY{p}{)}\PY{p}{)}
         \PY{n}{dff}\PY{o}{.}\PY{n}{columns} \PY{o}{=} \PY{p}{[}\PY{l+s+s1}{\PYZsq{}}\PY{l+s+s1}{Quantidade}\PY{l+s+s1}{\PYZsq{}}\PY{p}{]}
         \PY{n}{dff}\PY{o}{.}\PY{n}{sort\PYZus{}values}\PY{p}{(}\PY{n}{by} \PY{o}{=}\PY{p}{[}\PY{l+s+s1}{\PYZsq{}}\PY{l+s+s1}{Quantidade}\PY{l+s+s1}{\PYZsq{}}\PY{p}{]}\PY{p}{,} \PY{n}{ascending}\PY{o}{=}\PY{k+kc}{False}\PY{p}{)}
\end{Verbatim}


\begin{Verbatim}[commandchars=\\\{\}]
{\color{outcolor}Out[{\color{outcolor}13}]:}                  Quantidade
         FORMA\_PAGAMENTO            
         Dinheiro               5066
         Cartão                 2121
\end{Verbatim}
            
    \begin{Verbatim}[commandchars=\\\{\}]
{\color{incolor}In [{\color{incolor}14}]:} \PY{c+c1}{\PYZsh{} Obter o registro em que a cidade seja Vitória da Conquista, }
         \PY{c+c1}{\PYZsh{} a comida é japonesa e o valor é 274.20}
         
         \PY{n}{filt} \PY{o}{=} \PY{n}{base}\PY{p}{[}\PY{p}{(}\PY{n}{base}\PY{p}{[}\PY{l+s+s1}{\PYZsq{}}\PY{l+s+s1}{CIDADE\PYZus{}USUARIO}\PY{l+s+s1}{\PYZsq{}}\PY{p}{]} \PY{o}{==} \PY{l+s+s2}{\PYZdq{}}\PY{l+s+s2}{Vitória da Conquista}\PY{l+s+s2}{\PYZdq{}}\PY{p}{)} 
                     \PY{o}{\PYZam{}} \PY{p}{(}\PY{n}{base}\PY{p}{[}\PY{l+s+s1}{\PYZsq{}}\PY{l+s+s1}{TIPO\PYZus{}ESTABELECIMENTO}\PY{l+s+s1}{\PYZsq{}}\PY{p}{]} \PY{o}{==} \PY{l+s+s2}{\PYZdq{}}\PY{l+s+s2}{Comida Japonesa}\PY{l+s+s2}{\PYZdq{}}\PY{p}{)}
                    \PY{o}{\PYZam{}} \PY{p}{(}\PY{n}{base}\PY{p}{[}\PY{l+s+s1}{\PYZsq{}}\PY{l+s+s1}{VALOR\PYZus{}PRODUTOS}\PY{l+s+s1}{\PYZsq{}}\PY{p}{]} \PY{o}{==} \PY{l+m+mf}{274.20}\PY{p}{)}
                    \PY{p}{]}
         \PY{n}{dff} \PY{o}{=} \PY{n}{pd}\PY{o}{.}\PY{n}{DataFrame}\PY{p}{(}\PY{n}{filt}\PY{o}{.}\PY{n}{groupby}\PY{p}{(}\PY{l+s+s1}{\PYZsq{}}\PY{l+s+s1}{VALOR\PYZus{}PRODUTOS}\PY{l+s+s1}{\PYZsq{}}\PY{p}{)}\PY{o}{.}\PY{n}{size}\PY{p}{(}\PY{p}{)}\PY{p}{)}
         \PY{n}{dff}\PY{o}{.}\PY{n}{columns} \PY{o}{=} \PY{p}{[}\PY{l+s+s1}{\PYZsq{}}\PY{l+s+s1}{Quantidade}\PY{l+s+s1}{\PYZsq{}}\PY{p}{]}
         \PY{n}{dff}\PY{o}{.}\PY{n}{sort\PYZus{}values}\PY{p}{(}\PY{n}{by} \PY{o}{=}\PY{p}{[}\PY{l+s+s1}{\PYZsq{}}\PY{l+s+s1}{Quantidade}\PY{l+s+s1}{\PYZsq{}}\PY{p}{]}\PY{p}{,} \PY{n}{ascending}\PY{o}{=}\PY{k+kc}{False}\PY{p}{)}
         \PY{n}{filt}
\end{Verbatim}


\begin{Verbatim}[commandchars=\\\{\}]
{\color{outcolor}Out[{\color{outcolor}14}]:}       DATA\_PEDIDO HORA\_PEDIDO DIA\_PEDIDO  VALOR\_PRODUTOS  TAXA\_ENTREGA  \textbackslash{}
         21612  2016-08-07       19:44     Friday           274.2           3.0   
         
                TOTAL\_PEDIDO FORMA\_PAGAMENTO  AVALIACAO    STATUS  ID\_ESTABELECIMENTO  \textbackslash{}
         21612         277.2          Cartão        NaN  Entregue                 280   
         
               TIPO\_ESTABELECIMENTO  ID\_USUARIO DDD\_USUARIO DATA\_CADASTRO\_USUARIO  \textbackslash{}
         21612      Comida Japonesa       55080          77            2016-08-07   
         
               PRIMEIRO\_PEDIDO BAIRRO\_USUARIO        CIDADE\_USUARIO SO\_DISPOSITIVO  
         21612             Sim      Boa Vista  Vitória da Conquista        Android  
\end{Verbatim}
            
    \begin{Verbatim}[commandchars=\\\{\}]
{\color{incolor}In [{\color{incolor}15}]:} \PY{c+c1}{\PYZsh{} Obter os valores dos produtos e a respectiva quantidade, da cidade de Vitória da}
         \PY{c+c1}{\PYZsh{} Conquista e que o tipo de estabelecimento seja Acarajé}
         
         \PY{c+c1}{\PYZsh{}Pergunta para ser respondida: Quais valores e quantos produtos }
         \PY{c+c1}{\PYZsh{} do respectivo valor, do tipo de estabelecimento lanchonete }
         \PY{c+c1}{\PYZsh{} na cidade de Vitória da conquista}
         
         \PY{n}{filt} \PY{o}{=} \PY{n}{base}\PY{p}{[}\PY{p}{(}\PY{n}{base}\PY{p}{[}\PY{l+s+s1}{\PYZsq{}}\PY{l+s+s1}{CIDADE\PYZus{}USUARIO}\PY{l+s+s1}{\PYZsq{}}\PY{p}{]} \PY{o}{==} \PY{l+s+s2}{\PYZdq{}}\PY{l+s+s2}{Vitória da Conquista}\PY{l+s+s2}{\PYZdq{}}\PY{p}{)} 
                     \PY{o}{\PYZam{}} \PY{p}{(}\PY{n}{base}\PY{p}{[}\PY{l+s+s1}{\PYZsq{}}\PY{l+s+s1}{TIPO\PYZus{}ESTABELECIMENTO}\PY{l+s+s1}{\PYZsq{}}\PY{p}{]} \PY{o}{==} \PY{l+s+s2}{\PYZdq{}}\PY{l+s+s2}{Acarajé}\PY{l+s+s2}{\PYZdq{}}\PY{p}{)}\PY{p}{]}
         \PY{n}{dff} \PY{o}{=} \PY{n}{pd}\PY{o}{.}\PY{n}{DataFrame}\PY{p}{(}\PY{n}{filt}\PY{o}{.}\PY{n}{groupby}\PY{p}{(}\PY{l+s+s1}{\PYZsq{}}\PY{l+s+s1}{VALOR\PYZus{}PRODUTOS}\PY{l+s+s1}{\PYZsq{}}\PY{p}{)}\PY{o}{.}\PY{n}{size}\PY{p}{(}\PY{p}{)}\PY{p}{)}
         \PY{n}{dff}\PY{o}{.}\PY{n}{columns} \PY{o}{=} \PY{p}{[}\PY{l+s+s1}{\PYZsq{}}\PY{l+s+s1}{Quantidade}\PY{l+s+s1}{\PYZsq{}}\PY{p}{]}
         \PY{n}{dff}\PY{o}{.}\PY{n}{sort\PYZus{}values}\PY{p}{(}\PY{n}{by} \PY{o}{=}\PY{p}{[}\PY{l+s+s1}{\PYZsq{}}\PY{l+s+s1}{Quantidade}\PY{l+s+s1}{\PYZsq{}}\PY{p}{]}\PY{p}{,} \PY{n}{ascending}\PY{o}{=}\PY{k+kc}{False}\PY{p}{)}
\end{Verbatim}


\begin{Verbatim}[commandchars=\\\{\}]
{\color{outcolor}Out[{\color{outcolor}15}]:}                 Quantidade
         VALOR\_PRODUTOS            
         37.0                    11
         35.0                     1
\end{Verbatim}
            
    \begin{Verbatim}[commandchars=\\\{\}]
{\color{incolor}In [{\color{incolor}16}]:} \PY{c+c1}{\PYZsh{}Obter os tipos de estabelecimento e as suas respectivas quantidades}
         \PY{c+c1}{\PYZsh{} da cidade de Vitória da Conquista.}
         
         \PY{c+c1}{\PYZsh{}Pergunta para ser respondida: Na cidade de vitória da conquista}
         \PY{c+c1}{\PYZsh{} quais e quantos são os tipos de estabelecimentos?}
         
         \PY{n}{filt} \PY{o}{=} \PY{n}{base}\PY{p}{[}\PY{p}{(}\PY{n}{base}\PY{p}{[}\PY{l+s+s1}{\PYZsq{}}\PY{l+s+s1}{CIDADE\PYZus{}USUARIO}\PY{l+s+s1}{\PYZsq{}}\PY{p}{]} \PY{o}{==} \PY{l+s+s2}{\PYZdq{}}\PY{l+s+s2}{Vitória da Conquista}\PY{l+s+s2}{\PYZdq{}}\PY{p}{)}\PY{p}{]}
         \PY{n}{dff} \PY{o}{=} \PY{n}{pd}\PY{o}{.}\PY{n}{DataFrame}\PY{p}{(}\PY{n}{filt}\PY{o}{.}\PY{n}{groupby}\PY{p}{(}\PY{l+s+s1}{\PYZsq{}}\PY{l+s+s1}{TIPO\PYZus{}ESTABELECIMENTO}\PY{l+s+s1}{\PYZsq{}}\PY{p}{)}\PY{o}{.}\PY{n}{size}\PY{p}{(}\PY{p}{)}\PY{p}{)}
         \PY{n}{dff}\PY{o}{.}\PY{n}{columns} \PY{o}{=} \PY{p}{[}\PY{l+s+s1}{\PYZsq{}}\PY{l+s+s1}{Quantidade}\PY{l+s+s1}{\PYZsq{}}\PY{p}{]}
         \PY{n}{dff}\PY{o}{.}\PY{n}{sort\PYZus{}values}\PY{p}{(}\PY{n}{by} \PY{o}{=}\PY{p}{[}\PY{l+s+s1}{\PYZsq{}}\PY{l+s+s1}{Quantidade}\PY{l+s+s1}{\PYZsq{}}\PY{p}{]}\PY{p}{,} \PY{n}{ascending}\PY{o}{=}\PY{k+kc}{False}\PY{p}{)}
\end{Verbatim}


\begin{Verbatim}[commandchars=\\\{\}]
{\color{outcolor}Out[{\color{outcolor}16}]:}                                Quantidade
         TIPO\_ESTABELECIMENTO                     
         Lanchonete                           7065
         Pizzaria                             3733
         Marmitex                             3491
         Comida Natural                       2128
         Restaurante                          1975
         Comida Japonesa                      1152
         Pizzaria/Esfiharia                   1075
         Restaurante/Tapiocaria                573
         Hot-Dog                               450
         Sopas                                 413
         Pizzaria/Lanchonete                   359
         Culinária Oriental                    332
         Doceria                               274
         Picoleteria                           252
         Espetinhos                            213
         Tapiocaria                            115
         Pães e Bolos                          109
         Pizzaria/Esfiharia/Pastelaria         104
         Saladas                                34
         Comida Árabe                           30
         Acarajé                                12
\end{Verbatim}
            
    \begin{Verbatim}[commandchars=\\\{\}]
{\color{incolor}In [{\color{incolor}17}]:} \PY{c+c1}{\PYZsh{}Obter as cidades do usuário e suas respectivas quantidades}
         \PY{c+c1}{\PYZsh{} na base de dados}
         
         \PY{c+c1}{\PYZsh{}Pergunta para ser respondida: Quais e quantas cidades tem na }
         \PY{c+c1}{\PYZsh{} base de dados?}
         
         \PY{n}{df} \PY{o}{=} \PY{n}{pd}\PY{o}{.}\PY{n}{DataFrame}\PY{p}{(}\PY{n}{base}\PY{o}{.}\PY{n}{groupby}\PY{p}{(}\PY{l+s+s1}{\PYZsq{}}\PY{l+s+s1}{CIDADE\PYZus{}USUARIO}\PY{l+s+s1}{\PYZsq{}}\PY{p}{)}\PY{o}{.}\PY{n}{size}\PY{p}{(}\PY{p}{)}\PY{p}{)}
         \PY{n}{df}\PY{o}{.}\PY{n}{columns} \PY{o}{=} \PY{p}{[}\PY{l+s+s1}{\PYZsq{}}\PY{l+s+s1}{Quantidade}\PY{l+s+s1}{\PYZsq{}}\PY{p}{]}
         \PY{n}{df}
         \PY{n}{df}\PY{o}{.}\PY{n}{sort\PYZus{}values}\PY{p}{(}\PY{n}{by} \PY{o}{=}\PY{p}{[}\PY{l+s+s1}{\PYZsq{}}\PY{l+s+s1}{Quantidade}\PY{l+s+s1}{\PYZsq{}}\PY{p}{]}\PY{p}{,} \PY{n}{ascending}\PY{o}{=}\PY{k+kc}{False}\PY{p}{)}
\end{Verbatim}


\begin{Verbatim}[commandchars=\\\{\}]
{\color{outcolor}Out[{\color{outcolor}17}]:}                       Quantidade
         CIDADE\_USUARIO                  
         Vitória da Conquista       23889
         Ibicaraí                     103
         Brumado                       71
         Mortugaba                     35
         Sorocaba                       1
\end{Verbatim}
            
    \begin{Verbatim}[commandchars=\\\{\}]
{\color{incolor}In [{\color{incolor}18}]:} \PY{c+c1}{\PYZsh{}Obter as cidades do usuário e suas respectivas quantidades}
         \PY{c+c1}{\PYZsh{} dado que o tipo de estabelecimento lanchonete.}
         
         \PY{c+c1}{\PYZsh{} Pergunta para ser respondida: Em quais cidades e qual a quantidade, tem lanchonete?}
         
         \PY{n}{filt} \PY{o}{=} \PY{n}{base}\PY{p}{[}\PY{p}{(}\PY{n}{base}\PY{p}{[}\PY{l+s+s1}{\PYZsq{}}\PY{l+s+s1}{TIPO\PYZus{}ESTABELECIMENTO}\PY{l+s+s1}{\PYZsq{}}\PY{p}{]} \PY{o}{==} \PY{l+s+s2}{\PYZdq{}}\PY{l+s+s2}{Lanchonete}\PY{l+s+s2}{\PYZdq{}}\PY{p}{)}\PY{p}{]}
         \PY{n}{dff} \PY{o}{=} \PY{n}{pd}\PY{o}{.}\PY{n}{DataFrame}\PY{p}{(}\PY{n}{filt}\PY{o}{.}\PY{n}{groupby}\PY{p}{(}\PY{l+s+s1}{\PYZsq{}}\PY{l+s+s1}{CIDADE\PYZus{}USUARIO}\PY{l+s+s1}{\PYZsq{}}\PY{p}{)}\PY{o}{.}\PY{n}{size}\PY{p}{(}\PY{p}{)}\PY{p}{)}
         \PY{n}{dff}\PY{o}{.}\PY{n}{columns} \PY{o}{=} \PY{p}{[}\PY{l+s+s1}{\PYZsq{}}\PY{l+s+s1}{Quantidade}\PY{l+s+s1}{\PYZsq{}}\PY{p}{]}
         \PY{n}{dff}\PY{o}{.}\PY{n}{sort\PYZus{}values}\PY{p}{(}\PY{n}{by} \PY{o}{=}\PY{p}{[}\PY{l+s+s1}{\PYZsq{}}\PY{l+s+s1}{Quantidade}\PY{l+s+s1}{\PYZsq{}}\PY{p}{]}\PY{p}{,} \PY{n}{ascending}\PY{o}{=}\PY{k+kc}{False}\PY{p}{)}
\end{Verbatim}


\begin{Verbatim}[commandchars=\\\{\}]
{\color{outcolor}Out[{\color{outcolor}18}]:}                       Quantidade
         CIDADE\_USUARIO                  
         Vitória da Conquista        7065
         Ibicaraí                      62
\end{Verbatim}
            
    \begin{Verbatim}[commandchars=\\\{\}]
{\color{incolor}In [{\color{incolor}19}]:} \PY{c+c1}{\PYZsh{}Pergunta para ser respondida: Quais e quantos são os valores de}
         \PY{c+c1}{\PYZsh{} produtos do tipo de estabelecimento lanchonete?}
         
         
         \PY{n}{filt} \PY{o}{=} \PY{n}{base}\PY{p}{[}\PY{p}{(}\PY{n}{base}\PY{p}{[}\PY{l+s+s1}{\PYZsq{}}\PY{l+s+s1}{TIPO\PYZus{}ESTABELECIMENTO}\PY{l+s+s1}{\PYZsq{}}\PY{p}{]} \PY{o}{==} \PY{l+s+s2}{\PYZdq{}}\PY{l+s+s2}{Lanchonete}\PY{l+s+s2}{\PYZdq{}}\PY{p}{)}\PY{p}{]}
         \PY{n}{dff} \PY{o}{=} \PY{n}{pd}\PY{o}{.}\PY{n}{DataFrame}\PY{p}{(}\PY{n}{filt}\PY{o}{.}\PY{n}{groupby}\PY{p}{(}\PY{l+s+s1}{\PYZsq{}}\PY{l+s+s1}{VALOR\PYZus{}PRODUTOS}\PY{l+s+s1}{\PYZsq{}}\PY{p}{)}\PY{o}{.}\PY{n}{size}\PY{p}{(}\PY{p}{)}\PY{p}{)}
         \PY{n}{dff}\PY{o}{.}\PY{n}{columns} \PY{o}{=} \PY{p}{[}\PY{l+s+s1}{\PYZsq{}}\PY{l+s+s1}{Quantidade}\PY{l+s+s1}{\PYZsq{}}\PY{p}{]}
         \PY{n}{dff}\PY{o}{.}\PY{n}{sort\PYZus{}values}\PY{p}{(}\PY{n}{by} \PY{o}{=}\PY{p}{[}\PY{l+s+s1}{\PYZsq{}}\PY{l+s+s1}{Quantidade}\PY{l+s+s1}{\PYZsq{}}\PY{p}{]}\PY{p}{,} \PY{n}{ascending}\PY{o}{=}\PY{k+kc}{False}\PY{p}{)}
\end{Verbatim}


\begin{Verbatim}[commandchars=\\\{\}]
{\color{outcolor}Out[{\color{outcolor}19}]:}                 Quantidade
         VALOR\_PRODUTOS            
         13.00                  353
         20.90                  327
         11.00                  315
         15.00                  260
         22.00                  240
         19.00                  220
         18.00                  209
         12.00                  192
         14.00                  190
         17.00                  190
         9.50                   189
         20.00                  186
         26.00                  166
         16.00                  158
         10.00                  153
         21.00                  153
         24.00                  141
         12.50                  140
         10.50                  114
         30.00                  109
         15.90                   98
         23.00                   96
         16.50                   96
         14.50                   87
         8.00                    86
         25.00                   85
         30.90                   80
         11.50                   80
         20.50                   74
         16.90                   67
         {\ldots}                    {\ldots}
         42.80                    1
         43.40                    1
         43.80                    1
         47.80                    1
         48.30                    1
         48.70                    1
         48.80                    1
         49.30                    1
         49.50                    1
         38.40                    1
         37.99                    1
         37.90                    1
         34.60                    1
         30.75                    1
         31.25                    1
         31.30                    1
         31.40                    1
         32.40                    1
         33.90                    1
         34.30                    1
         34.95                    1
         37.30                    1
         34.97                    1
         35.40                    1
         35.70                    1
         35.80                    1
         35.90                    1
         36.80                    1
         36.90                    1
         139.70                   1
         
         [321 rows x 1 columns]
\end{Verbatim}
            
    \textbf{1.3} Verificamos que pela quantidade de pedidos realizados sobre
o estabelecimento, lanchonete recebeu a maior parte das vendas mesmo
sendo bastante distribuído os dados referentes aos outros
estabelecimentos.

    \begin{Verbatim}[commandchars=\\\{\}]
{\color{incolor}In [{\color{incolor}20}]:} \PY{c+c1}{\PYZsh{}Pergunta para ser respondida: Quais e quantos tipos}
         \PY{c+c1}{\PYZsh{} de estabelecimentos tem na base de dados?}
         
         \PY{n}{df} \PY{o}{=} \PY{n}{pd}\PY{o}{.}\PY{n}{DataFrame}\PY{p}{(}\PY{n}{base}\PY{o}{.}\PY{n}{groupby}\PY{p}{(}\PY{l+s+s1}{\PYZsq{}}\PY{l+s+s1}{TIPO\PYZus{}ESTABELECIMENTO}\PY{l+s+s1}{\PYZsq{}}\PY{p}{)}\PY{o}{.}\PY{n}{size}\PY{p}{(}\PY{p}{)}\PY{p}{)}
         \PY{n}{df}\PY{o}{.}\PY{n}{columns} \PY{o}{=} \PY{p}{[}\PY{l+s+s1}{\PYZsq{}}\PY{l+s+s1}{Quantidade}\PY{l+s+s1}{\PYZsq{}}\PY{p}{]}
         \PY{n}{df}
         \PY{n}{df}\PY{o}{.}\PY{n}{sort\PYZus{}values}\PY{p}{(}\PY{n}{by} \PY{o}{=}\PY{p}{[}\PY{l+s+s1}{\PYZsq{}}\PY{l+s+s1}{Quantidade}\PY{l+s+s1}{\PYZsq{}}\PY{p}{]}\PY{p}{,} \PY{n}{ascending}\PY{o}{=}\PY{k+kc}{False}\PY{p}{)}
\end{Verbatim}


\begin{Verbatim}[commandchars=\\\{\}]
{\color{outcolor}Out[{\color{outcolor}20}]:}                                Quantidade
         TIPO\_ESTABELECIMENTO                     
         Lanchonete                           7187
         Pizzaria                             3763
         Marmitex                             3597
         Comida Natural                       2135
         Restaurante                          2006
         Comida Japonesa                      1167
         Pizzaria/Esfiharia                   1103
         Restaurante/Tapiocaria                584
         Pizzaria/Lanchonete                   455
         Hot-Dog                               451
         Sopas                                 419
         Culinária Oriental                    334
         Doceria                               274
         Picoleteria                           252
         Espetinhos                            213
         Tapiocaria                            129
         Pães e Bolos                          109
         Pizzaria/Esfiharia/Pastelaria         105
         Saladas                                34
         Comida Árabe                           30
         Acarajé                                12
\end{Verbatim}
            
    \textbf{1.4} Foi verificado também pelo gráfico que a maioria dos
pedidos entregues foram pagos pela forma de pagamento em dinheiro. Mesmo
assim, vimos que poucos pedidos foram recusados.

    \begin{Verbatim}[commandchars=\\\{\}]
{\color{incolor}In [{\color{incolor}21}]:} \PY{n}{plt}\PY{o}{.}\PY{n}{rcParams}\PY{o}{.}\PY{n}{update}\PY{p}{(}\PY{p}{\PYZob{}}\PY{l+s+s1}{\PYZsq{}}\PY{l+s+s1}{text.color}\PY{l+s+s1}{\PYZsq{}}\PY{p}{:} \PY{l+s+s1}{\PYZsq{}}\PY{l+s+s1}{black}\PY{l+s+s1}{\PYZsq{}}\PY{p}{\PYZcb{}}\PY{p}{)}
         \PY{n}{fontsize} \PY{o}{=} \PY{l+m+mi}{14}
         \PY{n}{font\PYZus{}size}\PY{o}{=}\PY{l+m+mi}{14}
         \PY{n}{d1} \PY{o}{=} \PY{n}{pd}\PY{o}{.}\PY{n}{DataFrame}\PY{p}{(}\PY{n}{base}\PY{o}{.}\PY{n}{groupby}\PY{p}{(}\PY{l+s+s1}{\PYZsq{}}\PY{l+s+s1}{FORMA\PYZus{}PAGAMENTO}\PY{l+s+s1}{\PYZsq{}}\PY{p}{)}\PY{o}{.}\PY{n}{size}\PY{p}{(}\PY{p}{)}\PY{p}{)}
         \PY{n}{d1}\PY{o}{.}\PY{n}{columns} \PY{o}{=} \PY{p}{[}\PY{l+s+s1}{\PYZsq{}}\PY{l+s+s1}{Quantidade}\PY{l+s+s1}{\PYZsq{}}\PY{p}{]}
         \PY{n}{fig} \PY{o}{=} \PY{n}{plt}\PY{o}{.}\PY{n}{figure}\PY{p}{(}\PY{n}{figsize}\PY{o}{=}\PY{p}{(}\PY{l+m+mi}{15}\PY{p}{,}\PY{l+m+mi}{16}\PY{p}{)}\PY{p}{)}
         
         
         \PY{n}{colors} \PY{o}{=} \PY{p}{[}\PY{l+s+s1}{\PYZsq{}}\PY{l+s+s1}{yellowgreen}\PY{l+s+s1}{\PYZsq{}}\PY{p}{,} \PY{l+s+s1}{\PYZsq{}}\PY{l+s+s1}{gold}\PY{l+s+s1}{\PYZsq{}}\PY{p}{]}
         \PY{c+c1}{\PYZsh{} plot chart}
         \PY{n}{ax1} \PY{o}{=} \PY{n}{fig}\PY{o}{.}\PY{n}{add\PYZus{}subplot}\PY{p}{(}\PY{l+m+mi}{221}\PY{p}{)}
         \PY{n}{grafico} \PY{o}{=} \PY{n}{d1}\PY{o}{.}\PY{n}{plot}\PY{p}{(}\PY{n}{kind} \PY{o}{=} \PY{l+s+s1}{\PYZsq{}}\PY{l+s+s1}{pie}\PY{l+s+s1}{\PYZsq{}}\PY{p}{,} \PY{n}{autopct}\PY{o}{=}\PY{l+s+s1}{\PYZsq{}}\PY{l+s+si}{\PYZpc{}1.1f}\PY{l+s+si}{\PYZpc{}\PYZpc{}}\PY{l+s+s1}{\PYZsq{}}\PY{p}{,} 
                           \PY{n}{startangle}\PY{o}{=}\PY{l+m+mi}{90}\PY{p}{,} \PY{n}{shadow}\PY{o}{=}\PY{k+kc}{False}\PY{p}{,}
                          \PY{n}{figsize}\PY{o}{=}\PY{p}{(}\PY{l+m+mi}{5}\PY{p}{,}\PY{l+m+mi}{5}\PY{p}{)}\PY{p}{,} \PY{n}{fontsize}\PY{o}{=} \PY{n}{fontsize}\PY{p}{,} 
                           \PY{n}{colors} \PY{o}{=} \PY{n}{colors}\PY{p}{,} 
                           \PY{n}{legend} \PY{o}{=} \PY{k+kc}{True}\PY{p}{,}   \PY{n}{ax}\PY{o}{=}\PY{n}{ax1}\PY{p}{,} \PY{n}{subplots}\PY{o}{=}\PY{k+kc}{True}\PY{p}{)}
         \PY{n}{plt}\PY{o}{.}\PY{n}{title}\PY{p}{(}\PY{l+s+s1}{\PYZsq{}}\PY{l+s+s1}{Formas de pagamento}\PY{l+s+s1}{\PYZsq{}}\PY{p}{)}
         \PY{n}{plt}\PY{o}{.}\PY{n}{ylabel}\PY{p}{(}\PY{l+s+s1}{\PYZsq{}}\PY{l+s+s1}{\PYZsq{}}\PY{p}{)}
         \PY{n}{plt}\PY{o}{.}\PY{n}{legend}\PY{p}{(}\PY{n}{loc}\PY{o}{=}\PY{l+m+mi}{2}\PY{p}{,} \PY{n}{prop}\PY{o}{=}\PY{p}{\PYZob{}}\PY{l+s+s1}{\PYZsq{}}\PY{l+s+s1}{size}\PY{l+s+s1}{\PYZsq{}}\PY{p}{:} \PY{n}{fontsize}\PY{p}{\PYZcb{}}\PY{p}{)}
         
         \PY{n}{d2} \PY{o}{=} \PY{n}{pd}\PY{o}{.}\PY{n}{DataFrame}\PY{p}{(}\PY{n}{base}\PY{o}{.}\PY{n}{groupby}\PY{p}{(}\PY{l+s+s1}{\PYZsq{}}\PY{l+s+s1}{STATUS}\PY{l+s+s1}{\PYZsq{}}\PY{p}{)}\PY{o}{.}\PY{n}{size}\PY{p}{(}\PY{p}{)}\PY{p}{)}
         \PY{n}{d2}\PY{o}{.}\PY{n}{columns} \PY{o}{=} \PY{p}{[}\PY{l+s+s1}{\PYZsq{}}\PY{l+s+s1}{Quantidade}\PY{l+s+s1}{\PYZsq{}}\PY{p}{]}
         
         \PY{n}{ax2} \PY{o}{=} \PY{n}{fig}\PY{o}{.}\PY{n}{add\PYZus{}subplot}\PY{p}{(}\PY{l+m+mi}{222}\PY{p}{)}
         \PY{n}{grafico} \PY{o}{=} \PY{n}{d2}\PY{o}{.}\PY{n}{plot}\PY{p}{(}\PY{n}{kind} \PY{o}{=} \PY{l+s+s1}{\PYZsq{}}\PY{l+s+s1}{pie}\PY{l+s+s1}{\PYZsq{}}\PY{p}{,}\PY{n}{autopct}\PY{o}{=}\PY{l+s+s1}{\PYZsq{}}\PY{l+s+si}{\PYZpc{}1.1f}\PY{l+s+si}{\PYZpc{}\PYZpc{}}\PY{l+s+s1}{\PYZsq{}}\PY{p}{,} 
                           \PY{n}{startangle}\PY{o}{=}\PY{l+m+mi}{90}\PY{p}{,} \PY{n}{shadow}\PY{o}{=}\PY{k+kc}{False}\PY{p}{,} 
                            \PY{n}{figsize}\PY{o}{=}\PY{p}{(}\PY{l+m+mi}{5}\PY{p}{,}\PY{l+m+mi}{5}\PY{p}{)}\PY{p}{,}\PY{n}{fontsize}\PY{o}{=}\PY{n}{fontsize}\PY{p}{,} 
                           \PY{n}{colors} \PY{o}{=} \PY{n}{colors}\PY{p}{,} 
                           \PY{n}{legend} \PY{o}{=} \PY{k+kc}{True}\PY{p}{,}   \PY{n}{ax}\PY{o}{=}\PY{n}{ax2}\PY{p}{,} \PY{n}{subplots}\PY{o}{=}\PY{k+kc}{True}\PY{p}{)}
         \PY{n}{plt}\PY{o}{.}\PY{n}{title}\PY{p}{(}\PY{l+s+s1}{\PYZsq{}}\PY{l+s+s1}{Status do pedido}\PY{l+s+s1}{\PYZsq{}}\PY{p}{)}
         \PY{n}{plt}\PY{o}{.}\PY{n}{ylabel}\PY{p}{(}\PY{l+s+s1}{\PYZsq{}}\PY{l+s+s1}{\PYZsq{}}\PY{p}{)}
         \PY{n}{plt}\PY{o}{.}\PY{n}{legend}\PY{p}{(}\PY{n}{loc}\PY{o}{=}\PY{l+m+mi}{2}\PY{p}{,} \PY{n}{prop}\PY{o}{=}\PY{p}{\PYZob{}}\PY{l+s+s1}{\PYZsq{}}\PY{l+s+s1}{size}\PY{l+s+s1}{\PYZsq{}}\PY{p}{:} \PY{n}{fontsize}\PY{p}{\PYZcb{}}\PY{p}{)}
         
         \PY{c+c1}{\PYZsh{}ax3 = fig.add\PYZus{}subplot(223)}
         
         \PY{c+c1}{\PYZsh{}ax3.axis(\PYZsq{}off\PYZsq{})}
         \PY{n}{mpl\PYZus{}table} \PY{o}{=} \PY{n}{table}\PY{p}{(}\PY{n}{ax2}\PY{p}{,} \PY{n}{d2}\PY{p}{,} \PY{n}{loc}\PY{o}{=}\PY{l+s+s1}{\PYZsq{}}\PY{l+s+s1}{bottom}\PY{l+s+s1}{\PYZsq{}}\PY{p}{,} 
                           \PY{n}{rowLoc}\PY{o}{=}\PY{l+s+s1}{\PYZsq{}}\PY{l+s+s1}{left}\PY{l+s+s1}{\PYZsq{}}\PY{p}{,} 
                           \PY{n}{colLoc} \PY{o}{=} \PY{l+s+s1}{\PYZsq{}}\PY{l+s+s1}{center}\PY{l+s+s1}{\PYZsq{}}\PY{p}{)}
         \PY{n}{mpl\PYZus{}table}\PY{o}{.}\PY{n}{auto\PYZus{}set\PYZus{}font\PYZus{}size}\PY{p}{(}\PY{k+kc}{False}\PY{p}{)}
         \PY{n}{mpl\PYZus{}table}\PY{o}{.}\PY{n}{set\PYZus{}fontsize}\PY{p}{(}\PY{n}{font\PYZus{}size}\PY{p}{)}
         \PY{n}{mpl\PYZus{}table}\PY{o}{.}\PY{n}{scale}\PY{p}{(}\PY{l+m+mf}{0.4}\PY{p}{,}\PY{l+m+mf}{2.8}\PY{p}{)}
         
         \PY{c+c1}{\PYZsh{}ax4 = fig.add\PYZus{}subplot(224)}
         
         \PY{c+c1}{\PYZsh{}ax4.axis(\PYZsq{}off\PYZsq{})}
         \PY{n}{mpl\PYZus{}table} \PY{o}{=} \PY{n}{table}\PY{p}{(}\PY{n}{ax1}\PY{p}{,} \PY{n}{d1}\PY{p}{,} \PY{n}{loc}\PY{o}{=}\PY{l+s+s1}{\PYZsq{}}\PY{l+s+s1}{bottom}\PY{l+s+s1}{\PYZsq{}}\PY{p}{,} 
                           \PY{n}{rowLoc}\PY{o}{=}\PY{l+s+s1}{\PYZsq{}}\PY{l+s+s1}{left}\PY{l+s+s1}{\PYZsq{}}\PY{p}{,} 
                           \PY{n}{colLoc} \PY{o}{=} \PY{l+s+s1}{\PYZsq{}}\PY{l+s+s1}{center}\PY{l+s+s1}{\PYZsq{}}\PY{p}{)}
         \PY{n}{mpl\PYZus{}table}\PY{o}{.}\PY{n}{auto\PYZus{}set\PYZus{}font\PYZus{}size}\PY{p}{(}\PY{k+kc}{False}\PY{p}{)}
         \PY{n}{mpl\PYZus{}table}\PY{o}{.}\PY{n}{set\PYZus{}fontsize}\PY{p}{(}\PY{n}{font\PYZus{}size}\PY{p}{)}
         \PY{n}{mpl\PYZus{}table}\PY{o}{.}\PY{n}{scale}\PY{p}{(}\PY{l+m+mf}{0.4}\PY{p}{,}\PY{l+m+mf}{2.8}\PY{p}{)}
\end{Verbatim}


    \begin{center}
    \adjustimage{max size={0.9\linewidth}{0.9\paperheight}}{output_22_0.png}
    \end{center}
    { \hspace*{\fill} \\}
    
    \textbf{1.2} Neste gráfico de Boxplot foi possível verificar alguns
atributos e a sua distribuição no caso do ``valor dos produtos'', ``taxa
de entrega'' e do ``total de pedidos'', em que vimos a grande variedade
entres as amostras coletadas, os valores discrepantes e a divisão dos
dados quanto aos atributos mencionados.

    \begin{Verbatim}[commandchars=\\\{\}]
{\color{incolor}In [{\color{incolor}22}]:} \PY{c+c1}{\PYZsh{}Boxplot para verificar os valores discrepantes e a divisão dos dados}
         \PY{c+c1}{\PYZsh{} do atributos mencionados}
         
         \PY{n}{fig} \PY{o}{=} \PY{n}{plt}\PY{o}{.}\PY{n}{figure}\PY{p}{(}\PY{n}{figsize}\PY{o}{=}\PY{p}{(}\PY{l+m+mi}{15}\PY{p}{,}\PY{l+m+mi}{20}\PY{p}{)}\PY{p}{)}
         \PY{n}{ax5} \PY{o}{=} \PY{n}{fig}\PY{o}{.}\PY{n}{add\PYZus{}subplot}\PY{p}{(}\PY{l+m+mi}{221}\PY{p}{)}
         \PY{n}{grafico} \PY{o}{=} \PY{n}{base}\PY{p}{[}\PY{p}{[}\PY{l+s+s1}{\PYZsq{}}\PY{l+s+s1}{VALOR\PYZus{}PRODUTOS}\PY{l+s+s1}{\PYZsq{}}\PY{p}{]}\PY{p}{]}\PY{o}{.}\PY{n}{boxplot}\PY{p}{(}\PY{n}{figsize}\PY{o}{=}\PY{p}{(}\PY{l+m+mi}{20}\PY{p}{,}\PY{l+m+mi}{5}\PY{p}{)}\PY{p}{,} 
                                                    \PY{n}{fontsize}\PY{o}{=} \PY{n}{fontsize}\PY{p}{,}   \PY{n}{ax}\PY{o}{=}\PY{n}{ax5}\PY{p}{)}
         \PY{n}{plt}\PY{o}{.}\PY{n}{title}\PY{p}{(}\PY{l+s+s1}{\PYZsq{}}\PY{l+s+s1}{Valor dos produtos}\PY{l+s+s1}{\PYZsq{}}\PY{p}{)}
         
         
         \PY{n}{ax6} \PY{o}{=} \PY{n}{fig}\PY{o}{.}\PY{n}{add\PYZus{}subplot}\PY{p}{(}\PY{l+m+mi}{222}\PY{p}{)}
         \PY{n}{grafico} \PY{o}{=} \PY{n}{base}\PY{p}{[}\PY{p}{[}\PY{l+s+s1}{\PYZsq{}}\PY{l+s+s1}{TAXA\PYZus{}ENTREGA}\PY{l+s+s1}{\PYZsq{}}\PY{p}{]}\PY{p}{]}\PY{o}{.}\PY{n}{boxplot}\PY{p}{(}\PY{n}{figsize}\PY{o}{=}\PY{p}{(}\PY{l+m+mi}{20}\PY{p}{,}\PY{l+m+mi}{5}\PY{p}{)}\PY{p}{,} 
                                                  \PY{n}{fontsize}\PY{o}{=} \PY{n}{fontsize}\PY{p}{,}   \PY{n}{ax}\PY{o}{=}\PY{n}{ax6}\PY{p}{)}
         \PY{n}{plt}\PY{o}{.}\PY{n}{title}\PY{p}{(}\PY{l+s+s1}{\PYZsq{}}\PY{l+s+s1}{Taxa de entrega}\PY{l+s+s1}{\PYZsq{}}\PY{p}{)}
         
         
         \PY{n}{ax7} \PY{o}{=} \PY{n}{fig}\PY{o}{.}\PY{n}{add\PYZus{}subplot}\PY{p}{(}\PY{l+m+mi}{223}\PY{p}{)}
         \PY{n}{grafico} \PY{o}{=} \PY{n}{base}\PY{p}{[}\PY{p}{[}\PY{l+s+s1}{\PYZsq{}}\PY{l+s+s1}{TOTAL\PYZus{}PEDIDO}\PY{l+s+s1}{\PYZsq{}}\PY{p}{]}\PY{p}{]}\PY{o}{.}\PY{n}{boxplot}\PY{p}{(}\PY{n}{figsize}\PY{o}{=}\PY{p}{(}\PY{l+m+mi}{20}\PY{p}{,}\PY{l+m+mi}{5}\PY{p}{)}\PY{p}{,} 
                                                  \PY{n}{fontsize}\PY{o}{=} \PY{n}{fontsize}\PY{p}{,}   \PY{n}{ax}\PY{o}{=}\PY{n}{ax7}\PY{p}{)}
         \PY{n}{plt}\PY{o}{.}\PY{n}{title}\PY{p}{(}\PY{l+s+s1}{\PYZsq{}}\PY{l+s+s1}{Total de pedidos}\PY{l+s+s1}{\PYZsq{}}\PY{p}{)}
\end{Verbatim}


\begin{Verbatim}[commandchars=\\\{\}]
{\color{outcolor}Out[{\color{outcolor}22}]:} Text(0.5,1,'Total de pedidos')
\end{Verbatim}
            
    \begin{center}
    \adjustimage{max size={0.9\linewidth}{0.9\paperheight}}{output_24_1.png}
    \end{center}
    { \hspace*{\fill} \\}
    
    \begin{Verbatim}[commandchars=\\\{\}]
{\color{incolor}In [{\color{incolor}23}]:} \PY{c+c1}{\PYZsh{}Vendas pelo perído mencionado}
         
         \PY{c+c1}{\PYZsh{}plt.rcParams.update(\PYZob{}\PYZsq{}font.size\PYZsq{}: 10\PYZcb{})}
         
         
         \PY{n}{fig} \PY{o}{=} \PY{n}{plt}\PY{o}{.}\PY{n}{figure}\PY{p}{(}\PY{n}{figsize}\PY{o}{=}\PY{p}{(}\PY{l+m+mi}{15}\PY{p}{,}\PY{l+m+mi}{10}\PY{p}{)}\PY{p}{)}
         \PY{c+c1}{\PYZsh{}ax = fig.add\PYZus{}subplot(111)}
         \PY{n}{filtrado} \PY{o}{=} \PY{n}{base}\PY{p}{[}\PY{p}{(}\PY{n}{base}\PY{p}{[}\PY{l+s+s1}{\PYZsq{}}\PY{l+s+s1}{DATA\PYZus{}PEDIDO}\PY{l+s+s1}{\PYZsq{}}\PY{p}{]} \PY{o}{\PYZgt{}} \PY{l+s+s1}{\PYZsq{}}\PY{l+s+s1}{2016\PYZhy{}07\PYZhy{}05}\PY{l+s+s1}{\PYZsq{}}\PY{p}{)} 
                         \PY{o}{\PYZam{}} \PY{p}{(}\PY{n}{base}\PY{p}{[}\PY{l+s+s1}{\PYZsq{}}\PY{l+s+s1}{DATA\PYZus{}PEDIDO}\PY{l+s+s1}{\PYZsq{}}\PY{p}{]} \PY{o}{\PYZlt{}} \PY{l+s+s1}{\PYZsq{}}\PY{l+s+s1}{2016\PYZhy{}07\PYZhy{}15}\PY{l+s+s1}{\PYZsq{}}\PY{p}{)}\PY{p}{]}
         \PY{n}{time} \PY{o}{=} \PY{n}{pd}\PY{o}{.}\PY{n}{DataFrame}\PY{p}{(}\PY{n}{filtrado}\PY{o}{.}\PY{n}{groupby}\PY{p}{(}\PY{l+s+s1}{\PYZsq{}}\PY{l+s+s1}{DATA\PYZus{}PEDIDO}\PY{l+s+s1}{\PYZsq{}}\PY{p}{)}\PY{o}{.}\PY{n}{size}\PY{p}{(}\PY{p}{)}\PY{p}{)}
         \PY{n}{time}\PY{o}{.}\PY{n}{columns} \PY{o}{=} \PY{p}{[}\PY{l+s+s1}{\PYZsq{}}\PY{l+s+s1}{Quantidade}\PY{l+s+s1}{\PYZsq{}}\PY{p}{]}
         \PY{n}{grafico} \PY{o}{=} \PY{n}{plt}\PY{o}{.}\PY{n}{plot}\PY{p}{(}\PY{n}{time} \PY{p}{,} \PY{n}{linewidth}\PY{o}{=}\PY{l+m+mi}{3}\PY{p}{,} \PY{n}{markersize}\PY{o}{=}\PY{l+m+mi}{20}\PY{p}{)} 
         \PY{n}{plt}\PY{o}{.}\PY{n}{title}\PY{p}{(}\PY{l+s+s1}{\PYZsq{}}\PY{l+s+s1}{Perído de pedidos}\PY{l+s+s1}{\PYZsq{}}\PY{p}{)}
         
         \PY{c+c1}{\PYZsh{}time}
\end{Verbatim}


\begin{Verbatim}[commandchars=\\\{\}]
{\color{outcolor}Out[{\color{outcolor}23}]:} Text(0.5,1,'Perído de pedidos')
\end{Verbatim}
            
    \begin{center}
    \adjustimage{max size={0.9\linewidth}{0.9\paperheight}}{output_25_1.png}
    \end{center}
    { \hspace*{\fill} \\}
    
    \begin{Verbatim}[commandchars=\\\{\}]
{\color{incolor}In [{\color{incolor}24}]:} \PY{c+c1}{\PYZsh{}Verificando se a Data do pedido informada na base,}
         \PY{c+c1}{\PYZsh{} confere com o Dia do pedido informado.}
         
         \PY{c+c1}{\PYZsh{}Pergunta para ser respondida: O dia do pedido confere com a data do pedido informada?}
         
         \PY{n}{df1} \PY{o}{=} \PY{n}{pd}\PY{o}{.}\PY{n}{DataFrame}\PY{p}{(}\PY{n}{base}\PY{p}{[}\PY{l+s+s1}{\PYZsq{}}\PY{l+s+s1}{DATA\PYZus{}PEDIDO}\PY{l+s+s1}{\PYZsq{}}\PY{p}{]}\PY{p}{)}
         \PY{n}{df1}\PY{o}{.}\PY{n}{columns} \PY{o}{=} \PY{p}{[}\PY{l+s+s1}{\PYZsq{}}\PY{l+s+s1}{Data do pedido}\PY{l+s+s1}{\PYZsq{}}\PY{p}{]}
         
         \PY{n}{df2} \PY{o}{=} \PY{n}{pd}\PY{o}{.}\PY{n}{DataFrame}\PY{p}{(}\PY{n}{base}\PY{p}{[}\PY{l+s+s1}{\PYZsq{}}\PY{l+s+s1}{DIA\PYZus{}PEDIDO}\PY{l+s+s1}{\PYZsq{}}\PY{p}{]}\PY{p}{)}
         \PY{n}{df2}\PY{o}{.}\PY{n}{columns} \PY{o}{=} \PY{p}{[}\PY{l+s+s1}{\PYZsq{}}\PY{l+s+s1}{Dia do pedido (Base de dados)}\PY{l+s+s1}{\PYZsq{}}\PY{p}{]}
         
         \PY{n}{df3} \PY{o}{=} \PY{n}{pd}\PY{o}{.}\PY{n}{DataFrame}\PY{p}{(}\PY{n}{df1}\PY{o}{.}\PY{n}{apply}\PY{p}{(}\PY{k}{lambda} \PY{n}{x}\PY{p}{:} 
                                      \PY{n}{datetime}\PY{o}{.}\PY{n}{datetime}
                                      \PY{o}{.}\PY{n}{strptime}\PY{p}{(}\PY{n}{x}\PY{p}{[}\PY{l+s+s1}{\PYZsq{}}\PY{l+s+s1}{Data do pedido}\PY{l+s+s1}{\PYZsq{}}\PY{p}{]}\PY{p}{,}\PY{l+s+s1}{\PYZsq{}}\PY{l+s+s1}{\PYZpc{}}\PY{l+s+s1}{Y\PYZhy{}}\PY{l+s+s1}{\PYZpc{}}\PY{l+s+s1}{m\PYZhy{}}\PY{l+s+si}{\PYZpc{}d}\PY{l+s+s1}{\PYZsq{}}\PY{p}{)}
                                      \PY{o}{.}\PY{n}{strftime}\PY{p}{(}\PY{l+s+s1}{\PYZsq{}}\PY{l+s+s1}{\PYZpc{}}\PY{l+s+s1}{A}\PY{l+s+s1}{\PYZsq{}}\PY{p}{)}\PY{p}{,} \PY{n}{axis}\PY{o}{=}\PY{l+m+mi}{1}\PY{p}{)}\PY{p}{)}
         \PY{n}{df3}\PY{o}{.}\PY{n}{columns} \PY{o}{=} \PY{p}{[}\PY{l+s+s1}{\PYZsq{}}\PY{l+s+s1}{Dia do pedido (Verificado)}\PY{l+s+s1}{\PYZsq{}}\PY{p}{]}
         \PY{n}{df1} \PY{o}{=} \PY{n}{df1}\PY{o}{.}\PY{n}{join}\PY{p}{(}\PY{n}{df2}\PY{p}{)}
         \PY{n}{df1} \PY{o}{=} \PY{n}{df1}\PY{o}{.}\PY{n}{join}\PY{p}{(}\PY{n}{df3}\PY{p}{)}
         \PY{n}{df1}\PY{o}{.}\PY{n}{head}\PY{p}{(}\PY{l+m+mi}{10}\PY{p}{)}
\end{Verbatim}


\begin{Verbatim}[commandchars=\\\{\}]
{\color{outcolor}Out[{\color{outcolor}24}]:}   Data do pedido Dia do pedido (Base de dados) Dia do pedido (Verificado)
         0     2016-07-05                        Sunday                    Tuesday
         1     2016-07-05                        Sunday                    Tuesday
         2     2016-07-05                        Sunday                    Tuesday
         3     2016-07-06                        Monday                  Wednesday
         4     2016-07-07                       Tuesday                   Thursday
         5     2016-07-08                     Wednesday                     Friday
         6     2016-07-09                      Thursday                   Saturday
         7     2016-07-10                        Friday                     Sunday
         8     2016-07-12                        Sunday                    Tuesday
         9     2016-07-15                     Wednesday                     Friday
\end{Verbatim}
            
    \begin{Verbatim}[commandchars=\\\{\}]
{\color{incolor}In [{\color{incolor}25}]:} \PY{c+c1}{\PYZsh{}Pergunta para ser respondida: Todos os registros estão }
         \PY{c+c1}{\PYZsh{} com o dia do pedido informado \PYZdq{}errado\PYZdq{}?}
         \PY{n}{filtrado} \PY{o}{=} \PY{n}{df1}\PY{p}{[}\PY{p}{(}\PY{n}{df1}\PY{p}{[}\PY{l+s+s1}{\PYZsq{}}\PY{l+s+s1}{Dia do pedido (Base de dados)}\PY{l+s+s1}{\PYZsq{}}\PY{p}{]}
                         \PY{o}{==} \PY{n}{df1}\PY{p}{[}\PY{l+s+s1}{\PYZsq{}}\PY{l+s+s1}{Dia do pedido (Verificado)}\PY{l+s+s1}{\PYZsq{}}\PY{p}{]}\PY{p}{)}\PY{p}{]}
         \PY{n+nb}{len}\PY{p}{(}\PY{n}{filtrado}\PY{p}{)}
\end{Verbatim}


\begin{Verbatim}[commandchars=\\\{\}]
{\color{outcolor}Out[{\color{outcolor}25}]:} 0
\end{Verbatim}
            
    \begin{Verbatim}[commandchars=\\\{\}]
{\color{incolor}In [{\color{incolor}26}]:} \PY{c+c1}{\PYZsh{} Quantos pedidos receberam avaliação?}
         \PY{n}{filtrado} \PY{o}{=} \PY{n}{base}\PY{p}{[}\PY{p}{(}\PY{n}{base}\PY{p}{[}\PY{l+s+s1}{\PYZsq{}}\PY{l+s+s1}{AVALIACAO}\PY{l+s+s1}{\PYZsq{}}\PY{p}{]}\PY{o}{.}\PY{n}{isnull}\PY{p}{(}\PY{p}{)} \PY{o}{==} \PY{k+kc}{False} \PY{p}{)}\PY{p}{]}
         \PY{n+nb}{len}\PY{p}{(}\PY{n}{filtrado}\PY{p}{)}
\end{Verbatim}


\begin{Verbatim}[commandchars=\\\{\}]
{\color{outcolor}Out[{\color{outcolor}26}]:} 3625
\end{Verbatim}
            

    % Add a bibliography block to the postdoc
    
    
    
    \end{document}
