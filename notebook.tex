
% Default to the notebook output style

    


% Inherit from the specified cell style.




    
\documentclass[11pt]{article}

    
    
    \usepackage[T1]{fontenc}
    % Nicer default font (+ math font) than Computer Modern for most use cases
    \usepackage{mathpazo}

    % Basic figure setup, for now with no caption control since it's done
    % automatically by Pandoc (which extracts ![](path) syntax from Markdown).
    \usepackage{graphicx}
    % We will generate all images so they have a width \maxwidth. This means
    % that they will get their normal width if they fit onto the page, but
    % are scaled down if they would overflow the margins.
    \makeatletter
    \def\maxwidth{\ifdim\Gin@nat@width>\linewidth\linewidth
    \else\Gin@nat@width\fi}
    \makeatother
    \let\Oldincludegraphics\includegraphics
    % Set max figure width to be 80% of text width, for now hardcoded.
    \renewcommand{\includegraphics}[1]{\Oldincludegraphics[width=.8\maxwidth]{#1}}
    % Ensure that by default, figures have no caption (until we provide a
    % proper Figure object with a Caption API and a way to capture that
    % in the conversion process - todo).
    \usepackage{caption}
    \DeclareCaptionLabelFormat{nolabel}{}
    \captionsetup{labelformat=nolabel}

    \usepackage{adjustbox} % Used to constrain images to a maximum size 
    \usepackage{xcolor} % Allow colors to be defined
    \usepackage{enumerate} % Needed for markdown enumerations to work
    \usepackage{geometry} % Used to adjust the document margins
    \usepackage{amsmath} % Equations
    \usepackage{amssymb} % Equations
    \usepackage{textcomp} % defines textquotesingle
    % Hack from http://tex.stackexchange.com/a/47451/13684:
    \AtBeginDocument{%
        \def\PYZsq{\textquotesingle}% Upright quotes in Pygmentized code
    }
    \usepackage{upquote} % Upright quotes for verbatim code
    \usepackage{eurosym} % defines \euro
    \usepackage[mathletters]{ucs} % Extended unicode (utf-8) support
    \usepackage[utf8x]{inputenc} % Allow utf-8 characters in the tex document
    \usepackage{fancyvrb} % verbatim replacement that allows latex
    \usepackage{grffile} % extends the file name processing of package graphics 
                         % to support a larger range 
    % The hyperref package gives us a pdf with properly built
    % internal navigation ('pdf bookmarks' for the table of contents,
    % internal cross-reference links, web links for URLs, etc.)
    \usepackage{hyperref}
    \usepackage{longtable} % longtable support required by pandoc >1.10
    \usepackage{booktabs}  % table support for pandoc > 1.12.2
    \usepackage[inline]{enumitem} % IRkernel/repr support (it uses the enumerate* environment)
    \usepackage[normalem]{ulem} % ulem is needed to support strikethroughs (\sout)
                                % normalem makes italics be italics, not underlines
    

    
    
    % Colors for the hyperref package
    \definecolor{urlcolor}{rgb}{0,.145,.698}
    \definecolor{linkcolor}{rgb}{.71,0.21,0.01}
    \definecolor{citecolor}{rgb}{.12,.54,.11}

    % ANSI colors
    \definecolor{ansi-black}{HTML}{3E424D}
    \definecolor{ansi-black-intense}{HTML}{282C36}
    \definecolor{ansi-red}{HTML}{E75C58}
    \definecolor{ansi-red-intense}{HTML}{B22B31}
    \definecolor{ansi-green}{HTML}{00A250}
    \definecolor{ansi-green-intense}{HTML}{007427}
    \definecolor{ansi-yellow}{HTML}{DDB62B}
    \definecolor{ansi-yellow-intense}{HTML}{B27D12}
    \definecolor{ansi-blue}{HTML}{208FFB}
    \definecolor{ansi-blue-intense}{HTML}{0065CA}
    \definecolor{ansi-magenta}{HTML}{D160C4}
    \definecolor{ansi-magenta-intense}{HTML}{A03196}
    \definecolor{ansi-cyan}{HTML}{60C6C8}
    \definecolor{ansi-cyan-intense}{HTML}{258F8F}
    \definecolor{ansi-white}{HTML}{C5C1B4}
    \definecolor{ansi-white-intense}{HTML}{A1A6B2}

    % commands and environments needed by pandoc snippets
    % extracted from the output of `pandoc -s`
    \providecommand{\tightlist}{%
      \setlength{\itemsep}{0pt}\setlength{\parskip}{0pt}}
    \DefineVerbatimEnvironment{Highlighting}{Verbatim}{commandchars=\\\{\}}
    % Add ',fontsize=\small' for more characters per line
    \newenvironment{Shaded}{}{}
    \newcommand{\KeywordTok}[1]{\textcolor[rgb]{0.00,0.44,0.13}{\textbf{{#1}}}}
    \newcommand{\DataTypeTok}[1]{\textcolor[rgb]{0.56,0.13,0.00}{{#1}}}
    \newcommand{\DecValTok}[1]{\textcolor[rgb]{0.25,0.63,0.44}{{#1}}}
    \newcommand{\BaseNTok}[1]{\textcolor[rgb]{0.25,0.63,0.44}{{#1}}}
    \newcommand{\FloatTok}[1]{\textcolor[rgb]{0.25,0.63,0.44}{{#1}}}
    \newcommand{\CharTok}[1]{\textcolor[rgb]{0.25,0.44,0.63}{{#1}}}
    \newcommand{\StringTok}[1]{\textcolor[rgb]{0.25,0.44,0.63}{{#1}}}
    \newcommand{\CommentTok}[1]{\textcolor[rgb]{0.38,0.63,0.69}{\textit{{#1}}}}
    \newcommand{\OtherTok}[1]{\textcolor[rgb]{0.00,0.44,0.13}{{#1}}}
    \newcommand{\AlertTok}[1]{\textcolor[rgb]{1.00,0.00,0.00}{\textbf{{#1}}}}
    \newcommand{\FunctionTok}[1]{\textcolor[rgb]{0.02,0.16,0.49}{{#1}}}
    \newcommand{\RegionMarkerTok}[1]{{#1}}
    \newcommand{\ErrorTok}[1]{\textcolor[rgb]{1.00,0.00,0.00}{\textbf{{#1}}}}
    \newcommand{\NormalTok}[1]{{#1}}
    
    % Additional commands for more recent versions of Pandoc
    \newcommand{\ConstantTok}[1]{\textcolor[rgb]{0.53,0.00,0.00}{{#1}}}
    \newcommand{\SpecialCharTok}[1]{\textcolor[rgb]{0.25,0.44,0.63}{{#1}}}
    \newcommand{\VerbatimStringTok}[1]{\textcolor[rgb]{0.25,0.44,0.63}{{#1}}}
    \newcommand{\SpecialStringTok}[1]{\textcolor[rgb]{0.73,0.40,0.53}{{#1}}}
    \newcommand{\ImportTok}[1]{{#1}}
    \newcommand{\DocumentationTok}[1]{\textcolor[rgb]{0.73,0.13,0.13}{\textit{{#1}}}}
    \newcommand{\AnnotationTok}[1]{\textcolor[rgb]{0.38,0.63,0.69}{\textbf{\textit{{#1}}}}}
    \newcommand{\CommentVarTok}[1]{\textcolor[rgb]{0.38,0.63,0.69}{\textbf{\textit{{#1}}}}}
    \newcommand{\VariableTok}[1]{\textcolor[rgb]{0.10,0.09,0.49}{{#1}}}
    \newcommand{\ControlFlowTok}[1]{\textcolor[rgb]{0.00,0.44,0.13}{\textbf{{#1}}}}
    \newcommand{\OperatorTok}[1]{\textcolor[rgb]{0.40,0.40,0.40}{{#1}}}
    \newcommand{\BuiltInTok}[1]{{#1}}
    \newcommand{\ExtensionTok}[1]{{#1}}
    \newcommand{\PreprocessorTok}[1]{\textcolor[rgb]{0.74,0.48,0.00}{{#1}}}
    \newcommand{\AttributeTok}[1]{\textcolor[rgb]{0.49,0.56,0.16}{{#1}}}
    \newcommand{\InformationTok}[1]{\textcolor[rgb]{0.38,0.63,0.69}{\textbf{\textit{{#1}}}}}
    \newcommand{\WarningTok}[1]{\textcolor[rgb]{0.38,0.63,0.69}{\textbf{\textit{{#1}}}}}
    
    
    % Define a nice break command that doesn't care if a line doesn't already
    % exist.
    \def\br{\hspace*{\fill} \\* }
    % Math Jax compatability definitions
    \def\gt{>}
    \def\lt{<}
    % Document parameters
    \title{Business Intelligence}
    
    
    

    % Pygments definitions
    
\makeatletter
\def\PY@reset{\let\PY@it=\relax \let\PY@bf=\relax%
    \let\PY@ul=\relax \let\PY@tc=\relax%
    \let\PY@bc=\relax \let\PY@ff=\relax}
\def\PY@tok#1{\csname PY@tok@#1\endcsname}
\def\PY@toks#1+{\ifx\relax#1\empty\else%
    \PY@tok{#1}\expandafter\PY@toks\fi}
\def\PY@do#1{\PY@bc{\PY@tc{\PY@ul{%
    \PY@it{\PY@bf{\PY@ff{#1}}}}}}}
\def\PY#1#2{\PY@reset\PY@toks#1+\relax+\PY@do{#2}}

\expandafter\def\csname PY@tok@w\endcsname{\def\PY@tc##1{\textcolor[rgb]{0.73,0.73,0.73}{##1}}}
\expandafter\def\csname PY@tok@c\endcsname{\let\PY@it=\textit\def\PY@tc##1{\textcolor[rgb]{0.25,0.50,0.50}{##1}}}
\expandafter\def\csname PY@tok@cp\endcsname{\def\PY@tc##1{\textcolor[rgb]{0.74,0.48,0.00}{##1}}}
\expandafter\def\csname PY@tok@k\endcsname{\let\PY@bf=\textbf\def\PY@tc##1{\textcolor[rgb]{0.00,0.50,0.00}{##1}}}
\expandafter\def\csname PY@tok@kp\endcsname{\def\PY@tc##1{\textcolor[rgb]{0.00,0.50,0.00}{##1}}}
\expandafter\def\csname PY@tok@kt\endcsname{\def\PY@tc##1{\textcolor[rgb]{0.69,0.00,0.25}{##1}}}
\expandafter\def\csname PY@tok@o\endcsname{\def\PY@tc##1{\textcolor[rgb]{0.40,0.40,0.40}{##1}}}
\expandafter\def\csname PY@tok@ow\endcsname{\let\PY@bf=\textbf\def\PY@tc##1{\textcolor[rgb]{0.67,0.13,1.00}{##1}}}
\expandafter\def\csname PY@tok@nb\endcsname{\def\PY@tc##1{\textcolor[rgb]{0.00,0.50,0.00}{##1}}}
\expandafter\def\csname PY@tok@nf\endcsname{\def\PY@tc##1{\textcolor[rgb]{0.00,0.00,1.00}{##1}}}
\expandafter\def\csname PY@tok@nc\endcsname{\let\PY@bf=\textbf\def\PY@tc##1{\textcolor[rgb]{0.00,0.00,1.00}{##1}}}
\expandafter\def\csname PY@tok@nn\endcsname{\let\PY@bf=\textbf\def\PY@tc##1{\textcolor[rgb]{0.00,0.00,1.00}{##1}}}
\expandafter\def\csname PY@tok@ne\endcsname{\let\PY@bf=\textbf\def\PY@tc##1{\textcolor[rgb]{0.82,0.25,0.23}{##1}}}
\expandafter\def\csname PY@tok@nv\endcsname{\def\PY@tc##1{\textcolor[rgb]{0.10,0.09,0.49}{##1}}}
\expandafter\def\csname PY@tok@no\endcsname{\def\PY@tc##1{\textcolor[rgb]{0.53,0.00,0.00}{##1}}}
\expandafter\def\csname PY@tok@nl\endcsname{\def\PY@tc##1{\textcolor[rgb]{0.63,0.63,0.00}{##1}}}
\expandafter\def\csname PY@tok@ni\endcsname{\let\PY@bf=\textbf\def\PY@tc##1{\textcolor[rgb]{0.60,0.60,0.60}{##1}}}
\expandafter\def\csname PY@tok@na\endcsname{\def\PY@tc##1{\textcolor[rgb]{0.49,0.56,0.16}{##1}}}
\expandafter\def\csname PY@tok@nt\endcsname{\let\PY@bf=\textbf\def\PY@tc##1{\textcolor[rgb]{0.00,0.50,0.00}{##1}}}
\expandafter\def\csname PY@tok@nd\endcsname{\def\PY@tc##1{\textcolor[rgb]{0.67,0.13,1.00}{##1}}}
\expandafter\def\csname PY@tok@s\endcsname{\def\PY@tc##1{\textcolor[rgb]{0.73,0.13,0.13}{##1}}}
\expandafter\def\csname PY@tok@sd\endcsname{\let\PY@it=\textit\def\PY@tc##1{\textcolor[rgb]{0.73,0.13,0.13}{##1}}}
\expandafter\def\csname PY@tok@si\endcsname{\let\PY@bf=\textbf\def\PY@tc##1{\textcolor[rgb]{0.73,0.40,0.53}{##1}}}
\expandafter\def\csname PY@tok@se\endcsname{\let\PY@bf=\textbf\def\PY@tc##1{\textcolor[rgb]{0.73,0.40,0.13}{##1}}}
\expandafter\def\csname PY@tok@sr\endcsname{\def\PY@tc##1{\textcolor[rgb]{0.73,0.40,0.53}{##1}}}
\expandafter\def\csname PY@tok@ss\endcsname{\def\PY@tc##1{\textcolor[rgb]{0.10,0.09,0.49}{##1}}}
\expandafter\def\csname PY@tok@sx\endcsname{\def\PY@tc##1{\textcolor[rgb]{0.00,0.50,0.00}{##1}}}
\expandafter\def\csname PY@tok@m\endcsname{\def\PY@tc##1{\textcolor[rgb]{0.40,0.40,0.40}{##1}}}
\expandafter\def\csname PY@tok@gh\endcsname{\let\PY@bf=\textbf\def\PY@tc##1{\textcolor[rgb]{0.00,0.00,0.50}{##1}}}
\expandafter\def\csname PY@tok@gu\endcsname{\let\PY@bf=\textbf\def\PY@tc##1{\textcolor[rgb]{0.50,0.00,0.50}{##1}}}
\expandafter\def\csname PY@tok@gd\endcsname{\def\PY@tc##1{\textcolor[rgb]{0.63,0.00,0.00}{##1}}}
\expandafter\def\csname PY@tok@gi\endcsname{\def\PY@tc##1{\textcolor[rgb]{0.00,0.63,0.00}{##1}}}
\expandafter\def\csname PY@tok@gr\endcsname{\def\PY@tc##1{\textcolor[rgb]{1.00,0.00,0.00}{##1}}}
\expandafter\def\csname PY@tok@ge\endcsname{\let\PY@it=\textit}
\expandafter\def\csname PY@tok@gs\endcsname{\let\PY@bf=\textbf}
\expandafter\def\csname PY@tok@gp\endcsname{\let\PY@bf=\textbf\def\PY@tc##1{\textcolor[rgb]{0.00,0.00,0.50}{##1}}}
\expandafter\def\csname PY@tok@go\endcsname{\def\PY@tc##1{\textcolor[rgb]{0.53,0.53,0.53}{##1}}}
\expandafter\def\csname PY@tok@gt\endcsname{\def\PY@tc##1{\textcolor[rgb]{0.00,0.27,0.87}{##1}}}
\expandafter\def\csname PY@tok@err\endcsname{\def\PY@bc##1{\setlength{\fboxsep}{0pt}\fcolorbox[rgb]{1.00,0.00,0.00}{1,1,1}{\strut ##1}}}
\expandafter\def\csname PY@tok@kc\endcsname{\let\PY@bf=\textbf\def\PY@tc##1{\textcolor[rgb]{0.00,0.50,0.00}{##1}}}
\expandafter\def\csname PY@tok@kd\endcsname{\let\PY@bf=\textbf\def\PY@tc##1{\textcolor[rgb]{0.00,0.50,0.00}{##1}}}
\expandafter\def\csname PY@tok@kn\endcsname{\let\PY@bf=\textbf\def\PY@tc##1{\textcolor[rgb]{0.00,0.50,0.00}{##1}}}
\expandafter\def\csname PY@tok@kr\endcsname{\let\PY@bf=\textbf\def\PY@tc##1{\textcolor[rgb]{0.00,0.50,0.00}{##1}}}
\expandafter\def\csname PY@tok@bp\endcsname{\def\PY@tc##1{\textcolor[rgb]{0.00,0.50,0.00}{##1}}}
\expandafter\def\csname PY@tok@fm\endcsname{\def\PY@tc##1{\textcolor[rgb]{0.00,0.00,1.00}{##1}}}
\expandafter\def\csname PY@tok@vc\endcsname{\def\PY@tc##1{\textcolor[rgb]{0.10,0.09,0.49}{##1}}}
\expandafter\def\csname PY@tok@vg\endcsname{\def\PY@tc##1{\textcolor[rgb]{0.10,0.09,0.49}{##1}}}
\expandafter\def\csname PY@tok@vi\endcsname{\def\PY@tc##1{\textcolor[rgb]{0.10,0.09,0.49}{##1}}}
\expandafter\def\csname PY@tok@vm\endcsname{\def\PY@tc##1{\textcolor[rgb]{0.10,0.09,0.49}{##1}}}
\expandafter\def\csname PY@tok@sa\endcsname{\def\PY@tc##1{\textcolor[rgb]{0.73,0.13,0.13}{##1}}}
\expandafter\def\csname PY@tok@sb\endcsname{\def\PY@tc##1{\textcolor[rgb]{0.73,0.13,0.13}{##1}}}
\expandafter\def\csname PY@tok@sc\endcsname{\def\PY@tc##1{\textcolor[rgb]{0.73,0.13,0.13}{##1}}}
\expandafter\def\csname PY@tok@dl\endcsname{\def\PY@tc##1{\textcolor[rgb]{0.73,0.13,0.13}{##1}}}
\expandafter\def\csname PY@tok@s2\endcsname{\def\PY@tc##1{\textcolor[rgb]{0.73,0.13,0.13}{##1}}}
\expandafter\def\csname PY@tok@sh\endcsname{\def\PY@tc##1{\textcolor[rgb]{0.73,0.13,0.13}{##1}}}
\expandafter\def\csname PY@tok@s1\endcsname{\def\PY@tc##1{\textcolor[rgb]{0.73,0.13,0.13}{##1}}}
\expandafter\def\csname PY@tok@mb\endcsname{\def\PY@tc##1{\textcolor[rgb]{0.40,0.40,0.40}{##1}}}
\expandafter\def\csname PY@tok@mf\endcsname{\def\PY@tc##1{\textcolor[rgb]{0.40,0.40,0.40}{##1}}}
\expandafter\def\csname PY@tok@mh\endcsname{\def\PY@tc##1{\textcolor[rgb]{0.40,0.40,0.40}{##1}}}
\expandafter\def\csname PY@tok@mi\endcsname{\def\PY@tc##1{\textcolor[rgb]{0.40,0.40,0.40}{##1}}}
\expandafter\def\csname PY@tok@il\endcsname{\def\PY@tc##1{\textcolor[rgb]{0.40,0.40,0.40}{##1}}}
\expandafter\def\csname PY@tok@mo\endcsname{\def\PY@tc##1{\textcolor[rgb]{0.40,0.40,0.40}{##1}}}
\expandafter\def\csname PY@tok@ch\endcsname{\let\PY@it=\textit\def\PY@tc##1{\textcolor[rgb]{0.25,0.50,0.50}{##1}}}
\expandafter\def\csname PY@tok@cm\endcsname{\let\PY@it=\textit\def\PY@tc##1{\textcolor[rgb]{0.25,0.50,0.50}{##1}}}
\expandafter\def\csname PY@tok@cpf\endcsname{\let\PY@it=\textit\def\PY@tc##1{\textcolor[rgb]{0.25,0.50,0.50}{##1}}}
\expandafter\def\csname PY@tok@c1\endcsname{\let\PY@it=\textit\def\PY@tc##1{\textcolor[rgb]{0.25,0.50,0.50}{##1}}}
\expandafter\def\csname PY@tok@cs\endcsname{\let\PY@it=\textit\def\PY@tc##1{\textcolor[rgb]{0.25,0.50,0.50}{##1}}}

\def\PYZbs{\char`\\}
\def\PYZus{\char`\_}
\def\PYZob{\char`\{}
\def\PYZcb{\char`\}}
\def\PYZca{\char`\^}
\def\PYZam{\char`\&}
\def\PYZlt{\char`\<}
\def\PYZgt{\char`\>}
\def\PYZsh{\char`\#}
\def\PYZpc{\char`\%}
\def\PYZdl{\char`\$}
\def\PYZhy{\char`\-}
\def\PYZsq{\char`\'}
\def\PYZdq{\char`\"}
\def\PYZti{\char`\~}
% for compatibility with earlier versions
\def\PYZat{@}
\def\PYZlb{[}
\def\PYZrb{]}
\makeatother


    % Exact colors from NB
    \definecolor{incolor}{rgb}{0.0, 0.0, 0.5}
    \definecolor{outcolor}{rgb}{0.545, 0.0, 0.0}



    
    % Prevent overflowing lines due to hard-to-break entities
    \sloppy 
    % Setup hyperref package
    \hypersetup{
      breaklinks=true,  % so long urls are correctly broken across lines
      colorlinks=true,
      urlcolor=urlcolor,
      linkcolor=linkcolor,
      citecolor=citecolor,
      }
    % Slightly bigger margins than the latex defaults
    
    \geometry{verbose,tmargin=1in,bmargin=1in,lmargin=1in,rmargin=1in}
    
    

    \begin{document}
    
    
    \maketitle
    
    

    
    \begin{quote}
Universidade Federal da Bahia
\end{quote}

\begin{quote}
Instituto de Matemática e Estatística
\end{quote}

\begin{quote}
Departamento de Ciência da Computação
\end{quote}

\begin{quote}
MATA60 - Banco de Dados
\end{quote}

\begin{quote}
Docente: Vaninha Vieira
\end{quote}

\begin{quote}
Alunos: Angelmário Santana, Tassia Silva e Litiano Moura
\end{quote}

\begin{quote}
Data: 31 de Maio de 2018
\end{quote}

\begin{quote}
\begin{quote}
\subsection{Tema: Business
Intelligence}\label{tema-business-intelligence}

\begin{quote}
\mbox{}%
\paragraph{RELATÓRIO DA ANÁLISE ESTATÍSTICA DA BASE DE DADOS SOBRE
PEDIDOS DE ALIMENTOS REALIZADOS POR APLICATIVO NO 2° SEMESTRE DE
2016.}\label{relatuxf3rio-da-anuxe1lise-estatuxedstica-da-base-de-dados-sobre-pedidos-de-alimentos-realizados-por-aplicativo-no-2-semestre-de-2016.}
\end{quote}
\end{quote}
\end{quote}

\subsubsection{Introdução}\label{introduuxe7uxe3o}

Nos últimos anos um mercado que vem chamando a atenção de investidores é
o mercado de delivery, que está cada vez mais virando tendência entre os
brasileiros. Segundo a Associação Brasileira de Bares e Restaurantes
(ABRASEL) já em 2015, o mercado brasileiro movimentava 9 bilhões de
reais por ano e em 2017 o faturamento passou dos 10 bilhões.

Muitos estabelecimentos que tinham apenas espaços físicos, têm investido
no serviço de delivery para atrair clientes e aumentar o faturamento. O
SEBRAE reforça a preferência dos consumidores por lugares que ofereçam
entrega em domicílio, e afirma que 12\%, segundo pesquisa realizada, não
possuem nem loja física, e seguem trabalhando apenas com entregas
inclusive por aplicativos.

Muitos consumidores entrevistados afirmaram que pelo comodismo de não
ter que enfrentar o trânsito pelas taxas de serviços e de
estacionamento. Entre outras, faz mais sentido ir no aplicativo e
escolher no cardápio e solicitar a entrega, além da facilidade de ver
vários tipos de comida com apenas alguns cliques, tem atraído vários
clientes mesmo havendo taxa de entrega. Ainda assim vale a pena conferir
as ofertas oferecidas pelo comércio nos aplicativos, explica a pesquisa.

Neste trabalho, foi nos concedido uma base de dados que mostra a
respeito das entregas de alimentos em algumas cidades brasileiras com
intuito de inferir informações a respeito e aplicar Business
Intelligence.

\subsubsection{Analisando os dados por meio
estatístico}\label{analisando-os-dados-por-meio-estatuxedstico}

Após pesquisar sobre o assunto, foi realizada uma análise na base da
equipe. Foi necessário analisar os atributos da base para entender o
comportamento dos dados. A equipe contou com ajuda do
\href{http://jupyter.org/}{Jupyter Notebook} que através de programação
em Python nos ajudou a manipular os dados e desenvolver os gráficos que
serão apresentados a seguir para construção do trabalho.

    \begin{Verbatim}[commandchars=\\\{\}]
{\color{incolor}In [{\color{incolor}489}]:} \PY{k+kn}{import} \PY{n+nn}{numpy} \PY{k}{as} \PY{n+nn}{np}
          \PY{k+kn}{import} \PY{n+nn}{pandas} \PY{k}{as} \PY{n+nn}{pd}
          \PY{k+kn}{import} \PY{n+nn}{statsmodels} \PY{k}{as} \PY{n+nn}{st}
          \PY{k+kn}{import} \PY{n+nn}{matplotlib}\PY{n+nn}{.}\PY{n+nn}{pyplot} \PY{k}{as} \PY{n+nn}{plt}
          \PY{k+kn}{import} \PY{n+nn}{seaborn} \PY{k}{as} \PY{n+nn}{sns}
          \PY{k+kn}{from} \PY{n+nn}{pandas}\PY{n+nn}{.}\PY{n+nn}{plotting} \PY{k}{import} \PY{n}{table}
          \PY{k+kn}{from} \PY{n+nn}{ipywidgets} \PY{k}{import} \PY{n}{widgets}
          \PY{k+kn}{import} \PY{n+nn}{datetime}
\end{Verbatim}


    \begin{Verbatim}[commandchars=\\\{\}]
{\color{incolor}In [{\color{incolor}490}]:} \PY{c+c1}{\PYZsh{}Carregando a base de dados}
          \PY{n}{base} \PY{o}{=} \PY{n}{pd}\PY{o}{.}\PY{n}{read\PYZus{}csv}\PY{p}{(}\PY{l+s+s1}{\PYZsq{}}\PY{l+s+s1}{aplicativo.csv}\PY{l+s+s1}{\PYZsq{}}\PY{p}{)}
\end{Verbatim}


    \begin{itemize}
\tightlist
\item
  O primeiro passo foi observar a base de dados para obter uma visão
  geral dos atributos que existem na base.
\end{itemize}

    \begin{Verbatim}[commandchars=\\\{\}]
{\color{incolor}In [{\color{incolor}491}]:} \PY{c+c1}{\PYZsh{}5 primeiros registros \PYZhy{} Visão inicial dos dados}
          \PY{n}{base}\PY{o}{.}\PY{n}{head}\PY{p}{(}\PY{l+m+mi}{5}\PY{p}{)}
\end{Verbatim}


\begin{Verbatim}[commandchars=\\\{\}]
{\color{outcolor}Out[{\color{outcolor}491}]:}   DATA\_PEDIDO HORA\_PEDIDO DIA\_PEDIDO  VALOR\_PRODUTOS  TAXA\_ENTREGA  \textbackslash{}
          0  2016-07-05       19:51     Sunday            16.0           4.0   
          1  2016-07-05       20:58     Sunday            28.0           4.0   
          2  2016-07-05       21:35     Sunday            13.0           4.0   
          3  2016-07-06       23:22     Monday            11.5           4.0   
          4  2016-07-07       20:08    Tuesday            19.0           4.0   
          
             TOTAL\_PEDIDO FORMA\_PAGAMENTO  AVALIACAO    STATUS  ID\_ESTABELECIMENTO  \textbackslash{}
          0          20.0        Dinheiro        NaN  Entregue                  16   
          1          32.0        Dinheiro        NaN  Recusado                  16   
          2          17.0        Dinheiro        NaN  Recusado                  16   
          3          15.5        Dinheiro        NaN  Entregue                  16   
          4          23.0        Dinheiro        NaN  Recusado                  16   
          
            TIPO\_ESTABELECIMENTO  ID\_USUARIO DDD\_USUARIO DATA\_CADASTRO\_USUARIO  \textbackslash{}
          0           Lanchonete       50720          77            2016-07-05   
          1           Lanchonete       48784          77            2016-06-18   
          2           Lanchonete        7016          77            2015-08-10   
          3           Lanchonete       48536          77            2016-06-15   
          4           Lanchonete       21160          77            2015-12-21   
          
            PRIMEIRO\_PEDIDO BAIRRO\_USUARIO        CIDADE\_USUARIO SO\_DISPOSITIVO  
          0             Sim        Ipanema  Vitória da Conquista        Android  
          1             Não       Candeias  Vitória da Conquista            iOS  
          2             Não        Urbis I  Vitória da Conquista        Android  
          3             Não     Alto Maron  Vitória da Conquista        Android  
          4             Não       Candeias  Vitória da Conquista        Android  
\end{Verbatim}
            
    \begin{itemize}
\tightlist
\item
  Neste gráfico de Boxplot foi possível verificar alguns atributos e a
  sua distribuição no caso do ``valor dos produtos'', ``taxa de
  entrega'' e do ``total de pedidos'', em que vimos a grande variedade
  entres as amostras coletadas, os valores discrepantes e a divisão dos
  dados quanto aos atributos mencionados.
\end{itemize}

    \begin{Verbatim}[commandchars=\\\{\}]
{\color{incolor}In [{\color{incolor}492}]:} \PY{c+c1}{\PYZsh{}Boxplot para verificar os valores discrepantes e a divisão dos dados}
          \PY{c+c1}{\PYZsh{} do atributos mencionados}
          \PY{k}{def} \PY{n+nf}{boxplot\PYZus{}analise}\PY{p}{(}\PY{p}{)}\PY{p}{:}
              \PY{n}{fig} \PY{o}{=} \PY{n}{plt}\PY{o}{.}\PY{n}{figure}\PY{p}{(}\PY{n}{figsize}\PY{o}{=}\PY{p}{(}\PY{l+m+mi}{15}\PY{p}{,}\PY{l+m+mi}{20}\PY{p}{)}\PY{p}{)}
              \PY{n}{ax5} \PY{o}{=} \PY{n}{fig}\PY{o}{.}\PY{n}{add\PYZus{}subplot}\PY{p}{(}\PY{l+m+mi}{221}\PY{p}{)}
              \PY{n}{grafico} \PY{o}{=} \PY{n}{base}\PY{p}{[}\PY{p}{[}\PY{l+s+s1}{\PYZsq{}}\PY{l+s+s1}{VALOR\PYZus{}PRODUTOS}\PY{l+s+s1}{\PYZsq{}}\PY{p}{]}\PY{p}{]}\PY{o}{.}\PY{n}{boxplot}\PY{p}{(}\PY{n}{figsize}\PY{o}{=}\PY{p}{(}\PY{l+m+mi}{20}\PY{p}{,}\PY{l+m+mi}{5}\PY{p}{)}\PY{p}{,} 
                                                     \PY{n}{fontsize}\PY{o}{=} \PY{n}{fontsize}\PY{p}{,}   \PY{n}{ax}\PY{o}{=}\PY{n}{ax5}\PY{p}{)}
              \PY{n}{plt}\PY{o}{.}\PY{n}{title}\PY{p}{(}\PY{l+s+s1}{\PYZsq{}}\PY{l+s+s1}{Valor dos produtos}\PY{l+s+s1}{\PYZsq{}}\PY{p}{)}
          
          
              \PY{n}{ax6} \PY{o}{=} \PY{n}{fig}\PY{o}{.}\PY{n}{add\PYZus{}subplot}\PY{p}{(}\PY{l+m+mi}{222}\PY{p}{)}
              \PY{n}{grafico} \PY{o}{=} \PY{n}{base}\PY{p}{[}\PY{p}{[}\PY{l+s+s1}{\PYZsq{}}\PY{l+s+s1}{TAXA\PYZus{}ENTREGA}\PY{l+s+s1}{\PYZsq{}}\PY{p}{]}\PY{p}{]}\PY{o}{.}\PY{n}{boxplot}\PY{p}{(}\PY{n}{figsize}\PY{o}{=}\PY{p}{(}\PY{l+m+mi}{20}\PY{p}{,}\PY{l+m+mi}{5}\PY{p}{)}\PY{p}{,} 
                                                   \PY{n}{fontsize}\PY{o}{=} \PY{n}{fontsize}\PY{p}{,}   \PY{n}{ax}\PY{o}{=}\PY{n}{ax6}\PY{p}{)}
              \PY{n}{plt}\PY{o}{.}\PY{n}{title}\PY{p}{(}\PY{l+s+s1}{\PYZsq{}}\PY{l+s+s1}{Taxa de entrega}\PY{l+s+s1}{\PYZsq{}}\PY{p}{)}
          
          
              \PY{n}{ax7} \PY{o}{=} \PY{n}{fig}\PY{o}{.}\PY{n}{add\PYZus{}subplot}\PY{p}{(}\PY{l+m+mi}{223}\PY{p}{)}
              \PY{n}{grafico} \PY{o}{=} \PY{n}{base}\PY{p}{[}\PY{p}{[}\PY{l+s+s1}{\PYZsq{}}\PY{l+s+s1}{TOTAL\PYZus{}PEDIDO}\PY{l+s+s1}{\PYZsq{}}\PY{p}{]}\PY{p}{]}\PY{o}{.}\PY{n}{boxplot}\PY{p}{(}\PY{n}{figsize}\PY{o}{=}\PY{p}{(}\PY{l+m+mi}{20}\PY{p}{,}\PY{l+m+mi}{5}\PY{p}{)}\PY{p}{,} 
                                                   \PY{n}{fontsize}\PY{o}{=} \PY{n}{fontsize}\PY{p}{,}   \PY{n}{ax}\PY{o}{=}\PY{n}{ax7}\PY{p}{)}
              \PY{n}{plt}\PY{o}{.}\PY{n}{title}\PY{p}{(}\PY{l+s+s1}{\PYZsq{}}\PY{l+s+s1}{Total de pedidos}\PY{l+s+s1}{\PYZsq{}}\PY{p}{)}
              
          
          \PY{n}{boxplot\PYZus{}analise}\PY{p}{(}\PY{p}{)}
\end{Verbatim}


    \begin{center}
    \adjustimage{max size={0.9\linewidth}{0.9\paperheight}}{output_6_0.png}
    \end{center}
    { \hspace*{\fill} \\}
    
    \begin{Verbatim}[commandchars=\\\{\}]
{\color{incolor}In [{\color{incolor}493}]:} \PY{c+c1}{\PYZsh{}base.describe()}
          \PY{c+c1}{\PYZsh{}min(base[\PYZsq{}DATA\PYZus{}PEDIDO\PYZsq{}])}
          \PY{c+c1}{\PYZsh{}max(base[\PYZsq{}DATA\PYZus{}PEDIDO\PYZsq{}])}
\end{Verbatim}


    \begin{Verbatim}[commandchars=\\\{\}]
{\color{incolor}In [{\color{incolor}494}]:} \PY{c+c1}{\PYZsh{} Formas de pagamentos e a respectiva quantidade,}
          \PY{c+c1}{\PYZsh{} dado que o status do pedido é Recusado.}
          
          \PY{n}{filt} \PY{o}{=} \PY{n}{base}\PY{p}{[}\PY{p}{(}\PY{n}{base}\PY{p}{[}\PY{l+s+s1}{\PYZsq{}}\PY{l+s+s1}{STATUS}\PY{l+s+s1}{\PYZsq{}}\PY{p}{]} \PY{o}{==} \PY{l+s+s2}{\PYZdq{}}\PY{l+s+s2}{Recusado}\PY{l+s+s2}{\PYZdq{}}\PY{p}{)}\PY{p}{]}
          \PY{n}{dff} \PY{o}{=} \PY{n}{pd}\PY{o}{.}\PY{n}{DataFrame}\PY{p}{(}\PY{n}{filt}\PY{o}{.}\PY{n}{groupby}\PY{p}{(}\PY{l+s+s1}{\PYZsq{}}\PY{l+s+s1}{FORMA\PYZus{}PAGAMENTO}\PY{l+s+s1}{\PYZsq{}}\PY{p}{)}\PY{o}{.}\PY{n}{size}\PY{p}{(}\PY{p}{)}\PY{p}{)}
          \PY{n}{dff}\PY{o}{.}\PY{n}{columns} \PY{o}{=} \PY{p}{[}\PY{l+s+s1}{\PYZsq{}}\PY{l+s+s1}{Quantidade}\PY{l+s+s1}{\PYZsq{}}\PY{p}{]}
          \PY{n}{dff}\PY{o}{.}\PY{n}{sort\PYZus{}values}\PY{p}{(}\PY{n}{by} \PY{o}{=}\PY{p}{[}\PY{l+s+s1}{\PYZsq{}}\PY{l+s+s1}{Quantidade}\PY{l+s+s1}{\PYZsq{}}\PY{p}{]}\PY{p}{,} \PY{n}{ascending}\PY{o}{=}\PY{k+kc}{False}\PY{p}{)}
\end{Verbatim}


\begin{Verbatim}[commandchars=\\\{\}]
{\color{outcolor}Out[{\color{outcolor}494}]:}                  Quantidade
          FORMA\_PAGAMENTO            
          Dinheiro               1030
          Cartão                  404
\end{Verbatim}
            
    \begin{Verbatim}[commandchars=\\\{\}]
{\color{incolor}In [{\color{incolor}495}]:} \PY{c+c1}{\PYZsh{}Obter os tipos de estabelecimento e as suas respectivas quantidades}
          \PY{c+c1}{\PYZsh{} da cidade de Vitória da Conquista.}
          
          \PY{n}{filt} \PY{o}{=} \PY{n}{base}\PY{p}{[}\PY{p}{(}\PY{n}{base}\PY{p}{[}\PY{l+s+s1}{\PYZsq{}}\PY{l+s+s1}{CIDADE\PYZus{}USUARIO}\PY{l+s+s1}{\PYZsq{}}\PY{p}{]} \PY{o}{==} \PY{l+s+s2}{\PYZdq{}}\PY{l+s+s2}{Vitória da Conquista}\PY{l+s+s2}{\PYZdq{}}\PY{p}{)}\PY{p}{]}
          \PY{n}{dff} \PY{o}{=} \PY{n}{pd}\PY{o}{.}\PY{n}{DataFrame}\PY{p}{(}\PY{n}{filt}\PY{o}{.}\PY{n}{groupby}\PY{p}{(}\PY{l+s+s1}{\PYZsq{}}\PY{l+s+s1}{TIPO\PYZus{}ESTABELECIMENTO}\PY{l+s+s1}{\PYZsq{}}\PY{p}{)}\PY{o}{.}\PY{n}{size}\PY{p}{(}\PY{p}{)}\PY{p}{)}
          \PY{n}{dff}\PY{o}{.}\PY{n}{columns} \PY{o}{=} \PY{p}{[}\PY{l+s+s1}{\PYZsq{}}\PY{l+s+s1}{Quantidade}\PY{l+s+s1}{\PYZsq{}}\PY{p}{]}
          \PY{n}{dff}\PY{o}{.}\PY{n}{sort\PYZus{}values}\PY{p}{(}\PY{n}{by} \PY{o}{=}\PY{p}{[}\PY{l+s+s1}{\PYZsq{}}\PY{l+s+s1}{Quantidade}\PY{l+s+s1}{\PYZsq{}}\PY{p}{]}\PY{p}{,} \PY{n}{ascending}\PY{o}{=}\PY{k+kc}{False}\PY{p}{)}
\end{Verbatim}


\begin{Verbatim}[commandchars=\\\{\}]
{\color{outcolor}Out[{\color{outcolor}495}]:}                                Quantidade
          TIPO\_ESTABELECIMENTO                     
          Lanchonete                           7065
          Pizzaria                             3733
          Marmitex                             3491
          Comida Natural                       2128
          Restaurante                          1975
          Comida Japonesa                      1152
          Pizzaria/Esfiharia                   1075
          Restaurante/Tapiocaria                573
          Hot-Dog                               450
          Sopas                                 413
          Pizzaria/Lanchonete                   359
          Culinária Oriental                    332
          Doceria                               274
          Picoleteria                           252
          Espetinhos                            213
          Tapiocaria                            115
          Pães e Bolos                          109
          Pizzaria/Esfiharia/Pastelaria         104
          Saladas                                34
          Comida Árabe                           30
          Acarajé                                12
\end{Verbatim}
            
    \begin{itemize}
\tightlist
\item
  A tabela mostrou que referenciava dados sobre pedidos de alimentos
  realizados através de aplicativo para estabelecimentos, que atendiam
  por meio de delivery em que a coleta foi realizada no segundo semestre
  de 2016 de cidades brasileiras com enfoque no estado da Bahia,
  observando a discrepância, já que tivemos apenas um pedido em Sorocaba
  interior de São Paulo
\end{itemize}

    \begin{Verbatim}[commandchars=\\\{\}]
{\color{incolor}In [{\color{incolor}496}]:} \PY{c+c1}{\PYZsh{}Obter as cidades do usuário e suas respectivas quantidades}
          \PY{c+c1}{\PYZsh{} na base de dados}
          
          
          \PY{n}{df} \PY{o}{=} \PY{n}{pd}\PY{o}{.}\PY{n}{DataFrame}\PY{p}{(}\PY{n}{base}\PY{o}{.}\PY{n}{groupby}\PY{p}{(}\PY{l+s+s1}{\PYZsq{}}\PY{l+s+s1}{CIDADE\PYZus{}USUARIO}\PY{l+s+s1}{\PYZsq{}}\PY{p}{)}\PY{o}{.}\PY{n}{size}\PY{p}{(}\PY{p}{)}\PY{p}{)}
          \PY{n}{df}\PY{o}{.}\PY{n}{columns} \PY{o}{=} \PY{p}{[}\PY{l+s+s1}{\PYZsq{}}\PY{l+s+s1}{Quantidade}\PY{l+s+s1}{\PYZsq{}}\PY{p}{]}
          \PY{n}{df}
          \PY{n}{df}\PY{o}{.}\PY{n}{sort\PYZus{}values}\PY{p}{(}\PY{n}{by} \PY{o}{=}\PY{p}{[}\PY{l+s+s1}{\PYZsq{}}\PY{l+s+s1}{Quantidade}\PY{l+s+s1}{\PYZsq{}}\PY{p}{]}\PY{p}{,} \PY{n}{ascending}\PY{o}{=}\PY{k+kc}{False}\PY{p}{)}
\end{Verbatim}


\begin{Verbatim}[commandchars=\\\{\}]
{\color{outcolor}Out[{\color{outcolor}496}]:}                       Quantidade
          CIDADE\_USUARIO                  
          Vitória da Conquista       23889
          Ibicaraí                     103
          Brumado                       71
          Mortugaba                     35
          Sorocaba                       1
\end{Verbatim}
            
    \begin{itemize}
\tightlist
\item
  Verificamos que pela quantidade de pedidos realizados sobre o
  estabelecimento, lanchonete recebeu a maior parte das vendas mesmo
  sendo bastante distribuído os dados referentes aos outros
  estabelecimentos.
\end{itemize}

    \begin{Verbatim}[commandchars=\\\{\}]
{\color{incolor}In [{\color{incolor}497}]:} \PY{n}{df} \PY{o}{=} \PY{n}{pd}\PY{o}{.}\PY{n}{DataFrame}\PY{p}{(}\PY{n}{base}\PY{o}{.}\PY{n}{groupby}\PY{p}{(}\PY{l+s+s1}{\PYZsq{}}\PY{l+s+s1}{TIPO\PYZus{}ESTABELECIMENTO}\PY{l+s+s1}{\PYZsq{}}\PY{p}{)}\PY{o}{.}\PY{n}{size}\PY{p}{(}\PY{p}{)}\PY{p}{)}
          \PY{n}{df}\PY{o}{.}\PY{n}{columns} \PY{o}{=} \PY{p}{[}\PY{l+s+s1}{\PYZsq{}}\PY{l+s+s1}{Quantidade}\PY{l+s+s1}{\PYZsq{}}\PY{p}{]}
          \PY{n}{df}
          \PY{n}{df}\PY{o}{.}\PY{n}{sort\PYZus{}values}\PY{p}{(}\PY{n}{by} \PY{o}{=}\PY{p}{[}\PY{l+s+s1}{\PYZsq{}}\PY{l+s+s1}{Quantidade}\PY{l+s+s1}{\PYZsq{}}\PY{p}{]}\PY{p}{,} \PY{n}{ascending}\PY{o}{=}\PY{k+kc}{False}\PY{p}{)}
\end{Verbatim}


\begin{Verbatim}[commandchars=\\\{\}]
{\color{outcolor}Out[{\color{outcolor}497}]:}                                Quantidade
          TIPO\_ESTABELECIMENTO                     
          Lanchonete                           7187
          Pizzaria                             3763
          Marmitex                             3597
          Comida Natural                       2135
          Restaurante                          2006
          Comida Japonesa                      1167
          Pizzaria/Esfiharia                   1103
          Restaurante/Tapiocaria                584
          Pizzaria/Lanchonete                   455
          Hot-Dog                               451
          Sopas                                 419
          Culinária Oriental                    334
          Doceria                               274
          Picoleteria                           252
          Espetinhos                            213
          Tapiocaria                            129
          Pães e Bolos                          109
          Pizzaria/Esfiharia/Pastelaria         105
          Saladas                                34
          Comida Árabe                           30
          Acarajé                                12
\end{Verbatim}
            
    \begin{itemize}
\tightlist
\item
  Foi verificado também pelo gráfico que a maioria dos pedidos entregues
  foram pagos pela forma de pagamento em dinheiro. Mesmo assim, vimos
  que poucos pedidos foram recusados.
\end{itemize}

    \begin{Verbatim}[commandchars=\\\{\}]
{\color{incolor}In [{\color{incolor}498}]:} \PY{n}{plt}\PY{o}{.}\PY{n}{rcParams}\PY{o}{.}\PY{n}{update}\PY{p}{(}\PY{p}{\PYZob{}}\PY{l+s+s1}{\PYZsq{}}\PY{l+s+s1}{text.color}\PY{l+s+s1}{\PYZsq{}}\PY{p}{:} \PY{l+s+s1}{\PYZsq{}}\PY{l+s+s1}{black}\PY{l+s+s1}{\PYZsq{}}\PY{p}{\PYZcb{}}\PY{p}{)}
          \PY{n}{fontsize} \PY{o}{=} \PY{l+m+mi}{14}
          \PY{n}{font\PYZus{}size}\PY{o}{=}\PY{l+m+mi}{14}
          \PY{n}{d1} \PY{o}{=} \PY{n}{pd}\PY{o}{.}\PY{n}{DataFrame}\PY{p}{(}\PY{n}{base}\PY{o}{.}\PY{n}{groupby}\PY{p}{(}\PY{l+s+s1}{\PYZsq{}}\PY{l+s+s1}{FORMA\PYZus{}PAGAMENTO}\PY{l+s+s1}{\PYZsq{}}\PY{p}{)}\PY{o}{.}\PY{n}{size}\PY{p}{(}\PY{p}{)}\PY{p}{)}
          \PY{n}{d1}\PY{o}{.}\PY{n}{columns} \PY{o}{=} \PY{p}{[}\PY{l+s+s1}{\PYZsq{}}\PY{l+s+s1}{Quantidade}\PY{l+s+s1}{\PYZsq{}}\PY{p}{]}
          \PY{n}{fig} \PY{o}{=} \PY{n}{plt}\PY{o}{.}\PY{n}{figure}\PY{p}{(}\PY{n}{figsize}\PY{o}{=}\PY{p}{(}\PY{l+m+mi}{15}\PY{p}{,}\PY{l+m+mi}{16}\PY{p}{)}\PY{p}{)}
          
          
          \PY{n}{colors} \PY{o}{=} \PY{p}{[}\PY{l+s+s1}{\PYZsq{}}\PY{l+s+s1}{yellowgreen}\PY{l+s+s1}{\PYZsq{}}\PY{p}{,} \PY{l+s+s1}{\PYZsq{}}\PY{l+s+s1}{gold}\PY{l+s+s1}{\PYZsq{}}\PY{p}{]}
          \PY{c+c1}{\PYZsh{} plot chart}
          \PY{n}{ax1} \PY{o}{=} \PY{n}{fig}\PY{o}{.}\PY{n}{add\PYZus{}subplot}\PY{p}{(}\PY{l+m+mi}{221}\PY{p}{)}
          \PY{n}{grafico} \PY{o}{=} \PY{n}{d1}\PY{o}{.}\PY{n}{plot}\PY{p}{(}\PY{n}{kind} \PY{o}{=} \PY{l+s+s1}{\PYZsq{}}\PY{l+s+s1}{pie}\PY{l+s+s1}{\PYZsq{}}\PY{p}{,} \PY{n}{autopct}\PY{o}{=}\PY{l+s+s1}{\PYZsq{}}\PY{l+s+si}{\PYZpc{}1.1f}\PY{l+s+si}{\PYZpc{}\PYZpc{}}\PY{l+s+s1}{\PYZsq{}}\PY{p}{,} 
                            \PY{n}{startangle}\PY{o}{=}\PY{l+m+mi}{90}\PY{p}{,} \PY{n}{shadow}\PY{o}{=}\PY{k+kc}{False}\PY{p}{,}
                           \PY{n}{figsize}\PY{o}{=}\PY{p}{(}\PY{l+m+mi}{5}\PY{p}{,}\PY{l+m+mi}{5}\PY{p}{)}\PY{p}{,} \PY{n}{fontsize}\PY{o}{=} \PY{n}{fontsize}\PY{p}{,} 
                            \PY{n}{colors} \PY{o}{=} \PY{n}{colors}\PY{p}{,} 
                            \PY{n}{legend} \PY{o}{=} \PY{k+kc}{True}\PY{p}{,}   \PY{n}{ax}\PY{o}{=}\PY{n}{ax1}\PY{p}{,} \PY{n}{subplots}\PY{o}{=}\PY{k+kc}{True}\PY{p}{)}
          \PY{n}{plt}\PY{o}{.}\PY{n}{title}\PY{p}{(}\PY{l+s+s1}{\PYZsq{}}\PY{l+s+s1}{Formas de pagamento}\PY{l+s+s1}{\PYZsq{}}\PY{p}{)}
          \PY{n}{plt}\PY{o}{.}\PY{n}{ylabel}\PY{p}{(}\PY{l+s+s1}{\PYZsq{}}\PY{l+s+s1}{\PYZsq{}}\PY{p}{)}
          \PY{n}{plt}\PY{o}{.}\PY{n}{legend}\PY{p}{(}\PY{n}{loc}\PY{o}{=}\PY{l+m+mi}{2}\PY{p}{,} \PY{n}{prop}\PY{o}{=}\PY{p}{\PYZob{}}\PY{l+s+s1}{\PYZsq{}}\PY{l+s+s1}{size}\PY{l+s+s1}{\PYZsq{}}\PY{p}{:} \PY{n}{fontsize}\PY{p}{\PYZcb{}}\PY{p}{)}
          
          \PY{n}{d2} \PY{o}{=} \PY{n}{pd}\PY{o}{.}\PY{n}{DataFrame}\PY{p}{(}\PY{n}{base}\PY{o}{.}\PY{n}{groupby}\PY{p}{(}\PY{l+s+s1}{\PYZsq{}}\PY{l+s+s1}{STATUS}\PY{l+s+s1}{\PYZsq{}}\PY{p}{)}\PY{o}{.}\PY{n}{size}\PY{p}{(}\PY{p}{)}\PY{p}{)}
          \PY{n}{d2}\PY{o}{.}\PY{n}{columns} \PY{o}{=} \PY{p}{[}\PY{l+s+s1}{\PYZsq{}}\PY{l+s+s1}{Quantidade}\PY{l+s+s1}{\PYZsq{}}\PY{p}{]}
          
          \PY{n}{ax2} \PY{o}{=} \PY{n}{fig}\PY{o}{.}\PY{n}{add\PYZus{}subplot}\PY{p}{(}\PY{l+m+mi}{222}\PY{p}{)}
          \PY{n}{grafico} \PY{o}{=} \PY{n}{d2}\PY{o}{.}\PY{n}{plot}\PY{p}{(}\PY{n}{kind} \PY{o}{=} \PY{l+s+s1}{\PYZsq{}}\PY{l+s+s1}{pie}\PY{l+s+s1}{\PYZsq{}}\PY{p}{,}\PY{n}{autopct}\PY{o}{=}\PY{l+s+s1}{\PYZsq{}}\PY{l+s+si}{\PYZpc{}1.1f}\PY{l+s+si}{\PYZpc{}\PYZpc{}}\PY{l+s+s1}{\PYZsq{}}\PY{p}{,} 
                            \PY{n}{startangle}\PY{o}{=}\PY{l+m+mi}{90}\PY{p}{,} \PY{n}{shadow}\PY{o}{=}\PY{k+kc}{False}\PY{p}{,} 
                             \PY{n}{figsize}\PY{o}{=}\PY{p}{(}\PY{l+m+mi}{5}\PY{p}{,}\PY{l+m+mi}{5}\PY{p}{)}\PY{p}{,}\PY{n}{fontsize}\PY{o}{=}\PY{n}{fontsize}\PY{p}{,} 
                            \PY{n}{colors} \PY{o}{=} \PY{n}{colors}\PY{p}{,} 
                            \PY{n}{legend} \PY{o}{=} \PY{k+kc}{True}\PY{p}{,}   \PY{n}{ax}\PY{o}{=}\PY{n}{ax2}\PY{p}{,} \PY{n}{subplots}\PY{o}{=}\PY{k+kc}{True}\PY{p}{)}
          \PY{n}{plt}\PY{o}{.}\PY{n}{title}\PY{p}{(}\PY{l+s+s1}{\PYZsq{}}\PY{l+s+s1}{Status do pedido}\PY{l+s+s1}{\PYZsq{}}\PY{p}{)}
          \PY{n}{plt}\PY{o}{.}\PY{n}{ylabel}\PY{p}{(}\PY{l+s+s1}{\PYZsq{}}\PY{l+s+s1}{\PYZsq{}}\PY{p}{)}
          \PY{n}{plt}\PY{o}{.}\PY{n}{legend}\PY{p}{(}\PY{n}{loc}\PY{o}{=}\PY{l+m+mi}{2}\PY{p}{,} \PY{n}{prop}\PY{o}{=}\PY{p}{\PYZob{}}\PY{l+s+s1}{\PYZsq{}}\PY{l+s+s1}{size}\PY{l+s+s1}{\PYZsq{}}\PY{p}{:} \PY{n}{fontsize}\PY{p}{\PYZcb{}}\PY{p}{)}
          
          \PY{c+c1}{\PYZsh{}ax3 = fig.add\PYZus{}subplot(223)}
          
          \PY{c+c1}{\PYZsh{}ax3.axis(\PYZsq{}off\PYZsq{})}
          \PY{n}{mpl\PYZus{}table} \PY{o}{=} \PY{n}{table}\PY{p}{(}\PY{n}{ax2}\PY{p}{,} \PY{n}{d2}\PY{p}{,} \PY{n}{loc}\PY{o}{=}\PY{l+s+s1}{\PYZsq{}}\PY{l+s+s1}{bottom}\PY{l+s+s1}{\PYZsq{}}\PY{p}{,} 
                            \PY{n}{rowLoc}\PY{o}{=}\PY{l+s+s1}{\PYZsq{}}\PY{l+s+s1}{left}\PY{l+s+s1}{\PYZsq{}}\PY{p}{,} 
                            \PY{n}{colLoc} \PY{o}{=} \PY{l+s+s1}{\PYZsq{}}\PY{l+s+s1}{center}\PY{l+s+s1}{\PYZsq{}}\PY{p}{)}
          \PY{n}{mpl\PYZus{}table}\PY{o}{.}\PY{n}{auto\PYZus{}set\PYZus{}font\PYZus{}size}\PY{p}{(}\PY{k+kc}{False}\PY{p}{)}
          \PY{n}{mpl\PYZus{}table}\PY{o}{.}\PY{n}{set\PYZus{}fontsize}\PY{p}{(}\PY{n}{font\PYZus{}size}\PY{p}{)}
          \PY{n}{mpl\PYZus{}table}\PY{o}{.}\PY{n}{scale}\PY{p}{(}\PY{l+m+mf}{0.4}\PY{p}{,}\PY{l+m+mf}{2.8}\PY{p}{)}
          
          \PY{c+c1}{\PYZsh{}ax4 = fig.add\PYZus{}subplot(224)}
          
          \PY{c+c1}{\PYZsh{}ax4.axis(\PYZsq{}off\PYZsq{})}
          \PY{n}{mpl\PYZus{}table} \PY{o}{=} \PY{n}{table}\PY{p}{(}\PY{n}{ax1}\PY{p}{,} \PY{n}{d1}\PY{p}{,} \PY{n}{loc}\PY{o}{=}\PY{l+s+s1}{\PYZsq{}}\PY{l+s+s1}{bottom}\PY{l+s+s1}{\PYZsq{}}\PY{p}{,} 
                            \PY{n}{rowLoc}\PY{o}{=}\PY{l+s+s1}{\PYZsq{}}\PY{l+s+s1}{left}\PY{l+s+s1}{\PYZsq{}}\PY{p}{,} 
                            \PY{n}{colLoc} \PY{o}{=} \PY{l+s+s1}{\PYZsq{}}\PY{l+s+s1}{center}\PY{l+s+s1}{\PYZsq{}}\PY{p}{)}
          \PY{n}{mpl\PYZus{}table}\PY{o}{.}\PY{n}{auto\PYZus{}set\PYZus{}font\PYZus{}size}\PY{p}{(}\PY{k+kc}{False}\PY{p}{)}
          \PY{n}{mpl\PYZus{}table}\PY{o}{.}\PY{n}{set\PYZus{}fontsize}\PY{p}{(}\PY{n}{font\PYZus{}size}\PY{p}{)}
          \PY{n}{mpl\PYZus{}table}\PY{o}{.}\PY{n}{scale}\PY{p}{(}\PY{l+m+mf}{0.4}\PY{p}{,}\PY{l+m+mf}{2.8}\PY{p}{)}
\end{Verbatim}


    \begin{center}
    \adjustimage{max size={0.9\linewidth}{0.9\paperheight}}{output_15_0.png}
    \end{center}
    { \hspace*{\fill} \\}
    
    \begin{itemize}
\tightlist
\item
  No gráficos de histograma, é possível através de atributos numéricos
  observar a distribuição na base de dados
\end{itemize}

    \begin{Verbatim}[commandchars=\\\{\}]
{\color{incolor}In [{\color{incolor}499}]:} \PY{c+c1}{\PYZsh{}Histograma dos atributos numéricos para }
          \PY{c+c1}{\PYZsh{}observar a distribuição na base de dados.}
          \PY{k}{def} \PY{n+nf}{histograma\PYZus{}analise}\PY{p}{(}\PY{p}{)}\PY{p}{:}
              \PY{n}{fontsize} \PY{o}{=} \PY{l+m+mi}{15}
              \PY{n}{font\PYZus{}size} \PY{o}{=} \PY{l+m+mi}{15}
          
              \PY{n}{fig} \PY{o}{=} \PY{n}{plt}\PY{o}{.}\PY{n}{figure}\PY{p}{(}\PY{n}{figsize}\PY{o}{=}\PY{p}{(}\PY{l+m+mi}{12}\PY{p}{,}\PY{l+m+mi}{10}\PY{p}{)}\PY{p}{)}
              \PY{n}{ax5} \PY{o}{=} \PY{n}{fig}\PY{o}{.}\PY{n}{add\PYZus{}subplot}\PY{p}{(}\PY{l+m+mi}{221}\PY{p}{)}
              \PY{n}{base}\PY{o}{.}\PY{n}{hist}\PY{p}{(}\PY{n}{column}\PY{o}{=}\PY{l+s+s1}{\PYZsq{}}\PY{l+s+s1}{VALOR\PYZus{}PRODUTOS}\PY{l+s+s1}{\PYZsq{}}\PY{p}{,} \PY{n}{bins}\PY{o}{=}\PY{l+m+mi}{100}\PY{p}{,} \PY{n}{figsize}\PY{o}{=}\PY{p}{(}\PY{l+m+mi}{20}\PY{p}{,}\PY{l+m+mi}{5}\PY{p}{)}\PY{p}{,} \PY{n}{ax}\PY{o}{=}\PY{n}{ax5}\PY{p}{)}
              \PY{n}{plt}\PY{o}{.}\PY{n}{title}\PY{p}{(}\PY{l+s+s1}{\PYZsq{}}\PY{l+s+s1}{Valor dos produtos}\PY{l+s+s1}{\PYZsq{}}\PY{p}{)}
          
          
              \PY{n}{ax6} \PY{o}{=} \PY{n}{fig}\PY{o}{.}\PY{n}{add\PYZus{}subplot}\PY{p}{(}\PY{l+m+mi}{222}\PY{p}{)}
              \PY{n}{base}\PY{o}{.}\PY{n}{hist}\PY{p}{(}\PY{n}{column}\PY{o}{=}\PY{l+s+s1}{\PYZsq{}}\PY{l+s+s1}{TAXA\PYZus{}ENTREGA}\PY{l+s+s1}{\PYZsq{}}\PY{p}{,} \PY{n}{bins}\PY{o}{=}\PY{l+m+mi}{100}\PY{p}{,} \PY{n}{figsize}\PY{o}{=}\PY{p}{(}\PY{l+m+mi}{20}\PY{p}{,}\PY{l+m+mi}{5}\PY{p}{)}\PY{p}{,}  \PY{n}{ax}\PY{o}{=}\PY{n}{ax6}\PY{p}{)}
              \PY{n}{plt}\PY{o}{.}\PY{n}{title}\PY{p}{(}\PY{l+s+s1}{\PYZsq{}}\PY{l+s+s1}{Taxa de entrega}\PY{l+s+s1}{\PYZsq{}}\PY{p}{)}
          
          
              \PY{n}{ax7} \PY{o}{=} \PY{n}{fig}\PY{o}{.}\PY{n}{add\PYZus{}subplot}\PY{p}{(}\PY{l+m+mi}{223}\PY{p}{)}
              \PY{n}{base}\PY{o}{.}\PY{n}{hist}\PY{p}{(}\PY{n}{column}\PY{o}{=}\PY{l+s+s1}{\PYZsq{}}\PY{l+s+s1}{TOTAL\PYZus{}PEDIDO}\PY{l+s+s1}{\PYZsq{}}\PY{p}{,} \PY{n}{bins}\PY{o}{=}\PY{l+m+mi}{100}\PY{p}{,} \PY{n}{figsize}\PY{o}{=}\PY{p}{(}\PY{l+m+mi}{20}\PY{p}{,}\PY{l+m+mi}{5}\PY{p}{)}\PY{p}{,}   \PY{n}{ax}\PY{o}{=}\PY{n}{ax7}\PY{p}{)}
              \PY{n}{plt}\PY{o}{.}\PY{n}{title}\PY{p}{(}\PY{l+s+s1}{\PYZsq{}}\PY{l+s+s1}{Total de pedidos}\PY{l+s+s1}{\PYZsq{}}\PY{p}{)}
          
              \PY{n}{dff} \PY{o}{=} \PY{n}{pd}\PY{o}{.}\PY{n}{DataFrame}\PY{p}{(}\PY{n}{base}\PY{o}{.}\PY{n}{groupby}\PY{p}{(}\PY{l+s+s1}{\PYZsq{}}\PY{l+s+s1}{ID\PYZus{}USUARIO}\PY{l+s+s1}{\PYZsq{}}\PY{p}{)}\PY{o}{.}\PY{n}{size}\PY{p}{(}\PY{p}{)}\PY{p}{)}
              \PY{n}{dff}\PY{o}{.}\PY{n}{columns} \PY{o}{=} \PY{p}{[}\PY{l+s+s1}{\PYZsq{}}\PY{l+s+s1}{Quantidade}\PY{l+s+s1}{\PYZsq{}}\PY{p}{]}
              \PY{n}{dff}\PY{o}{.}\PY{n}{sort\PYZus{}values}\PY{p}{(}\PY{n}{by} \PY{o}{=}\PY{p}{[}\PY{l+s+s1}{\PYZsq{}}\PY{l+s+s1}{Quantidade}\PY{l+s+s1}{\PYZsq{}}\PY{p}{]}\PY{p}{,} \PY{n}{ascending}\PY{o}{=}\PY{k+kc}{False}\PY{p}{)}
              \PY{n}{ax8} \PY{o}{=} \PY{n}{fig}\PY{o}{.}\PY{n}{add\PYZus{}subplot}\PY{p}{(}\PY{l+m+mi}{224}\PY{p}{)}
          
              \PY{n}{dff}\PY{o}{.}\PY{n}{hist}\PY{p}{(}\PY{n}{column}\PY{o}{=}\PY{l+s+s1}{\PYZsq{}}\PY{l+s+s1}{Quantidade}\PY{l+s+s1}{\PYZsq{}}\PY{p}{,} \PY{n}{bins}\PY{o}{=}\PY{l+m+mi}{100}\PY{p}{,} \PY{n}{figsize}\PY{o}{=}\PY{p}{(}\PY{l+m+mi}{20}\PY{p}{,}\PY{l+m+mi}{5}\PY{p}{)}\PY{p}{,}   \PY{n}{ax}\PY{o}{=}\PY{n}{ax8}\PY{p}{)}
              \PY{n}{plt}\PY{o}{.}\PY{n}{title}\PY{p}{(}\PY{l+s+s1}{\PYZsq{}}\PY{l+s+s1}{Quantidade de pedidos por usuário}\PY{l+s+s1}{\PYZsq{}}\PY{p}{)}
          
          
          \PY{n}{histograma\PYZus{}analise}\PY{p}{(}\PY{p}{)}
\end{Verbatim}


    \begin{center}
    \adjustimage{max size={0.9\linewidth}{0.9\paperheight}}{output_17_0.png}
    \end{center}
    { \hspace*{\fill} \\}
    
    \subsection{Pré-processamento}\label{pruxe9-processamento}

Para executar a etapa de pré-processamento, além da etapa de análise dos
dados por meio estatístico, observamos os dados para verificar a
existência dos seguintes itens:

\begin{enumerate}
\def\labelenumi{\arabic{enumi}.}
\tightlist
\item
  Instâncias duplicadas;
\item
  Valores ausentes;
\item
  Valores inconsistentes;
\item
  Outliers.
\end{enumerate}

Após essa verificação, executamos as etapas de pré-processamento.

\subsubsection{1. Instâncias duplicadas}\label{instuxe2ncias-duplicadas}

\begin{itemize}
\tightlist
\item
  Verificamos a quantidade de instâncias e atributos que a base possui:
\end{itemize}

    \begin{Verbatim}[commandchars=\\\{\}]
{\color{incolor}In [{\color{incolor}500}]:} \PY{c+c1}{\PYZsh{}Verificar quantas instâncias (linhas) e quantos atributos (colunas) tem a base}
          \PY{n}{base}\PY{o}{.}\PY{n}{shape}
\end{Verbatim}


\begin{Verbatim}[commandchars=\\\{\}]
{\color{outcolor}Out[{\color{outcolor}500}]:} (24359, 18)
\end{Verbatim}
            
    \begin{itemize}
\tightlist
\item
  Verificamos quantas instâncias duplicadas existem na base:
\end{itemize}

    \begin{Verbatim}[commandchars=\\\{\}]
{\color{incolor}In [{\color{incolor}501}]:} \PY{c+c1}{\PYZsh{}Verificar quantas instâncias (linhas) duplicadas}
          \PY{n}{base}\PY{o}{.}\PY{n}{duplicated}\PY{p}{(}\PY{p}{)}\PY{o}{.}\PY{n}{sum}\PY{p}{(}\PY{p}{)}
\end{Verbatim}


\begin{Verbatim}[commandchars=\\\{\}]
{\color{outcolor}Out[{\color{outcolor}501}]:} 7
\end{Verbatim}
            
    \begin{itemize}
\tightlist
\item
  Verificamos quais instâncias são duplicadas:
\end{itemize}

    \begin{Verbatim}[commandchars=\\\{\}]
{\color{incolor}In [{\color{incolor}502}]:} \PY{c+c1}{\PYZsh{}Verificar quais são as instâncias duplicadas}
          \PY{n}{base}\PY{o}{.}\PY{n}{loc}\PY{p}{[}\PY{n}{base}\PY{o}{.}\PY{n}{duplicated}\PY{p}{(}\PY{n}{keep} \PY{o}{=} \PY{k+kc}{False}\PY{p}{)}\PY{p}{,} \PY{p}{:}\PY{p}{]}
\end{Verbatim}


\begin{Verbatim}[commandchars=\\\{\}]
{\color{outcolor}Out[{\color{outcolor}502}]:}       DATA\_PEDIDO HORA\_PEDIDO DIA\_PEDIDO  VALOR\_PRODUTOS  TAXA\_ENTREGA  \textbackslash{}
          1679   2016-09-17       23:25   Thursday            10.0           2.0   
          1680   2016-09-17       23:25   Thursday            10.0           2.0   
          11793  2016-09-24       21:27   Thursday             0.0           3.0   
          11794  2016-09-24       21:27   Thursday             0.0           3.0   
          11795  2016-09-24       21:27   Thursday             0.0           3.0   
          11796  2016-09-24       21:27   Thursday             0.0           3.0   
          15616  2016-09-19       10:50   Saturday            12.0           2.0   
          15617  2016-09-19       10:50   Saturday            12.0           2.0   
          23685  2016-09-27       16:36     Sunday            16.0           4.0   
          23686  2016-09-27       16:36     Sunday            16.0           4.0   
          23688  2016-09-27       16:36     Sunday            16.0           4.0   
          
                 TOTAL\_PEDIDO FORMA\_PAGAMENTO  AVALIACAO    STATUS  ID\_ESTABELECIMENTO  \textbackslash{}
          1679           12.0        Dinheiro        NaN  Entregue                 168   
          1680           12.0        Dinheiro        NaN  Entregue                 168   
          11793           0.0        Dinheiro        NaN  Recusado                 936   
          11794           0.0        Dinheiro        NaN  Recusado                 936   
          11795           0.0        Dinheiro        NaN  Recusado                 936   
          11796           0.0        Dinheiro        NaN  Recusado                 936   
          15616          14.0        Dinheiro        NaN  Entregue                1400   
          15617          14.0        Dinheiro        NaN  Entregue                1400   
          23685          20.0          Cartão        NaN  Recusado                 640   
          23686          20.0          Cartão        NaN  Recusado                 640   
          23688          20.0          Cartão        NaN  Recusado                 640   
          
                TIPO\_ESTABELECIMENTO  ID\_USUARIO DDD\_USUARIO DATA\_CADASTRO\_USUARIO  \textbackslash{}
          1679            Lanchonete       52376          77            2016-07-18   
          1680            Lanchonete       52376          77            2016-07-18   
          11793           Lanchonete        7592          77            2015-08-19   
          11794           Lanchonete        7592          77            2015-08-19   
          11795           Lanchonete        7592          77            2015-08-19   
          11796           Lanchonete        7592          77            2015-08-19   
          15616             Marmitex       38936          73            2016-04-15   
          15617             Marmitex       38936          73            2016-04-15   
          23685       Comida Natural        7000          77            2015-08-09   
          23686       Comida Natural        7000          77            2015-08-09   
          23688       Comida Natural        7000          77            2015-08-09   
          
                PRIMEIRO\_PEDIDO BAIRRO\_USUARIO        CIDADE\_USUARIO SO\_DISPOSITIVO  
          1679              Não         Sumaré  Vitória da Conquista        Android  
          1680              Não         Sumaré  Vitória da Conquista        Android  
          11793             Não       Candeias  Vitória da Conquista             PC  
          11794             Não       Candeias  Vitória da Conquista             PC  
          11795             Não       Candeias  Vitória da Conquista             PC  
          11796             Não       Candeias  Vitória da Conquista             PC  
          15616             Não       Candeias  Vitória da Conquista        Android  
          15617             Não       Candeias  Vitória da Conquista        Android  
          23685             Não   Universidade  Vitória da Conquista             PC  
          23686             Não   Universidade  Vitória da Conquista             PC  
          23688             Não   Universidade  Vitória da Conquista             PC  
\end{Verbatim}
            
    \begin{itemize}
\tightlist
\item
  Por se tratar de poucas instâncias duplicadas, foi decidido remover as
  mesmas:
\end{itemize}

    \begin{Verbatim}[commandchars=\\\{\}]
{\color{incolor}In [{\color{incolor}503}]:} \PY{n}{novaBase} \PY{o}{=} \PY{n}{base}\PY{o}{.}\PY{n}{drop\PYZus{}duplicates}\PY{p}{(}\PY{n}{keep}\PY{o}{=}\PY{l+s+s2}{\PYZdq{}}\PY{l+s+s2}{first}\PY{l+s+s2}{\PYZdq{}}\PY{p}{)}
\end{Verbatim}


    \begin{itemize}
\tightlist
\item
  Quantidade de instâncias após remoção:
\end{itemize}

    \begin{Verbatim}[commandchars=\\\{\}]
{\color{incolor}In [{\color{incolor}504}]:} \PY{n}{novaBase}\PY{o}{.}\PY{n}{shape}
\end{Verbatim}


\begin{Verbatim}[commandchars=\\\{\}]
{\color{outcolor}Out[{\color{outcolor}504}]:} (24352, 18)
\end{Verbatim}
            
    \subsubsection{2. Valores ausentes}\label{valores-ausentes}

\begin{itemize}
\tightlist
\item
  Verificamos quantos valores ausentes e de quais atributos são:
\end{itemize}

    \begin{Verbatim}[commandchars=\\\{\}]
{\color{incolor}In [{\color{incolor}505}]:} \PY{c+c1}{\PYZsh{}Gráfico que exibe a quantidade valores ausentes }
          \PY{c+c1}{\PYZsh{}para cada atributo da base}
          
          \PY{n}{fontsize} \PY{o}{=} \PY{l+m+mi}{15}
          \PY{n}{font\PYZus{}size} \PY{o}{=} \PY{l+m+mi}{15}
          \PY{n}{df} \PY{o}{=} \PY{n}{pd}\PY{o}{.}\PY{n}{DataFrame}\PY{p}{(}\PY{n}{novaBase}\PY{o}{.}\PY{n}{isnull}\PY{p}{(}\PY{p}{)}\PY{o}{.}\PY{n}{sum}\PY{p}{(}\PY{p}{)}\PY{p}{)}
          
          \PY{n}{df}\PY{o}{.}\PY{n}{columns} \PY{o}{=} \PY{p}{[}\PY{l+s+s1}{\PYZsq{}}\PY{l+s+s1}{Quantidade}\PY{l+s+s1}{\PYZsq{}}\PY{p}{]}
          \PY{n}{ax8} \PY{o}{=}  \PY{n}{plt}\PY{o}{.}\PY{n}{subplot2grid}\PY{p}{(}\PY{p}{(}\PY{l+m+mi}{2}\PY{p}{,} \PY{l+m+mi}{2}\PY{p}{)}\PY{p}{,} \PY{p}{(}\PY{l+m+mi}{0}\PY{p}{,} \PY{l+m+mi}{0}\PY{p}{)}\PY{p}{,} \PY{n}{colspan}\PY{o}{=}\PY{l+m+mi}{2}\PY{p}{)}
          \PY{n}{grafico} \PY{o}{=} \PY{n}{df}\PY{o}{.}\PY{n}{plot}\PY{o}{.}\PY{n}{barh}\PY{p}{(}\PY{n}{figsize}\PY{o}{=}\PY{p}{(}\PY{l+m+mi}{10}\PY{p}{,}\PY{l+m+mi}{16}\PY{p}{)}\PY{p}{,} \PY{n}{fontsize}\PY{o}{=} \PY{n}{fontsize}\PY{p}{,}   
                                 \PY{n}{ax}\PY{o}{=}\PY{n}{ax8}\PY{p}{,} 
                                 \PY{n}{facecolor}\PY{o}{=}\PY{l+s+s1}{\PYZsq{}}\PY{l+s+s1}{blue}\PY{l+s+s1}{\PYZsq{}}\PY{p}{,} 
                                 \PY{n}{edgecolor}\PY{o}{=}\PY{l+s+s1}{\PYZsq{}}\PY{l+s+s1}{white}\PY{l+s+s1}{\PYZsq{}}\PY{p}{,} 
                                 \PY{n}{legend} \PY{o}{=} \PY{k+kc}{False}\PY{p}{)}
          \PY{n}{plt}\PY{o}{.}\PY{n}{title}\PY{p}{(}\PY{l+s+s1}{\PYZsq{}}\PY{l+s+s1}{Valores ausentes}\PY{l+s+s1}{\PYZsq{}}\PY{p}{)}
          
          \PY{n}{df} \PY{o}{=} \PY{n}{df}\PY{p}{[}\PY{p}{(}\PY{n}{df}\PY{p}{[}\PY{l+s+s1}{\PYZsq{}}\PY{l+s+s1}{Quantidade}\PY{l+s+s1}{\PYZsq{}}\PY{p}{]} \PY{o}{\PYZgt{}} \PY{l+m+mi}{0}\PY{p}{)}\PY{p}{]}
          \PY{n}{df}
          \PY{n}{mpl\PYZus{}table} \PY{o}{=} \PY{n}{table}\PY{p}{(}\PY{n}{ax8}\PY{p}{,} \PY{n}{df}\PY{p}{,} \PY{n}{loc}\PY{o}{=}\PY{l+s+s1}{\PYZsq{}}\PY{l+s+s1}{best}\PY{l+s+s1}{\PYZsq{}}\PY{p}{,} \PY{n}{rowLoc}\PY{o}{=}\PY{l+s+s1}{\PYZsq{}}\PY{l+s+s1}{left}\PY{l+s+s1}{\PYZsq{}}\PY{p}{,} \PY{n}{colLoc} \PY{o}{=} \PY{l+s+s1}{\PYZsq{}}\PY{l+s+s1}{center}\PY{l+s+s1}{\PYZsq{}}\PY{p}{)}
          \PY{n}{mpl\PYZus{}table}\PY{o}{.}\PY{n}{auto\PYZus{}set\PYZus{}font\PYZus{}size}\PY{p}{(}\PY{k+kc}{False}\PY{p}{)}
          \PY{n}{mpl\PYZus{}table}\PY{o}{.}\PY{n}{set\PYZus{}fontsize}\PY{p}{(}\PY{n}{font\PYZus{}size}\PY{p}{)}
          \PY{n}{mpl\PYZus{}table}\PY{o}{.}\PY{n}{scale}\PY{p}{(}\PY{l+m+mf}{0.2}\PY{p}{,}\PY{l+m+mf}{2.8}\PY{p}{)}
\end{Verbatim}


    \begin{center}
    \adjustimage{max size={0.9\linewidth}{0.9\paperheight}}{output_29_0.png}
    \end{center}
    { \hspace*{\fill} \\}
    
    \begin{itemize}
\tightlist
\item
  Verificamos se as 260 instâncias de valor ausente para o atributo
  \textbf{CIDADE\_USUARIO} são as mesmas para o
  \textbf{BAIRRO\_USUARIO}, e vice-versa:
\end{itemize}

    \begin{Verbatim}[commandchars=\\\{\}]
{\color{incolor}In [{\color{incolor}506}]:} \PY{c+c1}{\PYZsh{}(linha, coluna)}
          \PY{n}{novaBase}\PY{p}{[}\PY{n}{novaBase}\PY{p}{[}\PY{l+s+s1}{\PYZsq{}}\PY{l+s+s1}{CIDADE\PYZus{}USUARIO}\PY{l+s+s1}{\PYZsq{}}\PY{p}{]}\PY{o}{.}\PY{n}{isnull}\PY{p}{(}\PY{p}{)} \PY{o}{\PYZam{}} \PY{n}{novaBase}\PY{p}{[}\PY{l+s+s1}{\PYZsq{}}\PY{l+s+s1}{BAIRRO\PYZus{}USUARIO}\PY{l+s+s1}{\PYZsq{}}\PY{p}{]}\PY{o}{.}\PY{n}{isnull}\PY{p}{(}\PY{p}{)}\PY{p}{]}\PY{o}{.}\PY{n}{shape}
\end{Verbatim}


\begin{Verbatim}[commandchars=\\\{\}]
{\color{outcolor}Out[{\color{outcolor}506}]:} (260, 18)
\end{Verbatim}
            
    \begin{itemize}
\tightlist
\item
  Após observar a quantidade de valores ausentes e confirmar que o
  atributo \textbf{CIDADE\_USUARIO} e \textbf{BAIRRO\_USUARIO} estão
  correlacionados, foi decidido remover as instâncias que não contém
  valor para o atributo \textbf{CIDADE\_USUARIO} e
  \textbf{BAIRRO\_USUARIO}, por causa da sua baixa quantidade em relação
  a de instâncias da base:
\end{itemize}

    \begin{Verbatim}[commandchars=\\\{\}]
{\color{incolor}In [{\color{incolor}507}]:} \PY{n}{novaBase} \PY{o}{=} \PY{n}{novaBase}\PY{o}{.}\PY{n}{dropna}\PY{p}{(}\PY{n}{subset}\PY{o}{=}\PY{p}{[}\PY{l+s+s1}{\PYZsq{}}\PY{l+s+s1}{CIDADE\PYZus{}USUARIO}\PY{l+s+s1}{\PYZsq{}}\PY{p}{,} \PY{l+s+s1}{\PYZsq{}}\PY{l+s+s1}{BAIRRO\PYZus{}USUARIO}\PY{l+s+s1}{\PYZsq{}}\PY{p}{]}\PY{p}{,} \PY{n}{how}\PY{o}{=}\PY{l+s+s1}{\PYZsq{}}\PY{l+s+s1}{all}\PY{l+s+s1}{\PYZsq{}}\PY{p}{)}
\end{Verbatim}


    \begin{itemize}
\tightlist
\item
  Após remover as 260 instâncias, referentes aos valores ausentes
  \textbf{CIDADE\_USUARIO} e \textbf{BAIRRO\_USUARIO}, a base passou a
  ter 24092 linhas:
\end{itemize}

    \begin{Verbatim}[commandchars=\\\{\}]
{\color{incolor}In [{\color{incolor}508}]:} \PY{n}{novaBase}\PY{o}{.}\PY{n}{shape}
\end{Verbatim}


\begin{Verbatim}[commandchars=\\\{\}]
{\color{outcolor}Out[{\color{outcolor}508}]:} (24092, 18)
\end{Verbatim}
            
    \begin{itemize}
\tightlist
\item
  Antes de decidir o que fazer com o valores ausentes do atributo
  \textbf{AVALIACAO}, decidimos observar quais ocorrências para o mesmo
  e a quantidade:
\end{itemize}

    \begin{Verbatim}[commandchars=\\\{\}]
{\color{incolor}In [{\color{incolor}509}]:} \PY{k}{def} \PY{n+nf}{listar\PYZus{}ocorrencias\PYZus{}avaliacao}\PY{p}{(}\PY{n}{df}\PY{p}{)}\PY{p}{:}
              \PY{n}{r} \PY{o}{=} \PY{n}{pd}\PY{o}{.}\PY{n}{DataFrame}\PY{p}{(}\PY{n}{df}\PY{p}{[}\PY{l+s+s1}{\PYZsq{}}\PY{l+s+s1}{AVALIACAO}\PY{l+s+s1}{\PYZsq{}}\PY{p}{]}\PY{o}{.}\PY{n}{value\PYZus{}counts}\PY{p}{(}\PY{n}{dropna}\PY{o}{=}\PY{k+kc}{False}\PY{p}{)}\PY{p}{)}    
              \PY{n}{r}\PY{o}{.}\PY{n}{columns} \PY{o}{=} \PY{p}{[}\PY{l+s+s1}{\PYZsq{}}\PY{l+s+s1}{Quantidade}\PY{l+s+s1}{\PYZsq{}}\PY{p}{]}
              \PY{n}{total} \PY{o}{=} \PY{n}{r}\PY{p}{[}\PY{l+s+s1}{\PYZsq{}}\PY{l+s+s1}{Quantidade}\PY{l+s+s1}{\PYZsq{}}\PY{p}{]}\PY{o}{.}\PY{n}{sum}\PY{p}{(}\PY{p}{)}
              \PY{n}{r}\PY{p}{[}\PY{l+s+s1}{\PYZsq{}}\PY{l+s+s1}{\PYZpc{}}\PY{l+s+s1}{\PYZsq{}}\PY{p}{]} \PY{o}{=} \PY{n}{r}\PY{p}{[}\PY{l+s+s1}{\PYZsq{}}\PY{l+s+s1}{Quantidade}\PY{l+s+s1}{\PYZsq{}}\PY{p}{]}\PY{o}{.}\PY{n}{apply}\PY{p}{(}\PY{k}{lambda} \PY{n}{x}\PY{p}{:} \PY{p}{(}\PY{n}{x}\PY{o}{/}\PY{n}{total}\PY{p}{)}\PY{p}{)}
              \PY{k}{return} \PY{n}{r}
          
          \PY{n}{listar\PYZus{}ocorrencias\PYZus{}avaliacao}\PY{p}{(}\PY{n}{novaBase}\PY{p}{)}
\end{Verbatim}


\begin{Verbatim}[commandchars=\\\{\}]
{\color{outcolor}Out[{\color{outcolor}509}]:}       Quantidade         \%
          NaN        20503  0.851029
           5.0        2037  0.084551
           4.0         580  0.024074
           1.0         531  0.022041
           3.0         290  0.012037
           2.0         151  0.006268
\end{Verbatim}
            
    \begin{itemize}
\tightlist
\item
  Observando que 85\% das instâncias que possuem valores ausentes para o
  atributo \textbf{AVALIACAO}, a remoção dessas linhas não é uma opção.
  Como não podemos remover, decidimos atribuir o valor "0.0" para os
  valores ausentes do atributo \textbf{AVALIACAO}. A decisão se tornou
  mais consistente, porque o atributo \textbf{AVALIACAO}, mesmo sendo
  numérico, possui 5 valores (1.0, 2.0, 3.0, 4.0 e 5.0) bem separados e
  sequênciais. Com isso, inferimos que os valores ausentes seriam o
  valor "0.0":
\end{itemize}

    \begin{Verbatim}[commandchars=\\\{\}]
{\color{incolor}In [{\color{incolor}510}]:} \PY{n}{novaBase}\PY{p}{[}\PY{l+s+s1}{\PYZsq{}}\PY{l+s+s1}{AVALIACAO}\PY{l+s+s1}{\PYZsq{}}\PY{p}{]}\PY{o}{.}\PY{n}{fillna}\PY{p}{(}\PY{l+m+mf}{0.0}\PY{p}{,} \PY{n}{inplace}\PY{o}{=}\PY{k+kc}{True}\PY{p}{)}
          \PY{n}{listar\PYZus{}ocorrencias\PYZus{}avaliacao}\PY{p}{(}\PY{n}{novaBase}\PY{p}{)}
\end{Verbatim}


\begin{Verbatim}[commandchars=\\\{\}]
{\color{outcolor}Out[{\color{outcolor}510}]:}      Quantidade         \%
          0.0       20503  0.851029
          5.0        2037  0.084551
          4.0         580  0.024074
          1.0         531  0.022041
          3.0         290  0.012037
          2.0         151  0.006268
\end{Verbatim}
            
    \subsubsection{3. Valores inconsistentes}\label{valores-inconsistentes}

\begin{itemize}
\tightlist
\item
  Foi verificado se o atributo \textbf{DIA\_PEDIDO} confere com a dia do
  atributo \textbf{DATA\_PEDIDO}. Para isso utilizamos uma função da
  linguagem python, que informando a data, a mesma retorna o dia da
  semana:
\end{itemize}

    \begin{Verbatim}[commandchars=\\\{\}]
{\color{incolor}In [{\color{incolor}511}]:} \PY{c+c1}{\PYZsh{}Verificando se a Data do pedido informada na base,}
          \PY{c+c1}{\PYZsh{} confere com o Dia do pedido informado.}
          
          \PY{n}{df} \PY{o}{=} \PY{n}{novaBase}\PY{o}{.}\PY{n}{loc}\PY{p}{[} \PY{p}{:} \PY{p}{,} \PY{p}{[}\PY{l+s+s1}{\PYZsq{}}\PY{l+s+s1}{DATA\PYZus{}PEDIDO}\PY{l+s+s1}{\PYZsq{}}\PY{p}{,} \PY{l+s+s1}{\PYZsq{}}\PY{l+s+s1}{DIA\PYZus{}PEDIDO}\PY{l+s+s1}{\PYZsq{}}\PY{p}{]}\PY{p}{]}
          
          \PY{n}{df}\PY{p}{[}\PY{l+s+s1}{\PYZsq{}}\PY{l+s+s1}{Dia do pedido (Verificado pela função)}\PY{l+s+s1}{\PYZsq{}}\PY{p}{]} \PY{o}{=} \PY{n}{pd}\PY{o}{.}\PY{n}{DataFrame}\PY{p}{(}\PY{n}{df}\PY{o}{.}\PY{n}{apply}\PY{p}{(}\PY{k}{lambda} \PY{n}{x}\PY{p}{:} 
                                       \PY{n}{datetime}\PY{o}{.}\PY{n}{datetime}
                                       \PY{o}{.}\PY{n}{strptime}\PY{p}{(}\PY{n}{x}\PY{p}{[}\PY{l+s+s1}{\PYZsq{}}\PY{l+s+s1}{DATA\PYZus{}PEDIDO}\PY{l+s+s1}{\PYZsq{}}\PY{p}{]}\PY{p}{,}\PY{l+s+s1}{\PYZsq{}}\PY{l+s+s1}{\PYZpc{}}\PY{l+s+s1}{Y\PYZhy{}}\PY{l+s+s1}{\PYZpc{}}\PY{l+s+s1}{m\PYZhy{}}\PY{l+s+si}{\PYZpc{}d}\PY{l+s+s1}{\PYZsq{}}\PY{p}{)}
                                       \PY{o}{.}\PY{n}{strftime}\PY{p}{(}\PY{l+s+s1}{\PYZsq{}}\PY{l+s+s1}{\PYZpc{}}\PY{l+s+s1}{A}\PY{l+s+s1}{\PYZsq{}}\PY{p}{)}\PY{p}{,} \PY{n}{axis}\PY{o}{=}\PY{l+m+mi}{1}\PY{p}{)}\PY{p}{)}
          
          \PY{n}{df}\PY{o}{.}\PY{n}{head}\PY{p}{(}\PY{l+m+mi}{10}\PY{p}{)}
\end{Verbatim}


\begin{Verbatim}[commandchars=\\\{\}]
{\color{outcolor}Out[{\color{outcolor}511}]:}   DATA\_PEDIDO DIA\_PEDIDO Dia do pedido (Verificado pela função)
          0  2016-07-05     Sunday                                Tuesday
          1  2016-07-05     Sunday                                Tuesday
          2  2016-07-05     Sunday                                Tuesday
          3  2016-07-06     Monday                              Wednesday
          4  2016-07-07    Tuesday                               Thursday
          5  2016-07-08  Wednesday                                 Friday
          6  2016-07-09   Thursday                               Saturday
          7  2016-07-10     Friday                                 Sunday
          8  2016-07-12     Sunday                                Tuesday
          9  2016-07-15  Wednesday                                 Friday
\end{Verbatim}
            
    \begin{itemize}
\tightlist
\item
  Verificamos se existia alguma instância na base de dados, em que o dia
  do pedido verificado pela função era igual do atributo
  \textbf{DIA\_PEDIDO}:
\end{itemize}

    \begin{Verbatim}[commandchars=\\\{\}]
{\color{incolor}In [{\color{incolor}512}]:} \PY{n}{filtrado} \PY{o}{=} \PY{n}{df1}\PY{p}{[}\PY{p}{(}\PY{n}{df1}\PY{p}{[}\PY{l+s+s1}{\PYZsq{}}\PY{l+s+s1}{DIA\PYZus{}PEDIDO}\PY{l+s+s1}{\PYZsq{}}\PY{p}{]}
                          \PY{o}{==} \PY{n}{df1}\PY{p}{[}\PY{l+s+s1}{\PYZsq{}}\PY{l+s+s1}{Dia do pedido (Verificado pela função)}\PY{l+s+s1}{\PYZsq{}}\PY{p}{]}\PY{p}{)}\PY{p}{]}
          \PY{n+nb}{len}\PY{p}{(}\PY{n}{filtrado}\PY{p}{)}
\end{Verbatim}


\begin{Verbatim}[commandchars=\\\{\}]
{\color{outcolor}Out[{\color{outcolor}512}]:} 0
\end{Verbatim}
            
    \begin{itemize}
\tightlist
\item
  Devido a verificação, decidimos remover o atributo
  \textbf{DIA\_PEDIDO}:
\end{itemize}

    \begin{Verbatim}[commandchars=\\\{\}]
{\color{incolor}In [{\color{incolor}513}]:} \PY{n}{novaBase} \PY{o}{=} \PY{n}{novaBase}\PY{o}{.}\PY{n}{drop}\PY{p}{(}\PY{n}{columns}\PY{o}{=}\PY{p}{[}\PY{l+s+s1}{\PYZsq{}}\PY{l+s+s1}{DIA\PYZus{}PEDIDO}\PY{l+s+s1}{\PYZsq{}}\PY{p}{]}\PY{p}{)}
\end{Verbatim}


    \begin{itemize}
\tightlist
\item
  A base de dados passou a ter 17 atributos:
\end{itemize}

    \begin{Verbatim}[commandchars=\\\{\}]
{\color{incolor}In [{\color{incolor}514}]:} \PY{n}{novaBase}\PY{o}{.}\PY{n}{shape}
\end{Verbatim}


\begin{Verbatim}[commandchars=\\\{\}]
{\color{outcolor}Out[{\color{outcolor}514}]:} (24092, 17)
\end{Verbatim}
            
    \begin{itemize}
\tightlist
\item
  Observamos que para o atributo \textbf{TOTAL\_PEDIDO}, é dado pela
  soma de \textbf{VALOR\_PRODUTOS} e a \textbf{TAXA\_ENTREGA}.
  Adicionamos o atributo \textbf{STATUS} para efetuar outra verificação
  posteriormente, com isso resolvemos verificar nossa hipótese:
\end{itemize}

    \begin{Verbatim}[commandchars=\\\{\}]
{\color{incolor}In [{\color{incolor}515}]:} \PY{n}{df} \PY{o}{=} \PY{n}{novaBase}\PY{o}{.}\PY{n}{loc}\PY{p}{[}\PY{p}{:}\PY{p}{,} \PY{p}{[}\PY{l+s+s1}{\PYZsq{}}\PY{l+s+s1}{VALOR\PYZus{}PRODUTOS}\PY{l+s+s1}{\PYZsq{}}\PY{p}{,} 
                                  \PY{l+s+s1}{\PYZsq{}}\PY{l+s+s1}{TAXA\PYZus{}ENTREGA}\PY{l+s+s1}{\PYZsq{}}\PY{p}{,} 
                                  \PY{l+s+s1}{\PYZsq{}}\PY{l+s+s1}{TOTAL\PYZus{}PEDIDO}\PY{l+s+s1}{\PYZsq{}}\PY{p}{,}
                                   \PY{l+s+s1}{\PYZsq{}}\PY{l+s+s1}{STATUS}\PY{l+s+s1}{\PYZsq{}}\PY{p}{]}\PY{p}{]}
          \PY{n}{df}\PY{p}{[}\PY{l+s+s1}{\PYZsq{}}\PY{l+s+s1}{Total pedido (Verificado)}\PY{l+s+s1}{\PYZsq{}}\PY{p}{]} \PY{o}{=} \PY{n}{pd}\PY{o}{.}\PY{n}{DataFrame}\PY{p}{(}\PY{n}{df}
                                             \PY{o}{.}\PY{n}{apply}\PY{p}{(}\PY{k}{lambda} \PY{n}{x}\PY{p}{:} 
                                             \PY{p}{(}\PY{n}{x}\PY{p}{[}\PY{l+s+s1}{\PYZsq{}}\PY{l+s+s1}{VALOR\PYZus{}PRODUTOS}\PY{l+s+s1}{\PYZsq{}}\PY{p}{]} \PY{o}{+} 
                                              \PY{n}{x}\PY{p}{[}\PY{l+s+s1}{\PYZsq{}}\PY{l+s+s1}{TAXA\PYZus{}ENTREGA}\PY{l+s+s1}{\PYZsq{}}\PY{p}{]}\PY{p}{)}\PY{p}{,} 
                                                \PY{n}{axis}\PY{o}{=}\PY{l+m+mi}{1}\PY{p}{)}\PY{p}{)}
          
          \PY{n}{df}\PY{o}{.}\PY{n}{head}\PY{p}{(}\PY{l+m+mi}{10}\PY{p}{)}
\end{Verbatim}


\begin{Verbatim}[commandchars=\\\{\}]
{\color{outcolor}Out[{\color{outcolor}515}]:}    VALOR\_PRODUTOS  TAXA\_ENTREGA  TOTAL\_PEDIDO    STATUS  \textbackslash{}
          0            16.0           4.0          20.0  Entregue   
          1            28.0           4.0          32.0  Recusado   
          2            13.0           4.0          17.0  Recusado   
          3            11.5           4.0          15.5  Entregue   
          4            19.0           4.0          23.0  Recusado   
          5            42.0           4.0          46.0  Recusado   
          6             9.0           4.0          13.0  Recusado   
          7            29.0           4.0          33.0  Entregue   
          8            32.0           4.0          36.0  Entregue   
          9            12.0           4.0          16.0  Recusado   
          
             Total pedido (Verificado)  
          0                       20.0  
          1                       32.0  
          2                       17.0  
          3                       15.5  
          4                       23.0  
          5                       46.0  
          6                       13.0  
          7                       33.0  
          8                       36.0  
          9                       16.0  
\end{Verbatim}
            
    \begin{itemize}
\tightlist
\item
  Verificamos se existe alguém fora dessa regra:
\end{itemize}

    \begin{Verbatim}[commandchars=\\\{\}]
{\color{incolor}In [{\color{incolor}541}]:} \PY{n}{filtrado\PYZus{}diferente} \PY{o}{=} \PY{n}{df}\PY{p}{[}\PY{n}{np}\PY{o}{.}\PY{n}{isclose}\PY{p}{(}
              \PY{n}{df}\PY{p}{[}\PY{l+s+s1}{\PYZsq{}}\PY{l+s+s1}{TOTAL\PYZus{}PEDIDO}\PY{l+s+s1}{\PYZsq{}}\PY{p}{]}\PY{p}{,} 
              \PY{n}{df}\PY{p}{[}\PY{l+s+s1}{\PYZsq{}}\PY{l+s+s1}{Total pedido (Verificado)}\PY{l+s+s1}{\PYZsq{}}\PY{p}{]}\PY{p}{)} \PY{o}{==} \PY{k+kc}{False}\PY{p}{]}
          \PY{n+nb}{len}\PY{p}{(}\PY{n}{filtrado\PYZus{}diferente}\PY{p}{)}
\end{Verbatim}


\begin{Verbatim}[commandchars=\\\{\}]
{\color{outcolor}Out[{\color{outcolor}541}]:} 5
\end{Verbatim}
            
    \begin{itemize}
\tightlist
\item
  Pela verificação anterior, queremos saber se a diferença foi para mais
  ou para menos em relação ao \textbf{Total pedido (Verificado)}.
  Primeiro verificamos para mais:
\end{itemize}

    \begin{Verbatim}[commandchars=\\\{\}]
{\color{incolor}In [{\color{incolor}540}]:} \PY{n}{filtrado\PYZus{}para\PYZus{}mais} \PY{o}{=} \PY{n}{filtrado\PYZus{}diferente}\PY{p}{[}
              \PY{p}{(}\PY{n}{filtrado\PYZus{}diferente}\PY{p}{[}\PY{l+s+s1}{\PYZsq{}}\PY{l+s+s1}{TOTAL\PYZus{}PEDIDO}\PY{l+s+s1}{\PYZsq{}}\PY{p}{]}
              \PY{o}{\PYZgt{}} \PY{n}{filtrado\PYZus{}diferente}\PY{p}{[}\PY{l+s+s1}{\PYZsq{}}\PY{l+s+s1}{Total pedido (Verificado)}\PY{l+s+s1}{\PYZsq{}}\PY{p}{]}\PY{p}{)}\PY{p}{]}
          \PY{n+nb}{len}\PY{p}{(}\PY{n}{filtrado\PYZus{}para\PYZus{}mais}\PY{p}{)}
\end{Verbatim}


\begin{Verbatim}[commandchars=\\\{\}]
{\color{outcolor}Out[{\color{outcolor}540}]:} 0
\end{Verbatim}
            
    \begin{itemize}
\tightlist
\item
  Segundo verificamos para menos:
\end{itemize}

    \begin{Verbatim}[commandchars=\\\{\}]
{\color{incolor}In [{\color{incolor}539}]:} \PY{n}{filtrado\PYZus{}para\PYZus{}menos} \PY{o}{=} \PY{n}{filtrado\PYZus{}diferente}\PY{p}{[}
              \PY{p}{(}\PY{n}{filtrado\PYZus{}diferente}\PY{p}{[}\PY{l+s+s1}{\PYZsq{}}\PY{l+s+s1}{TOTAL\PYZus{}PEDIDO}\PY{l+s+s1}{\PYZsq{}}\PY{p}{]}
              \PY{o}{\PYZlt{}} \PY{n}{filtrado\PYZus{}diferente}\PY{p}{[}\PY{l+s+s1}{\PYZsq{}}\PY{l+s+s1}{Total pedido (Verificado)}\PY{l+s+s1}{\PYZsq{}}\PY{p}{]}\PY{p}{)}\PY{p}{]}
          \PY{n}{recusados\PYZus{}esperado} \PY{o}{=} \PY{n+nb}{len}\PY{p}{(}\PY{n}{filtrado\PYZus{}para\PYZus{}menos}\PY{p}{)}
          \PY{n}{recusados\PYZus{}esperado}
\end{Verbatim}


\begin{Verbatim}[commandchars=\\\{\}]
{\color{outcolor}Out[{\color{outcolor}539}]:} 5
\end{Verbatim}
            
    \begin{itemize}
\tightlist
\item
  Inferimos que para um pedido/instância, que tem como valor do atributo
  \textbf{TOTAL\_PEDIDO} inferior ao \textbf{Total pedido (Verificado)},
  o \textbf{STATUS} do pedido deveria ser \textbf{Recusado}.
  Verificamos:
\end{itemize}

    \begin{Verbatim}[commandchars=\\\{\}]
{\color{incolor}In [{\color{incolor}519}]:} \PY{n}{recusados} \PY{o}{=} \PY{n}{filtrado\PYZus{}para\PYZus{}menos}\PY{p}{[}\PY{p}{(}\PY{n}{filtrado\PYZus{}para\PYZus{}menos}\PY{p}{[}\PY{l+s+s1}{\PYZsq{}}\PY{l+s+s1}{STATUS}\PY{l+s+s1}{\PYZsq{}}\PY{p}{]} \PY{o}{==} \PY{l+s+s1}{\PYZsq{}}\PY{l+s+s1}{Recusado}\PY{l+s+s1}{\PYZsq{}}\PY{p}{)}\PY{p}{]}
          \PY{n+nb}{len}\PY{p}{(}\PY{n}{recusados}\PY{p}{)}
\end{Verbatim}


\begin{Verbatim}[commandchars=\\\{\}]
{\color{outcolor}Out[{\color{outcolor}519}]:} 4
\end{Verbatim}
            
    \begin{Verbatim}[commandchars=\\\{\}]
{\color{incolor}In [{\color{incolor}520}]:} \PY{n}{recusados}
\end{Verbatim}


\begin{Verbatim}[commandchars=\\\{\}]
{\color{outcolor}Out[{\color{outcolor}520}]:}        VALOR\_PRODUTOS  TAXA\_ENTREGA  TOTAL\_PEDIDO    STATUS  \textbackslash{}
          11793             0.0           3.0           0.0  Recusado   
          11804             0.0           3.0           0.0  Recusado   
          11805             0.0           3.0           0.0  Recusado   
          19900             0.0           4.0           0.0  Recusado   
          
                 Total pedido (Verificado)  
          11793                        3.0  
          11804                        3.0  
          11805                        3.0  
          19900                        4.0  
\end{Verbatim}
            
    \begin{itemize}
\tightlist
\item
  Pela execução anterior, observamos que existe uma instância com o
  atributo \textbf{STATUS} igual a \textbf{Entregue} e o
  \textbf{TOTAL\_PEDIDO} menor do que o \textbf{Total pedido
  (Verificado)}:
\end{itemize}

    \begin{Verbatim}[commandchars=\\\{\}]
{\color{incolor}In [{\color{incolor}521}]:} \PY{n}{entregue} \PY{o}{=} \PY{n}{filtrado\PYZus{}para\PYZus{}menos}\PY{p}{[}
              \PY{p}{(}\PY{n}{filtrado\PYZus{}para\PYZus{}menos}\PY{p}{[}\PY{l+s+s1}{\PYZsq{}}\PY{l+s+s1}{STATUS}\PY{l+s+s1}{\PYZsq{}}\PY{p}{]} \PY{o}{==} \PY{l+s+s1}{\PYZsq{}}\PY{l+s+s1}{Entregue}\PY{l+s+s1}{\PYZsq{}}\PY{p}{)}\PY{p}{]}
          \PY{n+nb}{len}\PY{p}{(}\PY{n}{entregue}\PY{p}{)}
\end{Verbatim}


\begin{Verbatim}[commandchars=\\\{\}]
{\color{outcolor}Out[{\color{outcolor}521}]:} 1
\end{Verbatim}
            
    \begin{Verbatim}[commandchars=\\\{\}]
{\color{incolor}In [{\color{incolor}522}]:} \PY{n}{entregue}
\end{Verbatim}


\begin{Verbatim}[commandchars=\\\{\}]
{\color{outcolor}Out[{\color{outcolor}522}]:}        VALOR\_PRODUTOS  TAXA\_ENTREGA  TOTAL\_PEDIDO    STATUS  \textbackslash{}
          22722            27.9          1.99          6.89  Entregue   
          
                 Total pedido (Verificado)  
          22722                      29.89  
\end{Verbatim}
            
    \begin{itemize}
\tightlist
\item
  Podemos considerar essa instância com valor inconsistente, portanto,
  decidimos remover a mesma da base de dados:
\end{itemize}

    \begin{Verbatim}[commandchars=\\\{\}]
{\color{incolor}In [{\color{incolor}523}]:} \PY{n}{i} \PY{o}{=} \PY{n}{entregue}\PY{o}{.}\PY{n}{index}
          \PY{n}{novaBase} \PY{o}{=} \PY{n}{novaBase}\PY{o}{.}\PY{n}{drop}\PY{p}{(}\PY{n}{i}\PY{p}{)}
          \PY{n}{novaBase}\PY{o}{.}\PY{n}{shape}
\end{Verbatim}


\begin{Verbatim}[commandchars=\\\{\}]
{\color{outcolor}Out[{\color{outcolor}523}]:} (24091, 17)
\end{Verbatim}
            
    \subsubsection{4. Outliers}\label{outliers}

\begin{itemize}
\tightlist
\item
  Utilizando a visualização e a análise do gráfico \textbf{boxplot}
  exibido neste relatório, observamos que existem outliers nos valores
  dos atributos \textbf{VALOR\_PRODUTOS} e \textbf{TOTAL\_PEDIDOS}:
\end{itemize}

    \begin{Verbatim}[commandchars=\\\{\}]
{\color{incolor}In [{\color{incolor}524}]:} \PY{n}{boxplot\PYZus{}analise}\PY{p}{(}\PY{p}{)}
\end{Verbatim}


    \begin{center}
    \adjustimage{max size={0.9\linewidth}{0.9\paperheight}}{output_65_0.png}
    \end{center}
    { \hspace*{\fill} \\}
    
    \begin{itemize}
\tightlist
\item
  Decidimos fazer a eliminação de valores acima de 60, tanto para o
  atributo \textbf{VALOR\_PRODUTOS} e \textbf{TOTAL\_PEDIDO}, com base
  no gráfico:
\end{itemize}

    \begin{Verbatim}[commandchars=\\\{\}]
{\color{incolor}In [{\color{incolor}538}]:} \PY{k}{def} \PY{n+nf}{remover\PYZus{}outliers}\PY{p}{(}\PY{n}{novaBase}\PY{p}{,} \PY{n}{valor}\PY{p}{)}\PY{p}{:}
              \PY{k}{return} \PY{n}{novaBase}\PY{o}{.}\PY{n}{drop}\PY{p}{(}
                  \PY{n}{novaBase}\PY{p}{[}\PY{p}{(}\PY{n}{novaBase}\PY{p}{[}\PY{l+s+s1}{\PYZsq{}}\PY{l+s+s1}{VALOR\PYZus{}PRODUTOS}\PY{l+s+s1}{\PYZsq{}}\PY{p}{]} \PY{o}{\PYZgt{}} \PY{n}{valor}\PY{p}{)} 
                  \PY{o}{\PYZam{}} \PY{p}{(}\PY{n}{novaBase}\PY{p}{[}\PY{l+s+s1}{\PYZsq{}}\PY{l+s+s1}{TOTAL\PYZus{}PEDIDO}\PY{l+s+s1}{\PYZsq{}}\PY{p}{]} \PY{o}{\PYZgt{}} \PY{n}{valor}\PY{p}{)}\PY{p}{]}\PY{o}{.}\PY{n}{index}\PY{p}{)}
              
          \PY{n}{novaBase} \PY{o}{=} \PY{n}{remover\PYZus{}outliers}\PY{p}{(}\PY{n}{novaBase}\PY{p}{,} \PY{l+m+mi}{60}\PY{p}{)}
\end{Verbatim}


    \begin{Verbatim}[commandchars=\\\{\}]
{\color{incolor}In [{\color{incolor}526}]:} \PY{n}{novaBase}\PY{o}{.}\PY{n}{shape}
\end{Verbatim}


\begin{Verbatim}[commandchars=\\\{\}]
{\color{outcolor}Out[{\color{outcolor}526}]:} (23297, 17)
\end{Verbatim}
            
    \begin{itemize}
\tightlist
\item
  Verificamos se nos \textbf{boxplot} existem ainda outliers:
\end{itemize}

    \begin{Verbatim}[commandchars=\\\{\}]
{\color{incolor}In [{\color{incolor}527}]:} \PY{k}{def} \PY{n+nf}{boxplot\PYZus{}novaBase}\PY{p}{(}\PY{n}{novaBase}\PY{p}{)}\PY{p}{:}
              \PY{n}{fig} \PY{o}{=} \PY{n}{plt}\PY{o}{.}\PY{n}{figure}\PY{p}{(}\PY{n}{figsize}\PY{o}{=}\PY{p}{(}\PY{l+m+mi}{15}\PY{p}{,}\PY{l+m+mi}{20}\PY{p}{)}\PY{p}{)}
              \PY{n}{ax5} \PY{o}{=} \PY{n}{fig}\PY{o}{.}\PY{n}{add\PYZus{}subplot}\PY{p}{(}\PY{l+m+mi}{221}\PY{p}{)}
              \PY{n}{grafico} \PY{o}{=} \PY{n}{novaBase}\PY{p}{[}\PY{p}{[}\PY{l+s+s1}{\PYZsq{}}\PY{l+s+s1}{VALOR\PYZus{}PRODUTOS}\PY{l+s+s1}{\PYZsq{}}\PY{p}{]}\PY{p}{]}\PY{o}{.}\PY{n}{boxplot}\PY{p}{(}\PY{n}{figsize}\PY{o}{=}\PY{p}{(}\PY{l+m+mi}{20}\PY{p}{,}\PY{l+m+mi}{5}\PY{p}{)}\PY{p}{,} 
                                                 \PY{n}{fontsize}\PY{o}{=} \PY{n}{fontsize}\PY{p}{,}   \PY{n}{ax}\PY{o}{=}\PY{n}{ax5}\PY{p}{)}
              \PY{n}{plt}\PY{o}{.}\PY{n}{title}\PY{p}{(}\PY{l+s+s1}{\PYZsq{}}\PY{l+s+s1}{Valor dos produtos}\PY{l+s+s1}{\PYZsq{}}\PY{p}{)}
          
          
              \PY{n}{ax7} \PY{o}{=} \PY{n}{fig}\PY{o}{.}\PY{n}{add\PYZus{}subplot}\PY{p}{(}\PY{l+m+mi}{222}\PY{p}{)}
              \PY{n}{grafico} \PY{o}{=} \PY{n}{novaBase}\PY{p}{[}\PY{p}{[}\PY{l+s+s1}{\PYZsq{}}\PY{l+s+s1}{TOTAL\PYZus{}PEDIDO}\PY{l+s+s1}{\PYZsq{}}\PY{p}{]}\PY{p}{]}\PY{o}{.}\PY{n}{boxplot}\PY{p}{(}\PY{n}{figsize}\PY{o}{=}\PY{p}{(}\PY{l+m+mi}{20}\PY{p}{,}\PY{l+m+mi}{5}\PY{p}{)}\PY{p}{,} 
                                                   \PY{n}{fontsize}\PY{o}{=} \PY{n}{fontsize}\PY{p}{,}   \PY{n}{ax}\PY{o}{=}\PY{n}{ax7}\PY{p}{)}
              \PY{n}{plt}\PY{o}{.}\PY{n}{title}\PY{p}{(}\PY{l+s+s1}{\PYZsq{}}\PY{l+s+s1}{Total de pedidos}\PY{l+s+s1}{\PYZsq{}}\PY{p}{)}
              
          \PY{n}{boxplot\PYZus{}novaBase}\PY{p}{(}\PY{n}{novaBase}\PY{p}{)}
\end{Verbatim}


    \begin{center}
    \adjustimage{max size={0.9\linewidth}{0.9\paperheight}}{output_70_0.png}
    \end{center}
    { \hspace*{\fill} \\}
    
    \begin{itemize}
\tightlist
\item
  Decidimos fazer novamente a eliminação de valores acima de 55, tanto
  para o atributo \textbf{VALOR\_PRODUTOS} e \textbf{TOTAL\_PEDIDO}, com
  base no gráfico acima:
\end{itemize}

    \begin{Verbatim}[commandchars=\\\{\}]
{\color{incolor}In [{\color{incolor}528}]:} \PY{n}{novaBase} \PY{o}{=} \PY{n}{remover\PYZus{}outliers}\PY{p}{(}\PY{n}{novaBase}\PY{p}{,} \PY{l+m+mi}{55}\PY{p}{)}
\end{Verbatim}


    \begin{Verbatim}[commandchars=\\\{\}]
{\color{incolor}In [{\color{incolor}529}]:} \PY{n}{novaBase}\PY{o}{.}\PY{n}{shape}
\end{Verbatim}


\begin{Verbatim}[commandchars=\\\{\}]
{\color{outcolor}Out[{\color{outcolor}529}]:} (22908, 17)
\end{Verbatim}
            
    \begin{Verbatim}[commandchars=\\\{\}]
{\color{incolor}In [{\color{incolor}530}]:} \PY{n}{boxplot\PYZus{}novaBase}\PY{p}{(}\PY{n}{novaBase}\PY{p}{)}
\end{Verbatim}


    \begin{center}
    \adjustimage{max size={0.9\linewidth}{0.9\paperheight}}{output_74_0.png}
    \end{center}
    { \hspace*{\fill} \\}
    
    \begin{itemize}
\tightlist
\item
  Novamente a eliminação de valores acima de 55, apenas para o atributo
  \textbf{TOTAL\_PEDIDO}, com base no gráfico acima:
\end{itemize}

    \begin{Verbatim}[commandchars=\\\{\}]
{\color{incolor}In [{\color{incolor}531}]:} \PY{k}{def} \PY{n+nf}{remover\PYZus{}outliers\PYZus{}total\PYZus{}pedido}\PY{p}{(}\PY{n}{novaBase}\PY{p}{,} \PY{n}{valor}\PY{p}{)}\PY{p}{:}
              \PY{k}{return} \PY{n}{novaBase}\PY{o}{.}\PY{n}{drop}\PY{p}{(}\PY{n}{novaBase}\PY{p}{[}\PY{p}{(}\PY{n}{novaBase}\PY{p}{[}\PY{l+s+s1}{\PYZsq{}}\PY{l+s+s1}{TOTAL\PYZus{}PEDIDO}\PY{l+s+s1}{\PYZsq{}}\PY{p}{]} \PY{o}{\PYZgt{}} \PY{n}{valor}\PY{p}{)}\PY{p}{]}\PY{o}{.}\PY{n}{index}\PY{p}{)}
          \PY{n}{novaBase} \PY{o}{=} \PY{n}{remover\PYZus{}outliers\PYZus{}total\PYZus{}pedido}\PY{p}{(}\PY{n}{novaBase}\PY{p}{,} \PY{l+m+mi}{55}\PY{p}{)}
\end{Verbatim}


    \begin{Verbatim}[commandchars=\\\{\}]
{\color{incolor}In [{\color{incolor}532}]:} \PY{n}{novaBase}\PY{o}{.}\PY{n}{shape}
\end{Verbatim}


\begin{Verbatim}[commandchars=\\\{\}]
{\color{outcolor}Out[{\color{outcolor}532}]:} (22676, 17)
\end{Verbatim}
            
    \begin{Verbatim}[commandchars=\\\{\}]
{\color{incolor}In [{\color{incolor}533}]:} \PY{n}{boxplot\PYZus{}novaBase}\PY{p}{(}\PY{n}{novaBase}\PY{p}{)}
\end{Verbatim}


    \begin{center}
    \adjustimage{max size={0.9\linewidth}{0.9\paperheight}}{output_78_0.png}
    \end{center}
    { \hspace*{\fill} \\}
    
    \begin{itemize}
\tightlist
\item
  Outra vez a eliminação considerando valores acima de 52, tanto para o
  atributo \textbf{VALOR\_PRODUTOS} e \textbf{TOTAL\_PEDIDO}, logo em
  seguida apenas para \textbf{TOTAL\_PEDIDO}, com base no gráfico acima:
\end{itemize}

    \begin{Verbatim}[commandchars=\\\{\}]
{\color{incolor}In [{\color{incolor}534}]:} \PY{n}{novaBase} \PY{o}{=} \PY{n}{remover\PYZus{}outliers}\PY{p}{(}\PY{n}{novaBase}\PY{p}{,} \PY{l+m+mi}{52}\PY{p}{)}
          \PY{n}{novaBase} \PY{o}{=} \PY{n}{remover\PYZus{}outliers\PYZus{}total\PYZus{}pedido}\PY{p}{(}\PY{n}{novaBase}\PY{p}{,} \PY{l+m+mi}{52}\PY{p}{)}
\end{Verbatim}


    \begin{Verbatim}[commandchars=\\\{\}]
{\color{incolor}In [{\color{incolor}535}]:} \PY{n}{novaBase}\PY{o}{.}\PY{n}{shape}
\end{Verbatim}


\begin{Verbatim}[commandchars=\\\{\}]
{\color{outcolor}Out[{\color{outcolor}535}]:} (22429, 17)
\end{Verbatim}
            
    \begin{Verbatim}[commandchars=\\\{\}]
{\color{incolor}In [{\color{incolor}536}]:} \PY{n}{boxplot\PYZus{}novaBase}\PY{p}{(}\PY{n}{novaBase}\PY{p}{)}
\end{Verbatim}


    \begin{center}
    \adjustimage{max size={0.9\linewidth}{0.9\paperheight}}{output_82_0.png}
    \end{center}
    { \hspace*{\fill} \\}
    
    \begin{itemize}
\tightlist
\item
  A remoção foi satisfatória, não é preciso remover mais. Portanto, após
  a execução do pré-processamento ainda possuímos,boa quantidade de
  dados:
\end{itemize}

    \begin{Verbatim}[commandchars=\\\{\}]
{\color{incolor}In [{\color{incolor}537}]:} \PY{n}{novaBase}\PY{o}{.}\PY{n}{shape}
\end{Verbatim}


\begin{Verbatim}[commandchars=\\\{\}]
{\color{outcolor}Out[{\color{outcolor}537}]:} (22429, 17)
\end{Verbatim}
            

    % Add a bibliography block to the postdoc
    
    
    
    \end{document}
